
\section{OpenType features}

The font provides a number of ligatures, which are made by inserting the Zero Width Joiner (U+200D) between two characters. Here is a list of ligatures:

\begin{tabular}{lcc}
Name	& Sequence	& Apperance \\
\hline
Ligature A-U	& U+0430 U+200D U+0443	& {\glyphfont{\large а‍у}}	\\
Ligature El-U	& U+043B U+200D U+0443 & {\glyphfont{\large л‍у}}	\\
Ligature A-Izhitsa & U+0430 U+200D U+0475	& {\glyphfont{\large а‍ѵ}}	\\
Ligature El-Izhitsa & U+043B U+200D U+075 & {\glyphfont{\large л‍ѵ}}	\\
Ligature Te-Ve	& U+0442 U+200D U+0432	& {\glyphfont{\large т‍в}}	\\
Ligature Er-Yat	& U+0440 U+200D U+0463 & {\glyphfont{\large р‍ѣ}}	\\
\hline
\end{tabular}
\\

In OpenType, the following Stylistic alternatives are defined:
\newfontfamily{\salt}[Alternate=0]{Fedorovsk Unicode}
\newfontfamily{\salta}[Alternate=1]{Fedorovsk Unicode}
\newfontfamily{\saltb}[Alternate=2]{Fedorovsk Unicode}
\newfontfamily{\saltc}[Alternate=3]{Fedorovsk Unicode}

\begin{tabular}{lccccc}
	& Base Form	& Salt=0	& Salt=1	& Salt=2	& Salt=3 \\
\hline
U+0404	& {\glyphfont{\large Є}} & {\salt\large Є} \\
U+0426	& {\glyphfont{\large Ц}} & {\salt\large Ц} \\
U+0491	& {\glyphfont{\large ґ}} & {\salt\large ґ} \\
U+1F545	& {\glyphfont{\large 🕅 }}	& {\salt\large 🕅} & {\salta\large 🕅} & {\saltb\large 🕅} & {\saltc\large 🕅} \\
U+0463 U+0486	& {\glyphfont{\large ѣ҆}} & {\salt\large ѣ҆}  \\
U+0463 U+0300	& {\glyphfont{\large ѣ̀}} & {\salt\large ѣ̀} & {\salta\large ѣ̀} \\
U+0463 U+0301	& {\glyphfont{\large ѣ́}} & {\salt\large ѣ́} & {\salta\large ѣ́} \\
U+0463 U+0311	& {\glyphfont{\large ѣ̑}} & {\salt\large ѣ̑} & {\salta\large ѣ̑} \\
U+0463 U+0486 U+0301	& {\glyphfont{\large ѣ҆́}} & {\salt\large ѣ҆́}  \\
U+A64B U+0486	& {\glyphfont{\large ꙋ҆}} & {\salt\large ꙋ҆}  \\
U+A64B U+0300	& {\glyphfont{\large ꙋ̀}} & {\salt\large ꙋ̀} & {\salta\large ꙋ̀} \\
U+A64B U+0301	& {\glyphfont{\large ꙋ́}} & {\salt\large ꙋ́} & {\salta\large ꙋ́} \\
U+A64B U+0311	& {\glyphfont{\large ꙋ̑}} & {\salt\large ꙋ̑} & {\salta\large ꙋ̑} & {\saltb\large ꙋ̑} \\
U+A64B U+0486 U+0301	& {\glyphfont{\large ꙋ҆́}} & {\salt\large ꙋ҆́}  \\
\hline
\end{tabular}

Additionally, three stylistic sets have been defined in the font. Stylistic set 1 (``Right-side accents'') positions the accents over the Yat and the Uk on the right side and Stylistic set 2 (``Left-side accents'') positions the accents over the Yat and the Uk on the left side. These stylistic sets are useful when a text uses one of these positionings throughout. Stylistic set 10 (``Equal Baseline Variants'') sets the capital letters on the same baseline as the lowercase letters (useful for working with the font in an academic context where the traditionally lowered baseline of uppercase letters can cause vertical spacing issues when working with text that is both in Latin and Cyrillic scripts). Here is an example:

\newfontfamily{\base}[StylisticSet=10]{Fedorovsk Unicode}
{\Large \base
In pre-Nikonian editions of the Flowery Triodion, we find the Paschal Troparion in the following redaction: Хрⷭ҇то́съ вᲂскр҃се и҆з̾ ме́ртвыхъ, сме́ртїю на сме́рть настꙋпѝ.
}

The use of contextual substitution to select the narrow o is now DEPRECATED. The narrow O must be encoded separately as U+1C82.

\section{SIL Graphite features}

{\textbf Please note} that there is presently no Kerning in the Graphite version of the font. The stylistic alternatives of the Mark's Chapter symbol, the Letter Ge with Upturn, and the letters Ye and Tse have been duplicated as Graphite features in the TTF version of the font, with names ``Symbol for Mark's Chapter'', ``Ye'', ``Tse'', and ``Ghe'', respectively.
\newfontfamily{\graph}[Renderer=Graphite]{Fedorovsk Unicode TT}
\newfontfamily{\graphA}[Renderer=Graphite, RawFeature={Symbol for Mark's Chapter=Alternative 1}]{Fedorovsk Unicode TT}
\newfontfamily{\graphB}[Renderer=Graphite, RawFeature={Symbol for Mark's Chapter=Alternative 2}]{Fedorovsk Unicode TT}
\newfontfamily{\graphC}[Renderer=Graphite, RawFeature={Symbol for Mark's Chapter=Alternative 3}]{Fedorovsk Unicode TT}
\newfontfamily{\graphD}[Renderer=Graphite, RawFeature={Symbol for Mark's Chapter=Alternative 4}]{Fedorovsk Unicode TT}
\newfontfamily{\graphYe}[Renderer=Graphite, RawFeature={Ye=Alternative 1}]{Fedorovsk Unicode TT}
\newfontfamily{\graphTse}[Renderer=Graphite, RawFeature={Tse=Alternative 1}]{Fedorovsk Unicode TT}
\newfontfamily{\graphGhe}[Renderer=Graphite, RawFeature={Ghe=Alternative 1}]{Fedorovsk Unicode TT}

\begin{tabular}{lccccc}
	& Base form	& Alternative 1	& Alternative 2	& Alternative 3	& Alternative 4 \\
\hline
U+0404	& {\graph{\large Є }} & {\graphYe\large Є} \\
U+0426	& {\graph{\large Ц}} & {\graphTse\large Ц} \\
U+0491	& {\graph{\large ґ}} & {\graphGhe\large ґ} \\
U+1F545	& {\graph{\large 🕅 }}	& {\graphA{\large 🕅}} & {\graphB{\large 🕅}} & {\graphC{\large 🕅}} & {\graphD{\large 🕅}} \\
\hline
\end{tabular}

Two additional SIL Graphite features are defined: ``Accent Positions'', with values ``left'' and ``right'', which mimicks the behavior of stylistic sets 1 and 2; 
and ``Equal Baseline'' (with value ``yes''), which mimicks the behavior of stylistic set 10. Here is an example:

\newfontfamily{\raised}[Renderer=Graphite, RawFeature={Equal Baseline=Yes}]{Fedorovsk Unicode TT}

{\Large \raised
In pre-Nikonian editions of the Flowery Triodion, we find the Paschal Troparion in the following redaction: Хрⷭ҇то́съ вᲂскр҃се и҆з̾ ме́ртвыхъ, сме́ртїю на сме́рть настꙋпѝ.
}

\section{Sample Texts}
\newfontfamily{\right}[StylisticSet=1]{Fedorovsk Unicode}

\subsection{Apostol of Ivan Fedorov}

{\Large \right
\textcolor{red}{П}е́рвᲂе ᲂу҆́бо︀ сло́во︀ сᲂтвᲂри́хъ о҆ всѣ́хъ , ѽ , ѳео҆́филе , о҆ ниⷯже начѧ́тъ і︮с︯ , твᲂри́тиже и҆ ᲂу҆чи́ти . д︀о︀ него́же дн҃е , запᲂвѣ́д︀авъ а҆пⷭ҇лᲂмъ дх҃ᲂмъ ст҃ыⷨ , и҆́хже и҆ꙁбра̀ вᲂзнесе́сѧ . преⷣ ни́миже и҆ пᲂста́ви себѐ жи́ва по страд︀а́нїи свᲂе҆́мъ . во︀ мно́зехъ и҆́стинныхъ зна́менїи҆хъ . дн҃ьми четы́ридесѧтьми ꙗ҆влѧ́ꙗсѧ и҆́мъ и҆ гл҃ѧ ꙗ҆́же о҆ црⷭ҇твїи бж҃їи . сни́миже и҆ ꙗ҆д︀ы̀ , пᲂвелѣва́ше и҆́мъ ѿ і҆е҆рᲂсали́ма не ѿлꙋча́тисѧ . но̑ жда́ти о҆бѣтᲂва́нїе ѿч︮е︯е , е҆́же слы́шасте ѿ́ менѐ . ꙗ҆́кѡ і҆ѡ҃а́ннъ ᲂу҆́бо︀ крⷭ҇ти́лъ е҆́сть вᲂдо́ю . вы́же и҆́мате крести́тисѧ дх҃ᲂмъ ст҃ы́мъ , не по мно́ꙁѣхъ си́хъ д︀︮н︯еⷯ .

}

\subsection{Flowery Triodion}

{\Large \glyphfont

\textcolor{red}{стⷯры па́сцѣ . гла́съ , є҃ .} Д\textcolor{red}{а вᲂскрⷭ҇нетъ бг҃ъ ,꙳ и҆ разы́дꙋтсѧ вразѝ є҆гѡ̀ .}
Па́сха сщ҃е́ннаѧ на́мъ дне́сь пᲂказа́сѧ , па́сха но́ва ст҃а́ѧ , па́сха таи́нственнаѧ , па́сха всечестна́ѧ , па́сха хрⷭ҇та̀ и҆зба́вителѧ , па́сха непᲂро́чнаѧ , па́сха вели́каѧ , па́сха вѣ́рнымъ , па́сха двѣ́ри ра́йскїѧ на́мъ ѿверза́ющаѧ , па́сха всѣ́хъ ѡ҆сщ҃а́ющаѧ вѣ́рныхъ .
}
