\section{Private Use Area}
This font places some characters in the Private Use Area (PUA). For the general PUA mappings, please see the Ponomar PUA Allocation Policy at \url{http://www.ponomar.net/files/pua_policy.pdf}.

\section{Ligatures}

The font provides a number of ligatures, which are made by inserting the Zero Width Joiner (U+200D) between two characters. Here is a list of ligatures:
\newfontfamily{\graph}[Renderer=Graphite]{Menaion Unicode TT}
%\newfontfamily{\graph}{Menaion Unicode}

\begin{tabular}{lcc}
Name	& Sequence	& Apperance \\
\hline
Small Ligature I-Ye &	U+0438 U+200D U+0435 	& {\glyphfont{\large и‍е }} \\
Small Ligature En-I	&	U+043d U+200D U+0438 	& {\graph{\large н‍и }} \\
Small Ligature En-Small Yus	& U+043d U+200D U+0467 	& {\glyphfont{\large н‍ѧ }} \\
Small Ligature Es-Ve	&	U+0441 U+200D U+0432 	& {\glyphfont{\large с‍в }} \\
Small Ligature Te-Er	&	U+0442 U+200D U+0440 	& {\glyphfont{\large т‍р }} \\
Capital Litagure A-U	& 	U+0410 U+200D U+0423 	& {\glyphfont{\large А‍У }} \\
Small Ligature A-U	&	U+0430 U+200D U+0443 	& {\glyphfont{\large а‍у }} \\
Small Ligature A-Te		&	U+0430 U+200D U+0442 	& {\glyphfont{\large а‍т }} \\
Capital Ligature I-Ye	&	U+0418 U+200D U+0415 	& {\glyphfont{\large И‍Е }} \\
Capital Ligature El-Ge		&	U+041b U+200D U+0413 	& {\glyphfont{\large Л‍Г }} \\
Small Ligature El-Ge		&	U+043b U+200D U+0433 	& {\glyphfont{\large л‍г }} \\
Capital Ligature En-I	&	U+041d U+200D U+0418 	& {\glyphfont{\large Н‍И }} \\
Capital Ligature En-Small Yus	&	U+041d U+200D U+0466 	& {\graph{\large Н‍Ѧ }} \\
Capital Ligature Es-Ve		&	U+0421 U+200D U+0412 	& {\glyphfont{\large С‍В }} \\
Small Ligature Te-Yat		&	U+0442 U+200D U+0463 	& {\glyphfont{\large т‍ѣ }} \\
Capital Ligature Te-Ve	&	U+0422 U+200D U+0412	& {\glyphfont{\large Т‍В }} \\
Small Ligature Te-Ve		&	U+0442 U+200D U+0432	& {\glyphfont{\large т‍в }} \\
Capital Ligature Te-I		&	U+0422 U+200D U+0418 	& {\glyphfont{\large Т‍И }} \\
Small Ligature Te-I		&	U+0442 U+200D U+0438 	& {\glyphfont{\large т‍и }} \\
Capital Ligature Te-Er		&	U+0422 U+200D U+0420 	& {\glyphfont{\large Т‍Р }} \\
Ligature Capital A-Small Te	&	U+0410 U+200D U+0442 	& {\glyphfont{\large А‍т }} \\
Capital Ligature Te-Soft Sign	&	U+0422 U+200D U+042c 	& {\glyphfont{\large Т‍Ь }} \\
Small Ligature Te-Soft Sign	&	U+0442 U+200D U+044c 	& ‍{\graph{\large т‍ь }} \\
Small Ligature Te-A		&	U+0442 U+200D U+0430 	& {\glyphfont{\large т‍а }} \\
\hline
\end{tabular}
\\

{\textbf Please note also} that there is presently no Kerning in the Graphite version of the font. 

\section{Sample Texts}

\subsection{Ostromir Gospel (Cyrillic)}
\begin{tabular}{lr}
 1& {\Large \glyphfont    Искони бѣ слово } \\
 2& {\Large \glyphfont    и слово бѣ отъ  } \\
 3& {\Large \glyphfont   б҃а и б҃ъ бѣ} \\ 
 4& {\Large \glyphfont    слово  𝀏̃  се бѣ} \\ 
 5& {\Large \glyphfont    искони оу} \\ 
 6& {\Large \glyphfont    б҃а  ⁘  и тѣмь в̇са бꙑ-} \\ 
 7& {\Large \glyphfont    шѧ  𝀏̃  и беꙁ него ни-} \\ 
 8& {\Large \glyphfont    чьтоже не бꙑсть  ·} \\ 
 9& {\Large \glyphfont   ѥже бꙑсть  𝀏̃  въ то-} \\ 
10& {\Large \glyphfont    мь животъ бѣ  ·  и} \\ 
 1& {\Large \glyphfont    животъ бѣ свѣтъ} \\ 
 2& {\Large \glyphfont    чловѣкомъ  𝀏̃  и свѣ-} \\ 
 3& {\Large \glyphfont    тъ въ тьмѣ свьти-} \\ 
 4& {\Large \glyphfont    тьсѧ  ·  и тьма ѥго} \\ 
 5& {\Large \glyphfont    не обѧтъ  𝀏̃  бꙑсть} \\ 
 6& {\Large \glyphfont    члв҃къ посъланъ} \\ 
 7& {\Large \glyphfont    отъ б҃а  ·  имѧ ѥмоу} \\ 
 8& {\Large \glyphfont    иоанъ  𝀏̃  тъ приде} \\ 
 9& {\Large \glyphfont    въ съвѣдѣтель-} \\ 
10& {\Large \glyphfont    ство  ·  да съвѣдѣте-} \\ 
2.2  1& {\Large \glyphfont    льствоуѥть о свѣ-} \\ 
 2& {\Large \glyphfont    тѣ  𝀏̃  да вьси вѣрѫ} \\ 
 3& {\Large \glyphfont    имѫть имь  ⁘  не бѣ} \\ 
 4& {\Large \glyphfont    тъ свѣтъ  ⁘  нъ да} \\ 
 5& {\Large \glyphfont    съвѣдѣтельствоу-} \\ 
 6& {\Large \glyphfont    ѥть о свѣтѣ  𝀏̃̑ бѣ} \\ 
 7& {\Large \glyphfont    свѣтъ истиньнꙑ-} \\ 
 8& {\Large \glyphfont    и  ·  иже просвѣщаѥ-} \\ 
 9& {\Large \glyphfont    ть в́сꙗкого чл҃ка  ⸴} \\ 
10& {\Large \glyphfont   грѧдѫща въ миръ  𝀏̃̑} \\ 
\end{tabular}

\subsection{Assemaniev Gospel (Glagolitic)}
\begin{tabular}{lr}
1 & {\Large \graph   ⁘ ⰅⰂⰀ𞀌҇   ⰙⰕ҇   ⰋⰉ҇Ⱁ } \\
 2 & {\Large \graph  Ⰻⱄⰽⱁⱀⰹ ⰱⱑ } \\
 3 & {\Large \graph       ⱄⰾⱁⰲⱁ  · } \\
 4 & {\Large \graph      ⰻ ⱄⰾⱁⰲⱁ } \\
 5 & {\Large \graph       ⰱⱑ ⱋ̔ ⰱⰰ  · } \\
 6 & {\Large \graph      ⰻ ⰱ͞ⱏ ⰱⱑ } \\
 7 & {\Large \graph      ⱄⰾⱁⰲⱁ  · } \\
 8 & {\Large \graph   Ⱄⰵ ⰱⱑ ⰻ̔ⱄⰽⱁ- } \\
 9 & {\Large \graph     ⱀⰻ  ·  ⱋ̔ ⰱ꙯ⰰ  ·  ⰲⱐ- } \\
10 & {\Large \graph     ⱄⱑ ⱅⱑⰿⱏ ⰱⱏⰻ- } \\
11 & {\Large \graph     ⱎⱔ  ·  Ⰻ̔ ⰱⰵⰶ ⱀⰵⰳⱁ } \\
12 & {\Large \graph     ⱀⰹⱍⰵⱄⱁⰶⰵ } \\
13 & {\Large \graph     ⱀⰵ ⰱⱏⰻⱄⱅⱏ  ·  ⰵ̔- } \\
14 & {\Large \graph     ⰶⰵ ⰱⱏⱄⱅⱏ  · } \\
15 & {\Large \graph    Ⰲⱏ ⱅⱁⰿⱏ ⰶⰹⰲⱁ- } \\
16 & {\Large \graph     ⱅⱏ ⰱⱑ  ·  ⰻ ⰶⰹⰲⱁ- } \\
17 & {\Large \graph     ⱅⱏ ⰱⱑ ⱄⰲⱑⱅⱏ } \\
18 & {\Large \graph     ⱍⰾ҃ⰽⰿⱏ  ·  ⰻ̔ ⱄⰲⱁⱑ } \\
19 & {\Large \graph     ⰲⱏ ⱅⱐⰿⱑ ⱄⰲⱏ- } \\
20 & {\Large \graph     ⱅⰹⱅⱏ ⱄⱔ  ·  ⰻ ⱅⱐ- } \\
21 & {\Large \graph     ⰿⰰ ⰵ̔ⰳⱁ ⱀⰵ ⱁ̔ⰱⱔⱅ } \\
\end{tabular}

