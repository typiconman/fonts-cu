\documentclass[11pt]{ltxdoc}
\usepackage[usenames,dvipsnames,svgnames,table]{xcolor}
\usepackage{fontspec}
\usepackage{xltxtra,comment}
% code borrowed from Polyglossia documentation -- Thanks!
\definecolor{myblue}{rgb}{0.02,0.04,0.48}
\definecolor{lightblue}{rgb}{0.61,.8,.8}
\definecolor{myred}{rgb}{0.65,0.04,0.07}
\usepackage[
    bookmarks=true,
    colorlinks=true,
    linkcolor=myblue,
    urlcolor=myblue,
    citecolor=myblue,
    hyperindex=false,
    hyperfootnotes=false,
    pdftitle={Church Slavonic fonts},
    pdfauthor={Aleksandr Andreev},
    pdfkeywords={Church Slavic, Church Slavonic, Old Church Slavonic, Old Slavonic, fonts, Unicode}
    ]{hyperref}
\usepackage{polyglossia}
\setmainlanguage[Script=Cyrillic]{serbian}
\setotherlanguages{english,churchslavonic,romanian}
\usepackage{churchslavonic}
\usepackage{lettrine}

%% DOCUMENTATION VERSION AND RELEASE DATES
\def\filedate{6. септембар 2020.}
\def\fileversion{верзија 2.2}

%% fontspec declarations:
\setmainfont[Ligatures = TeX]{Libertinus Serif}
\setsansfont{DejaVu Sans}
\setmonofont[Scale=MatchLowercase]{DejaVu Sans Mono}
\newfontfamily\churchslavonicfont[Script=Cyrillic,Ligatures=TeX,Scale=1.33333333,HyphenChar="005F]{PonomarUnicode.otf} 
\newfontfamily{\slv}[Scale=MatchLowercase]{Ponomar Unicode}
\newfontfamily{\ust}[Scale=MatchLowercase]{Menaion Unicode}
\newfontfamily{\ind}[Scale=1.333333333]{Indiction Unicode}
\newfontfamily{\vertograd}[Scale=1.333333]{Vertograd Unicode}
\newfontfamily{\firaslav}[Scale=MatchLowercase]{FiraSlav-Regular}

\linespread{1.05}
%\lineskip=0pt
\lineskiplimit=0em
\frenchspacing
\EnableCrossrefs
\CodelineIndex
\RecordChanges
% COMMENT THE NEXT LINE TO INCLUDE THE CODE
\AtBeginDocument{\OnlyDescription}

\makeatletter
\def\ps@cuNum%
\let\@evenfoot\@oddfoot
}%
\def\cu@lettrine{\lettrine[lines=3,findent=0pt,nindent=0pt]}
\def\cuLettrine{\cu@tokenizeletter\cu@lettrine}
\renewcommand{\LettrineFontHook}{\ind \cuKinovarColor}
\makeatother
\begin{document}

\title{Црквенословенски фонтови}
\author{Александар Андрејев\thanks{Коментари се могу упутити \href{mailto:aleslavista@outlook.it}{Алесандру Ческинију}.} \and Никита Симонс}
\date{\filedate \qquad \fileversion\\
\footnotesize (\textsc{pdf}-датотека генерисана \today)}

\maketitle
\tableofcontents

\section{Увод}

Црквенословенски (такође познат као старословенски; код ISO 639-2 |cu|) јест књижевни језик који користе словенски народи; тренутно га користе као богослужбени језик Руска православна црква, друге месне православне цркве, као и разни унијати и заједнице Старовераца. Пакет \texttt{fonts-churchslavonic} пружа фонтове ради приказивања црквенословенског текста.

Фонтови су осмишљени за обраду Јуникод-текста кодираног по UTF-8. Текст кодиран по застарелим кодним страницама (као што су HIP-а и UCS-а) може се претворити у Јуникод уз коришћење одвојеног пакета услужних програма. За више информација погледајте \href{https://sci.ponomar.net/}{Slavonic Computing Initiative website}.

\subsection{Лиценца}

Фонтови дистрибуирани у том пакету лиценцирају се под отвореном лиценцом фонта SIL (верзија 1.1 или каснија).

Као слободни софтвер, ти фонтови се дистрибуирају у нади да ће бити корисни, но БЕЗ ИКАКВЕ ГАРАНЦИЈЕ; чак и без имплицитне гаранције ПРОДАЈНОСТИ или ПОГОДНОСТИ ЗА ОДРЕЂЕНУ СВРХУ. За више детаља погледајте \href{https://scripts.sil.org/cms/scripts/page.php?site_id=nrsi&id=ofl}{SIL Open Font License}.

Овај документ се лиценцира под Међународном лиценцом CreativeCommonsAttribution-ShareAlike4.0. Да бисте видели копију те лиценце, посетите \href{https://creativecommons.org/licenses/by-sa/4.0/}{CreativeCommons website}.

\subsection{Опис}

Пакет пружа неколико фонтова који су намењени за обраду црквенословенског текста разних редакија и других текстова повезаних са црквенословенским језиком: модерни црквенословенски текст (,,синодални црквеноловенски``), историјски штампани црквенословенски текст и рукописни црквенословенски текст писан уставом (и на ћирилици и на глагољици) као и текст на језику-саха (јакутски језик), алетском језику (наречје Лисичијих острва), те на румунској (молдавској) ћирилици, сви написани на црквеном писму. Покривеност разних фонтова се слаже са смерницама за покривеност фонта наведеним у \href{https://www.unicode.org/notes/tn41/}
{Unicode Technical Note \#41: Church Slavonic Typography in Unicode}. Уопштено говорећи, он укључује највећи део карактера (али не све) у Јуникод-блоковима Ћирилица, Допуна ћирилице, Продужена ћирилица-A, Продужена ћирилица-B, Продужена ћирилица-C (до Јуникода 9.0), Глагољица, и Допуна глагољице. Међутим, карактери који се не користе на црквенословенском језику нису укључени (сем неких карактера који се користе у модерном руском, украјинском, белоруском, српском и македонском језику ради компатибилности са неким апликацијама).


\subsection{Инсталирање и употреба}

Ако читате овај документ, онда сте вероватно већ преузели пакет фонтова. Можете проверити да ли имате најновију верзију посетом \href{https://sci.ponomar.net/}{Slavonic Computing Initiative website}.

\subsubsection{Формати фонтова}

Сви фонтови су тренутно доступни у једином формату:

\begin{description}
\item[\XeTeXpicfile "opentype.png" width 4mm] Фонтови \hyperlink{OT}{OpenType} са
PostScript контурама (и такозвани фонтови OpenType-CFF).

\item[\XeTeXpicfile "deprecated.png" width 4mm] Фонтови TrueType су сада застарели, те се више не пружају. Ако Вам затребају фонтови TrueType погледајте
\href{https://github.com/slavonic/fonts-cu-legacy/}{Legacy Fonts package}.
\end{description}

\noindent Могу Вам затребати застарели фонтови TrueType у следећим ситуацијама:

\begin{itemize}

% The information about versions before OpenOffice.org 3.2 is outdated
% and no longer viewed as relevant
%\item In older versions of OpenOffice.org, OpenType-CFF fonts 
%were not properly embedded into PDF files. Moreover, under Unix-based
%systems, OpenOffice.org could not access such fonts at all, so using TTF
%versions was the only option. This was fixed in OpenOffice.org 3.2 and LibreOffice.

\item OpenOffice.org и старије верзије LibreOffice-а захтевају употребу SIL Graphite-а,
која је доступна само у верзији TTF. То ограничење је поправљено почев од
LibreOffice-а 5.3, који сада има
\href{https://wiki.documentfoundation.org/ReleaseNotes/5.3}{пуну подршку OpenType-а}.

\item Фонтови OpenType-CFF нису били добро подржани у Јави пре Oracle Java SE 7.

\item У Мајкрософтовим производима, постављање глифова OpenType није подржано за глифове у Подручју приватне употребе или за карактере ван опсега Јуникода 7.0. Требало би користити LibreOffice ради постављања комбинованих глагољских карактера.

\end{itemize}

\subsection{Изворни пакети}

Можете такође преузети изворе FontForge за све фонтове са \href{https://github.com/typiconman/fonts-cu/}{репозиторијума GitHub}. То је корисно само ако намеравате да уредите фонтове у уређивачу фонта \href{https://fontforge.sourceforge.net}{FontForge}. Уопште, нећете добити никакво побољшање продуктивности поновним склапањем датотека фонтова, па зато поновно склапање из извора није препоручљиво, сем ако имате стварну потребу за изменом фонтова; на пример, за додавање сопствених додатних глифова Подручју приватне употребе.

\subsection{Системски захтеви}

Сви ти фонтови су велики Јуникод-фонтови и захтевају оперативни систем и софтверско окружење усклађено са Јуникодом. Ван окружења усклађеног са Јуникодом, само ћете бити у стању, највише, приступити првим глифовима фонта којих је 256.

\subsubsection{Мајкрософт Виндовс}

Фонтови кодирани по Јуникоду подржани су почев од Видновса 2000. Потребан Вам је програм за обраду текста који се може бавити документима заснованим на Јуникоду, као што су Мајкрософт Ворда 97 и каснијих верзија, или \href{https://www.libreoffice.org}{LibreOffice}. Ако користите \TeX{},
потребан Вам је мотор \TeX{} усклађен са Јуникодом, као што су \XeTeX{}-а или \LuaTeX-а. 

Потребан Вам је такође посебан начин куцања Јуникод-карактера који нису директно доступни са стандардних тастатура. Препоручујемо инсталирање црквенословенског или продуженог руског распореда тастатуре, доступног са \href{https://sci.ponomar.net/keyboard.html}
{Slavonic Computing Initiative website}. Такође је могуће куцање карактера уз коришћење услуге Мапе знакова Виндовс или по кодној тачки, но то није препоручљиво.

\subsubsection{ГНУ/Линукс}

Како бисте управљали фонтовима OpenType, Ваш систем треба да има библиотеку \href{https://freetype.sourceforge.net}{freetype} инсталирану и укључену; то се нормално дешава као? подразумевано у свим модерним дистрибуцијама. Потребан Вам је програм за обраду текста усклађен са Јуникодом, као што је \href{https://www.libreoffice.org}{LibreOffice}. Ако користите \TeX, потребан Вам је мотор \TeX{} усклађен са Јуникодом, као што су
\XeTeX{} или \LuaTeX.

Потребан Вам је управљачки програм тастатуре за уношење Јуникод-карактера. Под ГНУ/Линуксом, тиме управља библиотека и база података |m17n|. За више детаља погледајте \href{https://sci.ponomar.net/keyboard.html}{Slavonic Computing Initiative website}.

\subsubsection{OS X}

Није познато.

\subsection{Подручје приватне употребе}

Јуникод-Подручје приватне употребе (PUA) скуп је трију опсега кодних тачака (од U+E000-а до U+F8FF-а, Раван 15 и Раван 16) које гарантовано никада неће доделити карактерима Unicode Consortium: могу их користити трећа лица за смештање сопствених карактера. Slavonic Computing Initiative је успоставио индустријски стандард за смештање карактера у PUA, што се у целости описује у \href{https://www.ponomar.net/files/pua_policy.pdf}{PUA Allocation Policy}-у.

PUA у тим фонтовима садржава разне додатне глифове: контекстуалне алтернативе, стилске алтернативе, лигатуре, хипотетичке глифове и глифове за појединачну намену, разне глифове које нису још кодиране у Јуникоду и разне техничке симболе. Највећи део тих глифова (алтернативне глифове и лигатуре) нормално је приступачан посредством функција \hyperlink{OT}{OpenType}. Дакле, уопштено Вам није потребан директан приступ глифовима у PUA-у. Могући су неки изузеци:

\begin{itemize}

\item Уколико треба да приступите карактерима који још нису кодирани у Јуникоду и глифовима за појединачну намену.

\item Уколико треба да приступите алтернативним глифовима и лигатурама у застарелим системима који не подржавају функције OpenType.

\item Уколико сте компјутерски програмер и треба да радите са глифовима на нижем нивоу без ослањања на OpenType-у: располагање свим алтернативама пресликаним у PUA-у омогућава једноставан приступ глифовима кодном тачком уместо да радите са индексима глифова, који се могу променити између верзија фонта.

\end{itemize}

\noindent За карактере пресликане у PUA-у и друга техничка разматрања, погледајте \href{https://www.ponomar.net/files/pua_policy.pdf}{PUA Allocation Policy}.

\section{Технологија OpenType}
\hypertarget{OT}{}\label{OT}

OpenType је технологија ,,паметних фонтова`` намењена напредној типографији коју су развили Microsoft Corporation и Adobe Systems и заснива се на формату фонта TrueType. Она омогућава исправну типографију у комплексним писмима па пружа напредне типографске ефекте. Ово се постиже применом разних \textit{функција}, или \textit{знакова}, описаних у спецификацији OpenType. Неке од тих функција требало би да се укључе као подразумеване, док се друге сматрају опционалним, те их корисници могу укљичити и искључити када то желе.

\subsection{На Мајкрософт Виндовсу}

Како бисте користили напредне типографске функције, поред ,,паметног`` фонта (попут фонтова у пакету), потребна Вам је апликација усклађена са OpenType-ом. Не подржавају све апликације тренутно OpenType, те ни апликације које тврде да подржавају OpenType у ствари подржавају све функције или пружају сучеље за приступање функцијама. Старије верзије библиотеке Microsoft Uniscribe нису подржавале функције OpenType за ћирилицу и глагољицу, но почев од Виндовса 7, то је решено.

Уопштено говорећи, најбољи резултати се добијају у \XeTeX{}-у или \LuaTeX{}-у уз коришћење пакета \texttt{fontspec} или уз коришћење софтвера напредног стоног издаваштва као што је Adobe InDesign. Највећи део функција OpenType такође је приступачан у Мајкрософт Офису 2010 и каснијим верзијама. И LibreOffice подржава функције OpenType почев од верзије 4.1, те је подршка за укључивање и искључивање опционалних функција била додата у верзији 5.3. Погледајте одељак \hyperlink{LO}{Подршка напредних функција у Libreoffice-у}, у наставку.

\subsection{На ГНУ/Линуксу}

Подршку OpenType-а пружа библиотека обликовања HarfBuzz, која је приступачна преко FreeType-а, део већине стандардних дистрибуција X Window System-а. Дакле, OpenType ће бити доступан у било којој апликацији која користи FreeType, мада неким апликацијама недостаје сучеље за укључивање и ускључивање опционалних функција. Уопштено говорећи, најбољи резултати се добијају у \XeTeX{}-у или \LuaTeX{}-у уз коришћење пакета \texttt{fontspec}. LibreOffice такође подржава функције OpenType почев од верзије 4.1, те је подршка за укључивање и искључивање опционалних функција додата у верзији 5.3. Погледајте одељак \hyperlink{LO}{Подршка напредних функција у LibreOffice-у}, у наставку.

\subsection{Функције OpenType}

\subsubsection{Постављање комбинованих знакова}
\hypertarget{mark}{}

OpenType омогућава паметно постављање дијакритика: ако откуцате слово иза којега долази дијакритика, дијакритика ће се ставити тачно изнад или испод слова; то пружа функција \texttt{mark}. Поред тога, функција \texttt{mkmk} користи се за постављање двају знакова у односу на једну другу, тако да се додатна дијакритика може правилно сместити испод прве. То понашање се показује у наставку:

\begin{figure}[h]
\centering
\begin{tabular}{ll}
\large{  {\slv а}  + {\slv ◌́} → {\slv а́ } } &   \\
\large{ {\slv А}  + {\slv ◌́} → {\slv А́ } } & (постављање глифова преко функције \emph{mark}) \\
\large{ {\slv ◌ⷭ} + {\slv  ◌‍҇} → {\slv ◌ⷭ҇ } } & (постављање глифова преко функције \emph{mkmk}) \\
\end{tabular}
\end{figure}

Фонтови пружају правилне полазне тачке \texttt{mark} и \texttt{mkmk} за сва ћирилчка и глагољска слова па комбиноване знакова, омогућавајући Вам да их куцате у готово свакој комбинацији (чак и оне које су невероватне). Највећи део приказивача OpenType (сем старијих верзија библиотеке Adobe Cooltype) подржавају ове функције, па зато би требало да постигнете исправно постављање у највећем делу апликација усклађених са OpenType-ом (на пример, у MS Word-у 2010 или новијим верзијама, LibreOffice 4.1-у или новијим верзијама, те у \XeTeX{}-у).

\subsubsection{Слагање и разлагање глифова}
\hypertarget{ccmp}{}

Функција слагања / разлагања глифова (\texttt{ccmp}) користи се за слагање двају карактера у један глиф ради боље прераде глифова. Та функција се такође користи за стварање сложених облика основног глифа са дијакритичким знацима када употреба \texttt{mark} и \texttt{mkmk} самих по себи не може постићи потребно постављање. Такође се користи за стварање алтернативних облика глифова, као што су алтернативна верзија Псилија коришћена изнад великих слова и скраћени облици слова ,,Ук`` коришћени са нагласним знацима, као што се показује у доле наведеним примерима:

\begin{figure}[h]
\centering
\begin{tabular}{ll}
\large{ {\slv ◌҆} } $\rightarrow$ \large { {\slv  ◌ } } & (замена глифова уз коришћење функције \emph{ccmp}) \\
\large{ {\slv ◌҆}  + {\slv ◌̀} $\rightarrow$ {\slv ◌҆̀} } & (замена глифова уз коришћење функције \emph{ccmp}) \\
\large{ {\slv т}  + {\slv } + {\slv в} $\rightarrow$ {\slv т‍в } } & (замена глифова уз коришћење функције \emph{ccmp}) \\
\large{ {\slv ꙋ}  + {\slv ◌ⷯ} $\rightarrow$ {\slv ꙋⷯ } } & (контекстуална замена уз коришћење функције \emph{ccmp}) \\
\end{tabular}
\end{figure}

Уопштено говорећи, функција \texttt{ccmp} не треба да се искључује (и често ни не може), те би зато та функција требало да правилно ради у било којој апликацији усклађеној са OpenType-ом. За више детаља о лигатурама, погледајте \href{https://www.unicode.org/notes/tn41/}{Unicode Technical Note \#41: Church Slavonic Typography in Unicode}.

\subsubsection{Функције засноване на језику}

Функције засноване на језику као што су функције \texttt{locl} (локализовани облици) пружају приступ алтернативним облицима глифова специфичним за језик, као што је алтернативни облик ћириличког слова ,,И`` коришћеног у украјинском и белоруском језику:

\begin{figure}[h]
\centering
\begin{tabular}{ll}
\large{  {\slv і } } &  (црквенословенски текст) \\
\large{ {\slv і̇ } } & (украјински текст) \\
\end{tabular}
\end{figure}

Ради коришћења тих функција, потребна Вам је апликација усклађена са OpenType-ом која подржава одређивање језика текста, на пример \XeTeX{} или
\LuaTeX{} уз коришћење пакета \texttt{fontspec} или \texttt{polyglossia}. Пошто много софтверских апликација не омогућава да одредите црквенословенски као језик текста, претпоставља се да се фонт подразумевано користи за приказивање црквенословенског текста, па стога сви глифови имају црквенословенски изглед сем ако се не одреди други језик.

LibreOffice Вам омогућава да одредите да је текст на црквенословенском језику почев од верзије 5.0. То ће Вам омогућити да искористите друге функције, као што је црквенословенско растављање на слогове (за више информација погледајте
\href{https://sci.ponomar.net/tools.html}{Slavonic Computing Initiative website}). Microsoft Corporation не препознаје црквенословенски као важећи језик, па зато нећете моћи подесити црквенословенски као језик текста ни у једном Мајкрософтовом производу. \footnote{Молимо, немојте конктирати одржаваоце фонтова по овом питању. Уместо тога, жалите се Мајкрософтовом корисничком сервису у Србији на 0700 300 300 или у Црној Гори на 080 081 110.}

\subsubsection{Стилске алтернативе и стилски скупови}

Стилске алтернативе (функција \texttt{salt}) пружају варијантне облике глифова које корисници могу бирати по вољи. Типично, ово су глифови који се разликују од основног глифа само по графичном изгледу, где употреба тих глифова не прати ниједно правило засновано на језику или типографији, него се радије примењује једноставно украшавање. На пример, пружају се следеће варијантни облици Симбола за Маркова поглавља U+1F545:

\begin{center}
\begin{tabular}{ccccc}
U+1F545	& \multicolumn{4}{c}{Алтернативни глифови} \\
\hline
{\slv \Huge 🕅} &	\textcolor{gray}{\slv \Huge } & \textcolor{gray}{\slv \Huge } & \textcolor{gray}{\slv \Huge } & \textcolor{gray}{\slv \Huge }  \\
\hline
\end{tabular}
\end{center}

Стилски скупови се користе за укључивање групе стилских варијантних глифова, осмишљених да се визуелно хармонизују, те им аутоматски замењују подразумеване облике. OpenType омогућава да се одреде до 20 стилских скупова, означивши их као функције \texttt{ss01}, \texttt{ss02}, \ldots{} \texttt{ss20}.

Употреба стилских алтернатива и стилских скупова захтева апликацију усклађену са OpenType-ом која пружа сучеље за искључивање и укључивање напредних функција (пошто су те функције подразумевано искључене). Ово је могуће у \XeTeX{}-у
или \LuaTeX{}-у уз коришћење пакета \texttt{fontspec} и у LibreOffice-у
(почев од верзије 5.3) употребом посебне синтаксе која придодаје потребну опцију имену фонта. Погледајте одељак \hyperlink{LO}{Подршка напредних функција у LibreOffice-у}, у наставку. У Мајкрософт Офису 2010 и каснијим верзијама, Стилски скупови се могу искључити и укључити у оквиру |Функције OpenType| на картици |Више опција| дијалога |Фонт|. Међутим, Мајкрософт Офис не омогућава да истовремено одаберете вишеструке стилске скупове нити да приступите функцији |salt|. Ако је потребно, можете приступити алтернативним глифовима кодном тачком са Подручја приватне употребе (PUA). Међутим, не препоручује се ослањање на PUA-у као механизам размене података.

\subsection{Технологија SIL Graphite}

\hypertarget{Graphite}{}\label{Graphite} 

\begin{itemize}
\item[\XeTeXpicfile "deprecated.png" width 4mm] Од верзије 1.3 тог пакета, подршка за функције \href{https://scripts.sil.org/Graphite}{SIL Graphite} прекинута је. Ако Вам затреба подршка Graphite-а, погледајте \href{https://github.com/slavonic/fonts-cu-legacy/}{Legacy Fonts package}.
\end{itemize}

\hypertarget{LO}{}\label{LO}

\subsection{Подршка напредних функција у LibreOffice-у}

Подршка функција OpenType доступна је у LibreOffice-у и у свим дериватима OpenOffice.org-а почев од верзије 3.2 OpenOffice.org-а. Док ће исправно постављање, везивање и замене аутоматски радити, раније верзије LibreOffice-а нису имале ниједан механизам за искључивање и укључивање опционалних функција. Подршка искључивања и укључивања функција Graphite доступна је почев од LibreOffice верзије 4.1. Међутим, нема графичког сучеља који се може користити. Уместо тога, била је развијена синтакса посебног проширеног имена фонта: како бисте активирали опционалну функцију, њен идентификатор, а затим знак једнакости и идентификатор жељене поставке, директно се прилажу низу имена фонта. Амперсанд се користи за издвајање различитих парова функција/поставки.

На пример, следећи ,,фонт`` би требало да се користи за укључивање функције |ss01| (Стилски скуп 1):

\begin{verbatim}
Ponomar Unicode:ss01=1
\end{verbatim}

Иста синтакса се користи за искључивање и укључивање опционалних стилских алтернатива (|salt|), где \texttt{1} указује на први алтернативни глиф, \texttt{2} --  други алтернативни глиф, и тако даље. Запазите да ова функција није доступна у Apache OpenOffice-у; пошто Apache OpenOffice није добро одржаван, сугеришемо да корисници пређу у LibreOffice.

Ова функцијанолост ће добро доћи корисницима LibreOffice-а што се ослањају на аутоматско растављање на слогове. Пошто LibreOffice нема \href{https://bugs.documentfoundation.org/show_bug.cgi?id=85731}
{ниједан механизам за подешавање карактера за растављање на слогове}, фонтови Ponomar Unicode и Monomakh Unicode пружају подвлаку као карактер за растављање на слогове преко стилског скупа 1 у OpenType-у.

Свакако, директна измена фонта веома је неугодна, пошто је тешко сетити се кратких знакова и бројчаних вредности коришћених за идентификаторе функција/поставки у различитим фонтовима. Нажалост, тренутно нема графичког сучеља за подржавање исључивања и укључивања функција OpenType и SIL Graphite.

%You may try to install the \href{https://github.com/thanlwinsoft/groooext}
%{Graphite Font Extension}, which provides a dialog for easier feature selection.
%However, this extension has not been maintained since the passing of its
%developer in 2011, and so may not work correctly in later versions of LibreOffice.
%If you experience problems with Graphite features, you may get better
%results accessing glyphs directly by codepoint from the Private Use Area,
%though this is not recommended.

\section{Фонтови за синодални црквенословенски језик}

\subsection{Ponomar Unicode}

Ponomar Unicode је фонт који репродукује словни облик синодалних црквенословенских издања са почетка 20. века. Намењен је за обраду модерних црквенословенских текстова (синодални црквенословенски језик). Ponomar Unicode се заснива на фонту Hirmos UCS који је осмислио Влад Дорош, но га је изменио творац овог пакета. Примери текста сложеног по Ponomar Unicode-у представљени су у наставку.

%\lineskip=0pt
%\lineskiplimit=-5em

\begin{churchslavonic}
Бл҃же́нъ мꙋ́жъ, и҆́же не и҆́де на совѣ́тъ нечести́выхъ, и҆ на пꙋтѝ грѣ́шныхъ не ста̀, и҆ на сѣда́лищи гꙋби́телей не сѣ́де: но въ зако́нѣ гдⷭ҇ни во́лѧ є҆гѡ̀, и҆ въ зако́нѣ є҆гѡ̀ поꙋчи́тсѧ де́нь и҆ но́щь. И҆ бꙋ́детъ ꙗ҆́кѡ дре́во насажде́ное при и҆схо́дищихъ во́дъ, є҆́же пло́дъ сво́й да́стъ во вре́мѧ своѐ, и҆ ли́стъ є҆гѡ̀ не ѿпаде́тъ: и҆ всѧ̑, є҆ли̑ка а҆́ще твори́тъ, ᲂу҆спѣ́етъ. Не та́кѡ нечести́вїи, не та́кѡ: но ꙗ҆́кѡ пра́хъ, є҆го́же возмета́етъ вѣ́тръ ѿ лица̀ землѝ. Сегѡ̀ ра́ди не воскре́снꙋтъ нечести́вїи на сꙋ́дъ, нижѐ грѣ̑шницы въ совѣ́тъ првⷣныхъ. Ꙗ҆́кѡ вѣ́сть гдⷭ҇ь пꙋ́ть првⷣныхъ, и҆ пꙋ́ть нечести́выхъ поги́бнетъ.
\end{churchslavonic}

\textbf{Кијевски црквенословенски језик} користи неколико варијантних облика глифова, као што су U+1C81 Дугоного ,,Де`` ({\slv ᲁ}) и U+A641 Варијантно ,,Зе`` ({\slv ꙁ}):

\begin{churchslavonic}
Бл҃же́нъ мꙋ́жъ, и҆́же не и҆́ᲁе на совѣ́тъ нечести́выхъ, и҆ на пꙋтѝ грѣ́шныхъ не ста̀, и҆ на сѣᲁа́лищи гꙋби́телей не сѣ́ᲁе: но въ зако́нѣ гᲁⷭ҇ни во́лѧ є҆гѡ̀, и҆ въ зако́нѣ є҆гѡ̀ поꙋчи́тсѧ де́нь и҆ но́щь. И҆ бꙋ́ᲁетъ ꙗ҆́кѡ дре́во насажᲁе́ное при и҆схо́ᲁищихъ во́ᲁъ, є҆́же пло́ᲁъ сво́й да́стъ во вре́мѧ своѐ, и҆ ли́стъ є҆гѡ̀ не ѿпаᲁе́тъ: и҆ всѧ̑, є҆ли̑ка а҆́ще твори́тъ, ᲂу҆спѣ́етъ. Не та́кѡ нечести́вїи, не та́кѡ: но ꙗ҆́кѡ пра́хъ, є҆го́же воꙁмета́етъ вѣ́тръ ѿ лица̀ землѝ. Сегѡ̀ ра́ᲁи не воскре́снꙋтъ нечести́вїи на сꙋ́ᲁъ, нижѐ грѣ̑шницы въ совѣ́тъ првⷣныхъ. Ꙗ҆́кѡ вѣ́сть гᲁⷭ҇ь пꙋ́ть првⷣныхъ, и҆ пꙋ́ть нечести́выхъ поги́бнетъ.
\end{churchslavonic}

\textbf{Други језици} Фонт Ponomar Unicode се може користити такође за слагање богослужбених текстова на другим језицима који користе црквено ћириличко писмо. Три таква примера у потпуности подржава фонт: румунски језик (молдавски) на његовом ћирилчком писму, алеутски јеик (наречје Лисичјих острва или источни) на његовој азбуци, те јакутски језик (Саха) писан азбуком коју је владика Дионисије (Хитров) створио.

\noindent Пример Оченаша на румунској (молдавској) ћирилици: \\

\begin{churchslavonic}
Та́тъль но́стрꙋ ка́реле є҆́щй ꙟ҆ Че́рюрй: ᲃ︀фн҃цѣ́скъсе Нꙋ́меле тъ́ꙋ: ві́е ꙟ҆пъръці́ѧ та̀: фі́е во́ѧ та̀, прекꙋ́мь ꙟ҆ Че́рю̆ шѝ пре пъмѫ́нть. Пѫ́йнѣ но́астръ чѣ̀ ᲁепꙋ́рꙋрѣ ᲁъ́не но́аѡ а҆́стъꙁй. Шѝ не ꙗ҆́ртъ но́аѡ греша́леле но́астре, прекꙋ́мь шѝ но́й є҆ртъ́мь греши́цилѡрь но́щри. Ши́ нꙋ́не ᲁꙋ́че пре но́й ꙟ҆ и҆спи́тъ. Чѝ не и҆ꙁбъвѣ́ще ᲁе че́ль ръ́ꙋ. \\
\end{churchslavonic}

\noindent Пример Оченаша на алеутској ћирилици: \\ 

\begin{churchslavonic}
Тꙋмани́нъ А́даԟъ! А҆́манъ акꙋ́х̑тхинъ и́нинъ кꙋ́ҥинъ, А́са́нъ амчꙋг̑а́сѧ́да́г̑та, Аҥали́нъ а҆ԟа́г̑та, Анꙋхтана́тхинъ малга́г̑танъ и́нимъ кꙋ́ганъ ка́юхъ та́намъ кꙋ́ганъ. Ԟалга́дамъ анꙋхтана̀ ҥи̑нъ аԟача́ ꙋ̆а҆ѧ́мъ: ка́юхъ тꙋма́нинъ а́д̑ꙋнъ ҥи̑нъ игни́да, а҆ма́кꙋнъ тꙋ́манъ ка́юхъ малгалиги́нъ ҥи̑нъ ад̑ꙋг̑и́нанъ игнида́кꙋнъ: ка́юхъ тꙋ́манъ сꙋглатачх̑и́г̑анах̑тхинъ, та́г̑а ад̑алю́дамъ илѧ́нъ тꙋ́манъ аг̑г̑ича. \\
\end{churchslavonic}

\noindent Примера Оченаша на јакутском језику (Саха): \\ %[TODO: there's a problem with linespread in the following text:]

\begin{churchslavonic}
Халланнаръ юрдюлѧригѧрь баръ агабытъ бисенѧ ! Свѧтейдѧннинь а̄тыҥъ эенѧ ; кѧллинь царстваҥъ эенѧ ; сирь юрдюгѧрь кёҥюлюҥь эенѧ , халланъ юрдюгѧрь курдукъ боллунъ ; бюгюҥю кюннѧги асыръ аспытынъ бисенинь кулу бисеха бюгюнь ; бисиги да естѧрбитинь халларъ бисеха , хайтахъ бисиги да халларабытъ беэбить естѧхтѧрбитигѧрь ; килѧримѧ да бисигини альԫархайга ; хата быса бисигини албынтанъ . \\
\end{churchslavonic}

\subsubsection{Напредне функције фонта}

Ponomar Unicode ставља неке карактере у Подручје приватне употребе (PUA). За опште пресликавање PUA-а, погледајте \href{https://www.ponomar.net/files/pua_policy.pdf}{PUA Allocation Policy}.

Поред општих пресликавања PUA-а, неки карактери су додељени одељку отвореног опсега PUA-а. То су:

\begin{itemize}
\item U+F400 \textendash{} Алтернативе за глифове Допунске вишејезичне равни (SMP): овај одељак садржи копије у Основној вишејезичкој равни (BMP) за подршку у застарелим апликацијама. Тренутно, доступни су следећи: U+F400 - U+F405 \textendash{} Типиконови симболи (копије од U+1F540 до U+1F545).
\item U+F410 \textendash{} Облици представљања: садрже разне облике представљања и лигатуре које фонт користи изнутра. Уопште, ово нису намeњене томе да их могу позвати корисници или спољне апликације.
\item U+F420 \textendash{} Језичке алтернативе: садрже алтернативне облике глифова који су специфични за језик. Засада, ово су модерни интерпункцијски облици за употребу са латинским карактерима. Они нису намењени за спољно позвиање.
\item U+F441 и даље \textendash{} стилске алтернативе латинских карактера (готички облици). Оне се могу позвати преко Стилског скупа 2, но, ако је потребно, могу се директно позвати из PUA-а. Оне се пресликавају у истом редоследу као и у Основном латинском блоку, почев од U+F441-а (чему одговара U+0041 Латинско велико слово A). Поред репертоара Основног латинског, имамо и: U+F4DE \textendash{} Готички Торн; U+F4FE \textendash{} Мало слово готички Торн; те U+F575 \textendash{} Готички дуги S
\end{itemize}

Фонт пружа неколико лигатура, које се праве увођењем Знака форматирања без ширине без прелома (U+200D), између два карактера. Списак лигатура се наводи у табели~\ref{ligs1}.

\begin{table}[htbp]
\centering
\caption{Лигатуре доступне у Ponomar Unicode-у \label{ligs1}}
\begin{tabular}{lcc}
Име	& Секвенца	& Изглед \\
\hline
Лигатура А-У	& U+0430 U+200D U+0443	& {\slv{\large а‍у}}	\\
Лигатура Ел-У	& U+043B U+200D U+0443 & {\slv{\large л‍у}}	\\
Лигатура Те-Ве	& U+0442 U+200D U+0432	& {\slv{\large т‍в}}	\\
\hline
\end{tabular}
\end{table}

\noindent У OpenType-у, дефинише се неколико стилских алтернатива. Оне се наводе у табели~\ref{salt1}. Поред додатних глифова за Симбол за Марково поглавље, функција пружа алтернативне украсне облике слова U+0423 У, које изгледа тачно као U+A64A Ук (ова употреба се налази у неким публикацијама), те алтернативни облик за U+0404 Широки Је за употребу у контекстима где треба да се разликује од U+0415 Је (понајвише за украјински текст по црквенословенског стилу).

\newfontfamily{\salt}[Alternate=0]{Ponomar Unicode}
\newfontfamily{\salta}[Alternate=1]{Ponomar Unicode}
\newfontfamily{\saltb}[Alternate=2]{Ponomar Unicode}
\newfontfamily{\saltc}[Alternate=3]{Ponomar Unicode}
\newfontfamily{\saltd}[Alternate=4]{Ponomar Unicode}
\newfontfamily{\salte}[Alternate=5]{Ponomar Unicode}
\newfontfamily{\saltf}[Alternate=6]{Ponomar Unicode}
\newfontfamily{\saltg}[Alternate=7]{Ponomar Unicode}

\begin{table}[htbp]
\centering
\caption{Стилске алтернативе у Ponomar Unicode-у \label{salt1}}
\begin{tabular}{lccccc}
	& Основни облик	& \multicolumn{4}{c}{Алтернативни облици} \\
\hline
U+1F545	& {\slv{\large 🕅 }}	& {\salt\large 🕅} & {\salta\large 🕅} & {\saltb\large 🕅} & {\saltc\large 🕅}  \\
			&				& {\saltd\large 🕅} & {\salte\large 🕅} & {\saltf\large 🕅} & {\saltg\large 🕅} \\
U+0423		& {\slv\large У}	& {\salt\large У} \\
U+040E		& {\slv\large Ў}	& {\salt\large Ў} \\
U+0404		& {\slv\large Є}	& {\salt\large Є} \\
\hline
\end{tabular}
\end{table}

За ћирилчка слова, функција стилских алтернатива омогућава такође приступ скраћеним облицима; редослед алтернативних облика је увек: ниже скраћивање, горње скраћивање, лево скраћивање, десно скраћивање. Табела~\ref{trunc} показује који скраћени облици јесу доступни. Уопштено говорећи, скраћивањем би требало да аутоматски управља софтвер за стоно издаваштво и \TeX{}, мада је то тешко постићи.

\begin{table}[htbp]
\centering
\caption{Скраћени облици приступачни преко функције Стилских алтернатива
у Ponomar Unicode-у \label{trunc}}
\begin{tabular}{lccccc}
	& Основни облик	& \multicolumn{4}{c}{Скраћени облици} \\
\hline
U+0440	& {\slv{\large р }}	& {\salt\large р} &  \\
U+0443  & {\slv{\large у }}	& {\salt\large у} &  \\
U+0444  & {\slv{\large ф }}	& {\salt\large ф} & {\salta\large ф} \\
U+0445  & {\slv{\large х}}	& {\salt\large х} & {\salta\large х} & {\saltb\large х}  \\
U+0446  & {\slv{\large ц }}	& {\salt\large ц} &  \\
U+0449  & {\slv{\large щ }}	& {\salt\large щ} &  \\
U+0471  & {\slv{\large ѱ }}	& {\salt\large ѱ} &  {\salta\large ѱ}\\
U+A641  & {\slv{\large ꙁ }}	& {\salt\large ꙁ} &  \\
U+A64B  & {\slv{\large ꙋ }}	& {\salt\large ꙋ} & {\salta\large ꙋ} & {\saltb\large ꙋ} \\
\hline
\end{tabular}
\end{table}

Стилски скуп 1 (|ss01|) пружа се као привремено заобилазно решење за \href{https://bugs.documentfoundation.org/show_bug.cgi?id=85731}
{LibreOffice Bug 85731}, што Вам не омогућава да одредите карактер за растављање на слогове у LibreOffice-у. Када се укључи, замењује сва појављивања U+002D-а Цртице-минуса и U+2010-а Цртице U+005F-ом Ниском линијом (подвлаком) за употребу као карактер за растављање на слогове. Запазите да ће се ова функција напустити чим се потребна функционалност буде додала LibreOffice-у.

Дефинише се и Стилски скуп 2 (,,ss02``), Готички облици. Када се тај стилски скуп укључи, латинска слова се појављују на готици насупрот њиховим модерним облицима. То је корисно ради слагања латинског текста заједно са црквенословенским у неким контекстима. Погледајте следећи пример:

\newfontfamily{\blackletter}[StylisticSet=2]{Ponomar Unicode}

\begin{figure}[h]
\centering
\begin{tabular}{ll}
Редован &
{\slv \large The quick brown fox. 1234567890. А҆ сїѐ по слове́нски. } \\
Готички & 
{\blackletter \large The quick brown fox. 1234567890. А҆ сїѐ по слове́нски. } \\
\end{tabular}
\end{figure}

\noindent Запазите да се од верзије 2.0 фонта, цифре ASCII (обично називане
,,арапским бројевима``) пружају на латинском облику. Користите Стилски скуп 2 за приступање готичким облицима, ако је потребно.

\section{Фонтови за праниконов штампани црквенословенски језик}

\subsection{Fedorovsk Unicode}

Fedorovsk Unicode се заснива на фонту Fedorovsk који је осмислио Никита Симонс.
Поново је био кодиран за Јуникод, са функцијама OpenType које је додао Александр Андрејев. Фонт Fedorovsk намерава да репродукује фонт штампованих издања Ивана Фјодорова објављених у Москви, на пример, Апостол из 1564. Фонт је првенствено намењен за слагање праниконових (староверских) богослужбених текстова или за обреду таквих текстова у академском контексту.

\newfontfamily{\rightFedor}[StylisticSet=1,HyphenChar="200B]{Fedorovsk Unicode}

Ево примера из Апостола Ивана Фјодорова из 1564. године.

\begin{churchslavonic}
{\Large \rightFedor
\textcolor{red}{П}е́рвᲂе ᲂу҆́бо︀ сло́во︀ сᲂтвᲂри́хъ о҆ всѣ́хъ , ѽ , ѳео҆́филе , о҆ ниⷯже начѧ́тъ і︮с︯ , твᲂри́тиже и҆ ᲂу҆чи́ти . д︀о︀ него́же дн҃е , запᲂвѣ́д︀авъ а҆пⷭ҇лᲂмъ дх҃ᲂмъ ст҃ыⷨ , и҆́хже и҆ꙁбра̀ вᲂзнесе́сѧ . преⷣ ни́миже и҆ пᲂста́ви себѐ жи́ва по страд︀а́нїи свᲂе҆́мъ . во︀ мно́зехъ и҆́стинныхъ зна́менїи҆хъ . дн҃ьми четы́ридесѧтьми ꙗ҆влѧ́ꙗсѧ и҆́мъ и҆ гл҃ѧ ꙗ҆́же о҆ црⷭ҇твїи бж҃їи . сни́миже и҆ ꙗ҆д︀ы̀ , пᲂвелѣва́ше и҆́мъ ѿ і҆е҆рᲂсали́ма не ѿлꙋча́тисѧ . но̑ жда́ти о҆бѣтᲂва́нїе ѿч︮е︯е , е҆́же слы́шасте ѿ́ менѐ . ꙗ҆́кѡ і҆ѡ҃а́ннъ ᲂу҆́бо︀ крⷭ҇ти́лъ е҆́сть вᲂдо́ю . вы́же и҆́мате крести́тисѧ дх҃ᲂмъ ст҃ы́мъ , не по мно́ꙁѣхъ си́хъ д︀︮н︯еⷯ .
}
\end{churchslavonic}

Ево примера из Цветног триода из 1648. године.

\newfontfamily{\leftFedor}[StylisticSet=2,HyphenChar="200B]{Fedorovsk Unicode}

\begin{churchslavonic}
{\Large \leftFedor
\textcolor{red}{стⷯры па́сцѣ . гла́съ , є҃ .} Д\textcolor{red}{а вᲂскрⷭ҇нетъ бг҃ъ ,꙳ и҆ разы́дꙋтсѧ вразѝ є҆гѡ̀ .}
Па́сха сщ҃е́ннаѧ на́мъ дне́сь пᲂказа́сѧ , па́сха но́ва ст҃а́ѧ , па́сха таи́нственнаѧ , па́сха всечестна́ѧ , па́сха хрⷭ҇та̀ и҆зба́вителѧ , па́сха непᲂро́чнаѧ , па́сха вели́каѧ , па́сха вѣ́рнымъ , па́сха двѣ́ри ра́йскїѧ на́мъ ѿверза́ющаѧ , па́сха всѣ́хъ ѡ҆сщ҃а́ющаѧ вѣ́рныхъ .
}
\end{churchslavonic}

\subsubsection{Напредне функције фонта}

Фонт пружа неколико лигатура, које се праве увођењем Знака форматирања без ширине без прелома (U+200D) између два карактера. Списак лигатура се наводи у табели~\ref{ligs2}.

\begin{table}[htbp]
\centering
\caption{Лигатуре доступне у Fedorovsk Unicode-у \label{ligs2}}
\begin{tabular}{lcc}
Име	& Секвенца	& Изглед \\
\hline
Лигатура А-У	& U+0430 U+200D U+0443	& {\leftFedor{\large а‍у}}	\\
Лигатура Ел-У	& U+043B U+200D U+0443 & {\leftFedor{\large л‍у}}	\\
Лигатура А-Ижица & U+0430 U+200D U+0475	& {\leftFedor{\large а‍ѵ}}	\\
Лигатура Ел-Ижица & U+043B U+200D U+075 & {\leftFedor{\large л‍ѵ}}	\\
Лигатура Те-Ве	& U+0442 U+200D U+0432	& {\leftFedor{\large т‍в}}	\\
Лигатура Ер-Јат	& U+0440 U+200D U+0463 & {\leftFedor{\large р‍ѣ}}	\\
\hline
\end{tabular}
\end{table}

\noindent У OpenType-у, дефинише се неколико стилских алтернатива. Новоде се у табели~\ref{salt2}. Поред пружања алтернативних облика глифова за U+1F545 Симбол за Марково поглавље, оне Вам омогућавају да контролишете постављање дијакритичких знакова изнад извесних слова.

\newfontfamily{\glyphfont}{Fedorovsk Unicode}
\newfontfamily{\saltFedor}[Alternate=0]{Fedorovsk Unicode}
\newfontfamily{\saltaFedor}[Alternate=1]{Fedorovsk Unicode}
\newfontfamily{\saltbFedor}[Alternate=2]{Fedorovsk Unicode}
\newfontfamily{\saltcFedor}[Alternate=3]{Fedorovsk Unicode}
\newfontfamily{\saltdFedor}[Alternate=4]{Fedorovsk Unicode}
\newfontfamily{\salteFedor}[Alternate=5]{Fedorovsk Unicode}
\newfontfamily{\saltfFedor}[Alternate=6]{Fedorovsk Unicode}

\begin{table}[htbp]
\centering
\caption{Стилске алтернативе у Fedorovsk Unicode-у \label{salt2}}
\begin{tabular}{lcccccccc}
	& Основни облик	& \multicolumn{7}{c}{Алтернативни облици} \\
\hline
U+0404	& {\glyphfont{\large Є}} & {\saltFedor\large Є} \\
U+0426	& {\glyphfont{\large Ц}} & {\saltFedor\large Ц} \\
U+0491	& {\glyphfont{\large ґ}} & {\saltFedor\large ґ} \\
U+A64C	& {\glyphfont{\large Ꙍ}} & {\saltFedor\large Ꙍ} \\
U+047C	& {\glyphfont{\large Ѽ}} & {\saltFedor\large Ѽ} \\
U+047E	& {\glyphfont{\large Ѿ}} & {\saltFedor\large Ѿ} \\
U+047F	& {\glyphfont{\large ѿ}} & {\saltFedor\large ѿ} \\
U+1F545	& {\glyphfont{\large 🕅 }}	& {\saltFedor\large 🕅} & {\saltaFedor\large 🕅} & {\saltbFedor\large 🕅} & {\saltcFedor\large 🕅}  & {\saltdFedor\large 🕅} & {\salteFedor\large 🕅} & {\saltfFedor\large 🕅} \\
U+0463 U+0486	& {\glyphfont{\large ѣ҆}} & {\saltFedor\large ѣ҆}  \\
U+0463 U+0300	& {\glyphfont{\large ѣ̀}} & {\saltFedor\large ѣ̀} & {\saltaFedor\large ѣ̀} \\
U+0463 U+0301	& {\glyphfont{\large ѣ́}} & {\saltFedor\large ѣ́} & {\saltaFedor\large ѣ́} \\
U+0463 U+0311	& {\glyphfont{\large ѣ̑}} & {\saltFedor\large ѣ̑} & {\saltaFedor\large ѣ̑} \\
U+0463 U+0486 U+0301	& {\glyphfont{\large ѣ҆́}} & {\saltFedor\large ѣ҆́}  \\
U+A64B U+0486	& {\glyphfont{\large ꙋ҆}} & {\saltFedor\large ꙋ҆}  \\
U+A64B U+0300	& {\glyphfont{\large ꙋ̀}} & {\saltFedor\large ꙋ̀} & {\saltaFedor\large ꙋ̀} \\
U+A64B U+0301	& {\glyphfont{\large ꙋ́}} & {\saltFedor\large ꙋ́} & {\saltaFedor\large ꙋ́} \\
U+A64B U+0311	& {\glyphfont{\large ꙋ̑}} & {\saltFedor\large ꙋ̑} & {\saltaFedor\large ꙋ̑} & {\saltbFedor\large ꙋ̑} \\
U+A64B U+0486 U+0301	& {\glyphfont{\large ꙋ҆́}} & {\saltFedor\large ꙋ҆́}  \\
\hline
\end{tabular}
\end{table}

Поред тога, три стилска скупа су дефинисана у фонту. Стилски скуп 1 (,,Нагласци на десну страну``) поставља нагласке изнад Јата и Ука на десној страни и Стилски скуп 2 (,,Нагласци на леву страну``) поставља нагласке изнад Јата и Ука на левој страни. Ти стилски скупови могу послужити када текст свуда користи једно од тих постављања. Стилски скуп 10 (,,Варијанте са једнаком основном линијом``) ставља велика слова на исту основну линију као и мала слова (корисно за рад са фонтом у академском контексту где основна линија великих слова, која је традиционално спуштена, може изазвати проблеме са вертикалним простором када се обради текст који је и на латиници и на ћирилици). Ево примера:

\newfontfamily{\base}[StylisticSet=10]{Fedorovsk Unicode}

\begin{figure}[h]
\centering
\begin{tabular}{ll}
{\large \glyphfont Хрⷭ҇то́съ вᲂскр҃се и҆з̾ ме́ртвыхъ} & (редовни текст) \\
{\large \base Хрⷭ҇то́съ вᲂскр҃се и҆з̾ ме́ртвыхъ} & (укључен Стилски скуп 10) \\
\end{tabular}
\end{figure}

\section{Фонтови за обраду древних рукописа}

\subsection{Menaion Unicode}

Фонт Menaion треба да се користи за обраду текста рукописа из времена коришћења устава. Садржи пуни репертоар потребних ћирилчких и глагољских глифова као и глифове византијске екфонетске нотације какве се користе у ћирилчким или глагољским рукописима.

Фонт Menaion је првобитно осмислио Виктор A. Баранов на \href{http://www.manuscripts.ru/}{the Manuscript Project}-у. Поново га је кодирао за Јуникод Александар Андрејев, уз дозволу првобитног твораца.

\newfontfamily{\menaion}[HyphenChar="200B]{Menaion Unicode}

Узорци текста у Menaion Unicode-у су представљени сликама~\ref{men1}
и \ref{men2}. Запазите да су комбинована глагољска слова (Глагољска допуна) постала доступним у Јуникоду 9.0. У старијим верзијама Мајкрософтових софтвера, ваљда исправно постављање глифова за ове карактере уз коришћење функција OpenType неће бити могуће. За постизање жељеног резултата препоручујемо да користите LibreOffice, \XeTeX{}, \LuaTeX{}, или софтвере напредног стоног издаваштва као што је Adobe InDesign.

\begin{figure}[htbp]
\centering
\caption{Ћирилчки текст из Остромировог јеванђеља (XI век) \label{men1}}
\begin{tabular}{lr}
 1& {\Large \menaion    Искони бѣ слово } \\
 2& {\Large \menaion    и слово бѣ отъ  } \\
 3& {\Large \menaion   б҃а и б҃ъ бѣ} \\ 
 4& {\Large \menaion    слово  𝀏̃  се бѣ} \\ 
 5& {\Large \menaion    искони оу} \\ 
 6& {\Large \menaion    б҃а  ⁘  и тѣмь в̇са бꙑ} \\ 
 7& {\Large \menaion    шѧ  𝀏̃  и беꙁ него ни} \\ 
 8& {\Large \menaion    чьтоже не бꙑсть  ·} \\ 
 9& {\Large \menaion   ѥже бꙑсть  𝀏̃  въ то} \\ 
10& {\Large \menaion    мь животъ бѣ  ·  и} \\ 
 1& {\Large \menaion    животъ бѣ свѣтъ} \\ 
 2& {\Large \menaion    чловѣкомъ  𝀏̃  и свѣ} \\ 
 3& {\Large \menaion    тъ въ тьмѣ свьти} \\ 
 4& {\Large \menaion    тьсѧ  ·  и тьма ѥго} \\ 
 5& {\Large \menaion    не обѧтъ  𝀏̃  бꙑсть} \\ 
 6& {\Large \menaion    члв҃къ посъланъ} \\ 
 7& {\Large \menaion    отъ б҃а  ·  имѧ ѥмоу} \\ 
 8& {\Large \menaion    иоанъ  𝀏̃  тъ приде} \\ 
 9& {\Large \menaion    въ съвѣдѣтель} \\ 
10& {\Large \menaion    ство  ·  да съвѣдѣте} \\ 
2.2  1& {\Large \menaion    льствоуѥть о свѣ} \\ 
 2& {\Large \menaion    тѣ  𝀏̃  да вьси вѣрѫ} \\ 
 3& {\Large \menaion    имѫть имь  ⁘  не бѣ} \\ 
 4& {\Large \menaion    тъ свѣтъ  ⁘  нъ да} \\ 
 5& {\Large \menaion    съвѣдѣтельствоу} \\ 
 6& {\Large \menaion    ѥть о свѣтѣ  𝀏̃̑ бѣ} \\ 
 7& {\Large \menaion    свѣтъ истиньнꙑ} \\ 
 8& {\Large \menaion    и  ·  иже просвѣщаѥ} \\ 
 9& {\Large \menaion    ть в́сꙗкого чл҃ка  ⸴} \\ 
10& {\Large \menaion   грѧдѫща въ миръ  𝀏̃̑} \\ 
\end{tabular}
\end{figure}

\begin{figure}[htbp]
\centering
\caption{Глагољски текст из Асеманијевог јеванђеља (XI век) \label{men2}}
\begin{tabular}{lr}
1 & {\Large \menaion   ⁘ ⰅⰂⰀ𞀌҇   ⰙⰕ҇   ⰋⰉ҇Ⱁ } \\
 2 & {\Large \menaion  Ⰻⱄⰽⱁⱀⰹ ⰱⱑ } \\
 3 & {\Large \menaion       ⱄⰾⱁⰲⱁ  · } \\
 4 & {\Large \menaion      ⰻ ⱄⰾⱁⰲⱁ } \\
 5 & {\Large \menaion       ⰱⱑ ⱋ̔ ⰱⰰ  · } \\
 6 & {\Large \menaion      ⰻ ⰱ͞ⱏ ⰱⱑ } \\
 7 & {\Large \menaion      ⱄⰾⱁⰲⱁ  · } \\
 8 & {\Large \menaion   Ⱄⰵ ⰱⱑ ⰻ̔ⱄⰽⱁ} \\
 9 & {\Large \menaion     ⱀⰻ  ·  ⱋ̔ ⰱ꙯ⰰ  ·  ⰲⱐ} \\
10 & {\Large \menaion     ⱄⱑ ⱅⱑⰿⱏ ⰱⱏⰻ} \\
11 & {\Large \menaion     ⱎⱔ  ·  Ⰻ̔ ⰱⰵⰶ ⱀⰵⰳⱁ } \\
12 & {\Large \menaion     ⱀⰹⱍⰵⱄⱁⰶⰵ } \\
13 & {\Large \menaion     ⱀⰵ ⰱⱏⰻⱄⱅⱏ  ·  ⰵ̔} \\
14 & {\Large \menaion     ⰶⰵ ⰱⱏⱄⱅⱏ  · } \\
15 & {\Large \menaion    Ⰲⱏ ⱅⱁⰿⱏ ⰶⰹⰲⱁ} \\
16 & {\Large \menaion     ⱅⱏ ⰱⱑ  ·  ⰻ ⰶⰹⰲⱁ} \\
17 & {\Large \menaion     ⱅⱏ ⰱⱑ ⱄⰲⱑⱅⱏ } \\
18 & {\Large \menaion     ⱍⰾ҃ⰽⰿⱏ  ·  ⰻ̔ ⱄⰲⱁⱑ } \\
19 & {\Large \menaion     ⰲⱏ ⱅⱐⰿⱑ ⱄⰲⱏ} \\
20 & {\Large \menaion     ⱅⰹⱅⱏ ⱄⱔ  ·  ⰻ ⱅⱐ} \\
21 & {\Large \menaion     ⰿⰰ ⰵ̔ⰳⱁ ⱀⰵ ⱁ̔ⰱⱔⱅ } \\
\end{tabular}
\end{figure}

\subsubsection{Напредне функције фонта}

Фонт пружа неколико лигатура, које се праве увођењем Знака форматирања без ширине без прелома (U+200D) између два карактера. Списак лигатура се наводи у табели~\ref{menligs}.

\begin{table}[htbp]
\centering
\caption{Лигатуре доступне у фонту Menaion Unicode \label{menligs}}
\begin{tabular}{lcc}
Име	& Секвенца	& Изглед \\
\hline
Мала лигатура И-Је &	U+0438 U+200D U+0435 	& {\menaion{\large и‍е }} \\
Мала лигатура Ен-И	&	U+043d U+200D U+0438 	& {\menaion{\large н‍и }} \\
Мала лигатура Ен-Мали јус	& U+043d U+200D U+0467 	& {\menaion{\large н‍ѧ }} \\
Мала лигатура Ес-Ве	&	U+0441 U+200D U+0432 	& {\menaion{\large с‍в }} \\
Мала лигатура Те-Ер	&	U+0442 U+200D U+0440 	& {\menaion{\large т‍р }} \\
Велика лигатура А-У	& 	U+0410 U+200D U+0423 	& {\menaion{\large А‍У }} \\
Мала лигатура А-У	&	U+0430 U+200D U+0443 	& {\menaion{\large а‍у }} \\
Мала лигатура А-Те		&	U+0430 U+200D U+0442 	& {\menaion{\large а‍т }} \\
Велика лигатура И-Је	&	U+0418 U+200D U+0415 	& {\menaion{\large И‍Е }} \\
Велика лигатура Ел-Ге		&	U+041b U+200D U+0413 	& {\menaion{\large Л‍Г }} \\
Мала лигатура Ел-Ге		&	U+043b U+200D U+0433 	& {\menaion{\large л‍г }} \\
Велика лигатура Ен-И	&	U+041d U+200D U+0418 	& {\menaion{\large Н‍И }} \\
Велика лигатура Ен-Мали јус	&	U+041d U+200D U+0466 	& {\menaion{\large Н‍Ѧ }} \\
Велика лигатура Ес-Ве		&	U+0421 U+200D U+0412 	& {\menaion{\large С‍В }} \\
Мала лигатура Те-Јат		&	U+0442 U+200D U+0463 	& {\menaion{\large т‍ѣ }} \\
Велика лигатура Те-Ве	&	U+0422 U+200D U+0412	& {\menaion{\large Т‍В }} \\
Мала лигатура Те-Ве		&	U+0442 U+200D U+0432	& {\menaion{\large т‍в }} \\
Велика лигатура Те-И		&	U+0422 U+200D U+0418 	& {\menaion{\large Т‍И }} \\
Мала лигатура Те-И		&	U+0442 U+200D U+0438 	& {\menaion{\large т‍и }} \\
Велика лигатура Те-Ер		&	U+0422 U+200D U+0420 	& {\menaion{\large Т‍Р }} \\
Лигатура Велико А-Мало Те	&	U+0410 U+200D U+0442 	& {\menaion{\large А‍т }} \\
Велика лигатура Те-Меки знак	&	U+0422 U+200D U+042c 	& {\menaion{\large Т‍Ь }} \\
Мала лигатура Те-Меки знак	&	U+0442 U+200D U+044C 	& ‍{\menaion{\large т‍ь }} \\
Мала лигатура Те-А		&	U+0442 U+200D U+0430 	& {\menaion{\large т‍а }} \\
\hline
\end{tabular}
\end{table}

\section{Фонтови за академски рад}

\subsection{Monomakh Unicode}

Monomakh Unicode се заснива на фонту Monomachus који је осмислио Алексеј Крјуков. Он је био измењен уз дозволу. Monomakh Unicode је ћирилички фонт примењен у мешовитом уставном / полууставном стилу; њиме се могу задовољити потребе истраживања која се баве словенском историјом и филологијом. Он укључује све историјске ћирилчке карактере који се тренутно дефинишу у Јуникоду; укључује такође скуп латинских слова осмишљен да буде стилски компатибилан са ћирилчким делом. То може бити корисно ради слагања двојезичних издања на црквенословенском језику и језицима написаним на латинском писму, посебно они који користе много дијакритичких знакова, као што је у румунском примеру у наставку.

\newfontfamily{\monomakh}{Monomakh Unicode}

\begin{tabular}{p{2.25in}p{0.02in}p{2.25in}}
\begin{churchslavonic}
{\monomakh Бл҃же́нъ мꙋ́жъ, и҆́же не и҆́де на совѣ́тъ нечести́выхъ, и҆ на пꙋтѝ грѣ́шныхъ не ста̀, и҆ на сѣда́лищи гꙋби́телей не сѣ́де: но въ зако́нѣ гдⷭ҇ни во́лѧ є҆гѡ̀, и҆ въ зако́нѣ є҆гѡ̀ поꙋчи́тсѧ де́нь и҆ но́щь. И҆ бꙋ́детъ ꙗ҆́кѡ дре́во насажде́ное при и҆схо́дищихъ во́дъ, є҆́же пло́дъ сво́й да́стъ во вре́мѧ своѐ.}
\end{churchslavonic}
& &
\begin{romanian}
{\monomakh Fericit bărbatul, care n-a umblat în sfatul necredincioșilor și în calea păcătoșilor nu a stat și pe scaunul hulitorilor n-a șezut; ci în legea Domnului e voia lui și la legea Lui va cugeta ziua și noaptea. și va fi ca un pom răsădit lângă izvoarele apelor, care rodul său va da la vremea sa.}
\end{romanian}
\end{tabular}

\subsubsection{Напредне функције фонта}

Фонт Monomakh нуди неколико опционалних функција OpenType које могу укључити или искључити корисници. То су:

\begin{itemize}
\item Стилски скуп 1 (\emph{ss01}) се пружа као привремено решење  \href{https://bugs.documentfoundation.org/show_bug.cgi?id=85731}
{LibreOffice Bug 85731}-а, који Вам не омогућава да одредите карактер растављања на слогове у LibreOffice-у. Када се укључи, замењује сва појављивања U+002D-а Цртице-минуса и U+2010-а Цртице U+005F-ом Ниском цртом (подвлаком) за употребу као карактер растављања на слогове. Запазите да ће се ова функција напустити чим се потребна фунционалност буде додала LibreOffice-у.
\item Стилски скуп 6 (\emph{ss06}) приказује U+0456 Ћирилчко мало слово украјинско / белоруско И са једном горњом тачком и Стилски скуп 7 (\emph{ss07}) приказује исти карактер са две горње тачке. Подразумевано, U+0456 приказује се без тачки.
\item Стилски скуп 8 (\emph{ss08}) приказује карактере U+0417 Ћирилчко велико слово Зе и U+0437 Ћирилчко мало слово Зе као ,,оштру земљу``, тј., попут карактера U+A640 Ћирилчко велико слово Земља, односно U+A641 Ћирилчко мало слово Земља. Уопште, ова промена би требало да се направи на нивоу кодне тачке, па се зато употреба ове функције не препоручује.
\item Стилски скуп 9 (\emph{ss09}) приказује карактере U+0427-а Ћирилчко велико слово Че и U+0447-а Ћирилчко мало слово Че у њиховом архаичном облику, са доњим продужетком слова на средини (нпр., {\fontspec{Monomakh Unicode}[StylisticSet=9] ч} уместо {\monomakh ч}).
\item Стилски скуп 10 (\emph{ss10}) приказује карактере U+0429 Ћирилчко велико слово Шча и U+0449 Ћирилчко мало слово Шча у њиховом модерном облику, са доњим продужетком слова на десној (нпр., {\fontspec{Monomakh Unicode}[StylisticSet=10] щ} уместо {\monomakh щ}).
\item Стилски скуп 11 (\emph{ss11}) приказује карактере U+044B Ћирилчко велико слово Јеру и U+A651 Ћирилчко мало слово Јеру са Дебелим јером са два глифа повезана (нпр., {\fontspec{Monomakh Unicode}[StylisticSet=11] ы} уместо {\monomakh ы}).
\item Стилски скуп 13 (\emph{ss13}) приказује карактер U+0463 Ћирилчко мало слово Јат са левим стаблом продуженим до основне линије (нпр., као {\fontspec{Monomakh Unicode}[StylisticSet=13] ѣ}). Запазите да то није исто као и U+A653 Ћирилчко мало слово јотован Јат.
\item Иста функционалност тих стилских скупова такође пружа у OpenType-у функција Стилске алтернативе (\emph{salt}).
\item[\XeTeXpicfile "deprecated.png" width 4mm] Претходне верзије фонта су пружале Стилски скуп 1 (\emph{ss01}), који је приказивао U+015E Латинско велико слово S са седиљом, U+0162 Латинско велико слово T са седиљом, те њихове мале еквиваленте, као U+0218 Латинско велико слово S са доњом запетом, U+021A Латинско велико слово T са доњом запетом, те њихове мале еквиваленте. Међутим, пошто се коришћење U+015E-а, U+0162-а и њихових малих еквивалената ради кодирања румунског текста сматра погрешним, ова функција је застарела. Корисници се снажно подстичу да претворе свој текст на нивоу кодне тачке да би користили исправне карактере за румунски прапопис. Међутим, ради компатибилности са текстом који је погрешно кодиран, та функција је још доступна.
\item[\XeTeXpicfile "deprecated.png" width 4mm] Стилски скуп 15 (\emph{ss15}), који пружа комбинована ћирилчка слова са аутоматским \emph{покритијем} где то санкционише синодални правопис, такође је застарео и може се уклонити. Корисници би требало да експлицитно кодирају \emph{покритије} као U+0487 Комбиновано ћирилчко покритије. За више информација погледајте \href{https://www.unicode.org/notes/tn41/}
{UTN 41: Church Slavonic Typography in Unicode}.
\end{itemize}

Две додатне функције су биле доступне само у SIL Graphite-у; међутим подршка SIL Graphite-а је прекинута. Ако Вам требају ове функције, погледајте \href{https://github.com/slavonic/fonts-cu-legacy/}{Legacy Fonts package}. 
\begin{itemize}
\item[\XeTeXpicfile "deprecated.png" width 4mm] Graphite-функција ,,Convert Arabic Digits to Church Slavonic`` (\emph{cnum}), када се укључи, аутоматски ће приказати западне цифре (,,арапске бројеве``) као ћириличке бројеве. Ово помаже, на пример, за нумерисање страница у софтверу који не подржава ћириличке бројеве.
\item[\XeTeXpicfile "deprecated.png" width 4mm] Graphite-функција ,,Convert HIP-6B Keystrokes to Church Slavonic Characters`` (\emph{hipb}), када се укључи, приказаће текст кодиран у наслеђеној кодној страници HIP као црквенословенски. Коришћење те функције се не препоручује и корисници се уместо тога подстичу да претворе текст кодиран по HIP-у на Јуникод.
\end{itemize}

\subsection{Shafarik}

Фонт Shafarik је специјализован фонт намењен за академско представљање старословенског текста написаног и на ћириличком писму и на глагољском. Документација за фонт се пружа у \href{https://www.ponomar.net/files/documentation_3.5.pdf}{засебној датотеци}.

\section{Украсни фонтови}

\subsection{Indiction Unicode}

Фонт Indiction Unicode репродукује украсни стил иницијала који користе синодална црквенословенска издања из касних 1800-их.

Првобитни Indyction фонт је развио Владислав В. Дорош и дистрибуира се
под именом Indyction UCS као део CSLTeX-а, који се лиценира под Јавном лиценцом пројекта \LaTeX{}. Фонт је поново кодирао за Јуникод и уредио Александар Андрејев, те се сада
дистрибуира као Indiction Unicode под Јавном лиценцом фонтова SIL. Намењен је за коришћење са \emph{буквицима} (иницијалима) у модерним црквенословенским издањима.

\begin{churchslavonic}
\cuLettrine Бл҃же́нъ мꙋ́жъ, и҆́же не и҆́де на совѣ́тъ нечести́выхъ, и҆ на пꙋтѝ грѣ́шныхъ не ста̀, и҆ на сѣда́лищи гꙋби́телей не сѣ́де: но въ зако́нѣ гдⷭ҇ни во́лѧ є҆гѡ̀, и҆ въ зако́нѣ є҆гѡ̀ поꙋчи́тсѧ де́нь и҆ но́щь. И҆ бꙋ́детъ ꙗ҆́кѡ дре́во насажде́ное при и҆схо́дищихъ во́дъ, є҆́же пло́дъ сво́й да́стъ во вре́мѧ своѐ, и҆ ли́стъ є҆гѡ̀ не ѿпаде́тъ: и҆ всѧ̑, є҆ли̑ка а҆́ще твори́тъ, ᲂу҆спѣ́етъ.
\par
\end{churchslavonic}

\subsection{Vertograd Unicode}

Vertograd Unicode (заснован на Vertograd UCS-у Влада Дороша) други је фонт за украсне иницијале и наслове. Фонт се уопште користио у предреволуционарним издањима синодалног црквенословенског језика. Запазите да нека слова која се типично не појављују у великом облику нису доступна. Поднесите извештај? ако Вам затреба слово које није доступно.

\renewcommand{\LettrineFontHook}{\vertograd \cuKinovarColor}
\begin{churchslavonic}
\cuLettrine Бл҃же́нъ мꙋ́жъ, и҆́же не и҆́де на совѣ́тъ нечести́выхъ, и҆ на пꙋтѝ грѣ́шныхъ не ста̀, и҆ на сѣда́лищи гꙋби́телей не сѣ́де: но въ зако́нѣ гдⷭ҇ни во́лѧ є҆гѡ̀, и҆ въ зако́нѣ є҆гѡ̀ поꙋчи́тсѧ де́нь и҆ но́щь. И҆ бꙋ́детъ ꙗ҆́кѡ дре́во насажде́ное при и҆схо́дищихъ во́дъ, є҆́же пло́дъ сво́й да́стъ во вре́мѧ своѐ, и҆ ли́стъ є҆гѡ̀ не ѿпаде́тъ: и҆ всѧ̑, є҆ли̑ка а҆́ще твори́тъ, ᲂу҆спѣ́етъ.
\par
\end{churchslavonic}

\subsection{Cathisma Unicode}

Cathisma Unicode се заснива на Kathisma UCS-у, који је осмислио Влад Дорош. Фонт се користи за наслове у много богослужбених издања \textenglish{XVIII--XX} века.

\newfontfamily{\cathisma}{Cathisma Unicode}
\begin{center}
\begin{tabular}{c}
{\fontsize{42pt}{2em} \cathisma ЧИ́НЪ ОУ҆́ТРЕНИ ВСЕНО́ЩНАГѠ БДѢ́НЇѦ} \\
\end{tabular}
\end{center}

\subsection{Oglavie Unicode}

Oglavie Unicode се заснива на Oglavie UCS-у, који је осмислио Вкад Дорош. Фонт се такође користи за украсне наслове у много богослужбених издања \textenglish{XVIII--XX} века.

\newfontfamily{\oglavie}{Oglavie Unicode}
\begin{center}
\begin{tabular}{c}
{\fontsize{42pt}{2em} \oglavie ТРЇѠ́ДЬ НО́ТНАГѠ ПѢ́НЇѦ} \\
\end{tabular}
\end{center}

\subsection{Pomorsky Unicode}

Фонт Pomorsky Unicode је верна репродукција украсног краснописног стила књига и наслова поглавља, који су највероватније развили током 18. века писари староверног Виговске испоснице (Выговская пустынь). Он се често види у песничким рукописима, богослужбеним рукописима,
хагиографским и полемичким делима Поморцијских и Федосејевских заједница, те је традиционалан и ,,органски`` стил утиснутог натписа којем недостаје икакав очигледан утицај из западноевропске и латинске тупографије. Фонт Pomorsky је првобитно осмислио Никита Симонс. Намењен је за употребу са \emph{буквицима} (иницијалима) и украсним насловима.

\newfontfamily{\pomorsky}{Pomorsky Unicode}
\newfontfamily{\simple}[StylisticSet=1]{Pomorsky Unicode}
\newfontfamily{\pomorskysalt}[Alternate=0]{Pomorsky Unicode}
\newfontfamily{\pomorskysalta}[Alternate=1]{Pomorsky Unicode}

\begin{center}
\begin{tabular}{c}
{\fontsize{48pt}{2em} \pomorsky ЧИ́НЪ ВЕЧЕ́РНИ.} \\
\end{tabular}
\end{center}

\subsubsection{Напредне функције фонта}

Неколико верзија много глифова се пружа у фонту. Украшени облици слова су подразумевани и пружају се на кодним тачкама ћирилчких великих слова; она би требало да се користе што је више могуће. Једноставнији облици се могу користити кад год словима треба мање украшен изглед, или када би се диајакритике могле сукобити са украшавањем (или када се украшавање једног карактера сукоби са украшавањем другог); те једноставни облици су доступни као
\verb+Stylistic Set 1+. Има неколико додатних карактера који су стилске варијанте, што се пружа као стилске алтернативе (\verb+salt+). Пошто је фонт намењен за иницијале и наслове, карактери малих слова нису доступни. Основни облици, ,,једноставни`` облици, те се све стилске алтернативе карактера показују у табели~\ref{pomor}.

\begin{table}[htbp]
\centering
\caption{Облици карактера које пружа Pomorsky Unicode \label{pomor}}
{\fontsize{38pt}{1.5em}
\begin{tabular}{cccc}
	{\pomorsky А}{\simple А}{\pomorskysalt А}	& {\pomorsky Б}{\simple Б} & {\pomorsky В}{\simple В} & {\pomorsky Г}{\simple Г} \\

	{\pomorsky Е}{\simple Е}	& {\pomorsky Ж}{\simple Ж} & {\pomorsky Ѕ}{\simple Ѕ} & {\pomorsky З}{\simple З} \\
	
	{\pomorsky И}{\simple И}	& {\pomorsky Й}{\simple Й} & {\pomorsky І}{\simple І} & {\pomorsky Ї}{\simple Ї} \\

	{\pomorsky К}{\simple К}{\pomorskysalt К}{\pomorskysalta К}	& {\pomorsky Л}{\simple Л} & {\pomorsky М}{\simple М} & {\pomorsky Н}{\simple Н} \\

	{\pomorsky О}{\simple О}	& {\pomorsky Ѻ}{\simple Ѻ} & {\pomorsky П}{\simple П} & {\pomorsky Р}{\simple Р}{\pomorskysalt Р}{\pomorskysalta Р} \\

	{\pomorsky С}{\simple С}	& {\pomorsky Т}{\simple Т} & {\pomorsky ОУ}{\simple ОУ} & {\pomorsky Ꙋ}{\simple Ꙋ} \\

	{\pomorsky Ф}{\simple Ф}	& {\pomorsky Х}{\simple Х} & {\pomorsky Ѡ}{\simple Ѡ} & {\pomorsky Ѽ}{\simple Ѽ} \\

	{\pomorsky Ѿ}{\simple Ѿ}	& {\pomorsky Ц}{\simple Ц} & {\pomorsky Ч}{\simple Ч} & {\pomorsky Ш}{\simple Ш} \\

	{\pomorsky Щ}{\simple Щ}	& {\pomorsky Ъ}{\simple Ъ} & {\pomorsky Ы}{\simple Ы} & {\pomorsky Ь}{\simple Ь} \\

	{\pomorsky Ѣ}{\simple Ѣ}	& {\pomorsky Ю}{\simple Ю} & {\pomorsky Ꙗ}{\simple Ꙗ}{\pomorskysalt Ꙗ} & {\pomorsky Ѧ}{\simple Ѧ} \\

	{\pomorsky Ѯ}{\simple Ѯ}	& {\pomorsky Ѱ}{\simple Ѱ} & {\pomorsky Ѳ}{\simple Ѳ} & {\pomorsky Ѵ}{\simple Ѵ} \\
\end{tabular}
}
\end{table}

\section{Технички фонтови: FiraSlav}

Firaslav је фонт фиксне ширине за црквенословенски језик, намењен за уређивање црквенословенског текста у уређивачу текста. Сви се дијакритички знаци и комбинована слова представљају као размакнути симболи и изглед фиксне ширине се одржава. Фонт укључује и обичну варијанту (FiraSlav Regular) и подебљану (FiraSlav Bold). Он је посебно користан за апликације за развој софтвера или веб локација.

\section{Познати проблеми}

Погледајте \href{https://github.com/typiconman/fonts-cu/issues/}{Issue Tracker}. Пре него што пријавите проблеме, проверите да ли Ваш софтвер правилно подржава OpenType. Сугеришемо да проверите очекивано понашање у \XeTeX{}-у или \LuaTeX{}-у.

\section{Признања}

Творци би желели захвалити следећим људима:

\begin{itemize}

\item Владиславу Дорошу, који је дозволио поновно кодирање на Јуникод и измену његовог фонта \href{http://irmologion.ru/fonts.html}{Hirmos}, што је довело до стварања фонта Ponomar.

\item Виктору Баранову пројекта \href{http://www.manuscripts.ru/}{Manuscripts}, који је дозволио поновно кодирање и измену свог фонта Menaion.

\item Михаилу Ивановичу који је помогао у стварању карактера за Саха
(јакутски језик), делимично преузетих из његовог фонта Sakha UCS.

\item Алексеју Крјукову, који је одговорио на разна питања о FontForge-у,
дозволио измену и препаковање свог фонта Monomachus, 
те? чија обимна документација за фонт \href{https://github.com/akryukov/oldstand/}{Old Standard} је прочитана и делимично поново употребљена.

\item Мајку Крутикову, који је саставио пакет фонтова \TeX{}.

\item Алесандру Ческинију, који је превео ово упутство на српскохрватски језик.
\end{itemize}

\end{document}
