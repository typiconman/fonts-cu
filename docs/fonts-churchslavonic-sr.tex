\documentclass[11pt]{ltxdoc}
\usepackage[usenames,dvipsnames,svgnames,table]{xcolor}
\usepackage{fontspec}
\usepackage{xltxtra}
% code borrowed from Polyglossia documentation -- Thanks!
\definecolor{myblue}{rgb}{0.02,0.04,0.48}
\definecolor{lightblue}{rgb}{0.61,.8,.8}
\definecolor{myred}{rgb}{0.65,0.04,0.07}
\usepackage[
    bookmarks=true,
    colorlinks=true,
    linkcolor=myblue,
    urlcolor=myblue,
    citecolor=myblue,
    hyperindex=false,
    hyperfootnotes=false,
    pdftitle={Church Slavonic fonts},
    pdfauthor={Aleksandr Andreev},
    pdfkeywords={Church Slavic, Church Slavonic, Old Church Slavonic, Old Slavonic, fonts, Unicode}
    ]{hyperref}
\usepackage[babelshorthands]{polyglossia}
\setmainlanguage[script=Cyrillic]{serbian}
\setotherlanguages{english,churchslavonic,romanian}
\usepackage{churchslavonic}
\usepackage{lettrine}

%% DOCUMENTATION VERSION AND RELEASE DATES
\def\filedate{22. априла 2017}
\def\fileversion{верзија 1.2}

%% fontspec declarations:
%\IfFontExistsTF{cmr10}{T}{F}

\setmainfont[Ligatures=Common]{Linux Libertine O}
\setsansfont{DejaVu Sans}
\setmonofont[Scale=MatchLowercase]{DejaVu Sans Mono}
\newfontfamily\churchslavonicfont[Script=Cyrillic,Ligatures=TeX,Scale=1.33333333,HyphenChar="005F]{PonomarUnicode.otf} 
\newfontfamily{\slv}[Scale=MatchLowercase]{Ponomar Unicode}
\newfontfamily{\ust}[Scale=MatchLowercase]{Menaion Unicode}
\newfontfamily{\ind}{Indiction Unicode}
% hack because Polyglossia doesn't support babelshortands for Serbian
\def\lq{„\relax}%
\def\rq{“\relax}%

\linespread{1.05}
%\lineskip=0pt
\lineskiplimit=0em
\frenchspacing
\EnableCrossrefs
\CodelineIndex
\RecordChanges
% COMMENT THE NEXT LINE TO INCLUDE THE CODE
\AtBeginDocument{\OnlyDescription}

\makeatletter
\def\ps@cuNum%
\let\@evenfoot\@oddfoot
}%
\def\cu@lettrine{\lettrine[lines=3,findent=0pt,nindent=0pt,lraise=-0.4]}
\def\cuLettrine{\cu@tokenizeletter\cu@lettrine}
\renewcommand{\LettrineFontHook}{\ind \cuKinovarColor}
\makeatother
\begin{document}

\title{Црквенословенски фонтови}
\author{Александр Андрејев \and Јури Шардт \and Никита Симонз\thanks{Коментари могу да буду усмерени на \href{mailto:aleksandr.andreev@gmail.com}{aleksandr.andreev@gmail.com}.}}
\date{\filedate \qquad \fileversion\\
\footnotesize (\textsc{pdf} датотека генерисана \today)}

\maketitle
\tableofcontents

\section{Увод}

Црквенословенски (такође познат као Старословенски; код ISO 639-2 |cu|) књижевни је језик који користе Словенски народи; тренутно га користе као литургијски језик Српска православна црква, друге локалне православне цркве, као и разни Католици византијског обреда и заједнице Старообредница. Пакет \texttt{fonts-churchslavonic} обезбеђује фонтове за приказивање црквенословенског текста.

Фонтови су дизајнирани да раде са текстом Unicode кодираним по UTF-8. Текст кодиран по застарелим кодним страницама (као и HIP и UCS) може да буде претворен у Unicode користећи одвојене пакете услужних програма. Погледајте \href{http://sci.ponomar.net/}
{Slavonic Computing Initiative website} за више информација.

\section{Лиценца}

Фонтови који су снабдевени у овом пакету имају дуплу лиценцу по ГНУ-овој Општој јавној лиценци (верзија 3 или каснија) и Отвореној лиценци фонта SIL (верзија 1.1 или каснија). Отворена лиценца фонта SIL преферира се, пошто је то лиценца FLOSS намењена фонтовима. Дупла лиценца по ГНУ ОЈЛ-и задржавана је да се омогућава уградња тих фонтова унутар апликације са лиценцом ОЈЛ и за компатибилност са другим пројектима.

\subsection{Прави текст}

Фонтови који су знабдевени у том пакету слободни су софтвер: Ви можете да их редистрибуишете и/или их модификујете, у целини или делимично, ИЛИ под условима ГНУ-ове Опште јавне лиценце објављене од стране Задужбине за слободни софтвер, било верзија 3 Лиценце, било (по вашем избору), било која каснија верзија ИЛИ под условима Слободне лиценце фонта SIL, верзија 1.1, или (по вашем избору) било која каснија верзија, без резервираних имена фонта.

Постоји један посебан изузетак: ако Ви креирате документ који користи било какав од тих фонтова, па уградите фонт или неизмењен део фонта унутар документа, фонт сам по себи не резултира тиме да финални документ буде покривен ГНУ-овом Општом јавном лиценцом.
Међутим, овај изузетак не поништава било које друге разлоге из којих би било могуће да је документ покривен ГНУ-овом Општом јавном лиценцом.
Ако модификујете било какав од тих фонтова, можете да продужујете тај изузетак вашој верзији фонтова, али нисте обавезни да то учините.
Ако не желите да то урадите, избришите тај изузетак из ваше верзије.

Као слободни софтвер, ти се фонтови снабдевају у нади да ће бити корисни, но БЕЗ ИКАКВЕ ГАРАНЦИЈЕ; без чак и имплицитне гаранције ТРЖИШНОСТИ или ПОГОДНОСТИ ЗА ОДРЕЂЕНУ СВРХУ. Видите ГНУ-ову Општу јавну лиценцу или Отворену лиценцу фонта SIL за више детаља.

Овај је документ лиценциран под Међународном лиценцом Кријејтив Комонс Ауторство-Делити под истим условима 4.0. Да видите копију ове лиценце, посетите \href{http://creativecommons.org/licenses/by-sa/4.0/}{CreativeCommons website}.

\section{Увод}

Пакет обезбеђује неколико фонтова који су намењени за рад са црквенословенским текстом разних рецензија и других текстова повезаних са Црквенословенским: модерни црквенословенски текст (\lq{}Синодални црквеноловенски\rq{}), историјски штампани црквенословенски текст и рукописни уставни црквенословенски текст (било у ћирилици било у глагољици) као и текст у Саха (Јакутској), Алеутској (дијалект Лисичијих острва), па Румунској (Молдавској) ћирилици, сви написани у црквеном писму. Опсег се разних фонтова слаже са смерницама за опсег фонта наведеним у \href{http://www.unicode.org/notes/tn41/}
{Unicode Technical Note \#41: Church Slavonic Typography in Unicode}. Генерално говорећи, он укључује највећи део карактера (но не сви) у блоковима Unicode-а, Ћирилица,  Допуна ћирилице, Продужена ћирилица-A, Продужена ћирилица-B, Продужена ћирилица-C (до Unicode-а 9.0), Глагољица, па Допуна глагољице. Међутим, карактери који нису коришћени у Црквенословенскоме нису укључени (осим неких карактера коришћених у модерном Руском, Украјинском, Белоруском, Српском и Македонском за потребе компатибилности са неким апликацијама).

\section{Инсталација и употреба}

Ако читате овај документ, онда сте вероватно већ преузели пакет фонтова. Можете да проверите да ли имате најновију верзију посећујући \href{https://sci.ponomar.net/}{Slavonic Computing Initiative website}.

\subsection{Формати фонтова}

Сви су фонтови тренутно расположиви у два формата:

\begin{description}
\item[\XeTeXpicfile "truetype.png" width 4mm] TrueType фонтови, или, прецизније,
\hyperlink{OT}{OpenType} фонтови са TrueType контурама;

\item[\XeTeXpicfile "opentype.png" width 4mm] \hyperlink{OT}{OpenType} фонтови са
PostScript контурама (такође звани фонтови OpenType-CFF).
\end{description}

\noindent Имајте на уму да фонтови у ова два формата имају две различите екстензије датотеке: \texttt{*.ttf} за TrueType фонтове, \texttt{*.otf} за OpenType-CFF-ове. Било True-Type верзије било OpenType-CFF подржавају исти скуп напредних \hyperlink{OT}{OpenType} функција.

OpenType-CFF фонтови користе PostScript контуре, базиране на Безјеове криве трећег реда (кубне), док TrueType фонтови користе криве другог реда (квадратне). Има и значајне разлике у хинтингу (прилагођавање решетки): TrueType инструкције теоретски омогућавају достизање бољег квалитета рендеровања на екрану него PostScript хинтинг. Међутим, пошто је квалитетни хинтинг веома тежак процес, било PostScript хинтинг било TrueType инструкција фонтова праве се аутоматски, онда висококвалитетно прилагођавање решетки није расположиво.

Имајте на уму да је могуће инсталирати било TrueType верзију било OpenType-CFF истовремено. За ову сврху, TrueType фонтови садржавају \lq{}TT\rq{} суфикс у своја поља имена фонта/имена фамилије. Пошто су сви фонтови нацртани по кубним сплајновима (а онда претворени у квадратне за TTF верзију), па пошто су TrueType инструкције аутоматски генерисане, OpenType-CFF формат може теоретски да Вам да бољи квалитет рендеровања на екрану, мада у већини ситуација то неће бити приметно. Штавише, имајте на уму да само TTF верзија подржава \hyperlink{Graphite}{SIL Graphite}, онда ће требати да користите TrueType фонтове ако се жели подршка Graphite-а. Следећа разматрања су такође долична:

\begin{itemize}

% The information about versions before OpenOffice.org 3.2 is outdated
% and no longer viewed as relevant
%\item In older versions of OpenOffice.org, OpenType-CFF fonts 
%were not properly embedded into PDF files. Moreover, under Unix-based
%systems, OpenOffice.org could not access such fonts at all, so using TTF
%versions was the only option. This was fixed in OpenOffice.org 3.2 and LibreOffice.

\item Старије верзије LibreOffice-а и Apache OpenOffice нису имали механизам за укључивање и искључивање напредних OpenType функција, онда ако је требало да користите опционалне типографске функције у овим апликацијама, морали сте да користите SIL Graphite, што је само доступно у TTF верзији. Ово је ограничење поправљено од LibreOffice-а 5.3, који сада има \href{https://wiki.documentfoundation.org/ReleaseNotes/5.3}{потпуну подршку OpenType-а}

\item Подршка је OpenType-CFF фонтова била лоша у Јави, а онда смо препоручили употребу TTF верзија у Јава програмским окружењима. Изгледа да је овај проблем решен у Oracle Јави SE 7.

\item У Мајкрософтовим производима, позиционирање OpenType глифова није подржано за глифове у Подручју приватне употребе или карактере изван Unicode 7.0 домета. Требало би да користите LibreOffice ако Вам треба позиционирање комбинованих глагољских карактера.

\end{itemize}

Имајте на уму да Мајкрософт Виндоус проверава присуство дигиталног описа у TrueType фонту, сматрајући да би то омогућавало да се разликују \lq{}стари\rq{} TrueType фонтови од \lq{}модерних\rq{} OpenType фонтови са TrueType контурама. Фонтови у том пакету садржавају неаутентични опис да би се варали Мајкрософтови производи тако да дозвољавају употребу додатних TrueType функција.

\subsection{Изворни пакети}

Можете такође да преузимате FontForge изворе за све фонтове са \href{https://github.com/typiconman/fonts-cu/}{GitHub репозиторијума}. То је корисно само ако планирате да уредите фонтове у \href{http://fontforge.sourceforge.net}{FontForge} уређивачу фонта. Опћенито, нећете добити никакво побољшање продуктивности из поновног склапања датотека фонтова, онда поновно склапање из извора није препоручљиво, осим ако имате стварну потребу да модификујете фонтове, на пример, да додате сопствене додатне глифове Подручју приватне употребе.

\section{Системски захтеви}

Сви су ти фонтови велики Unicode фонтови и захтевају оперативни систем и софтверско окружење усклађено са Unicode-ом. Изван окружења усклађеног са Unicode-ом, само ћете моћи, највише, да приступите првим 256 глифова фонта.

\subsection{Мајкрософт Виндоус}

Unicode је подржан од Виндоуса 95, међутим да користете 
OpenType-CFF верзију фонтова, треба Вам барем Виндоус 2000. Требаће Вам програм за обраду текста који може да се бави са документима базираним на Unicode-у, као и Microsoft Word 97 и касније верзије, или \href{https://www.libreoffice.org}{LibreOffice}. Молимо имајте на уму да је одржавање Apache OpenOffice-а прекинуто, онда препоручујемо употребу LibreOffice-а уместо тога. Ако користите \TeX{},
требаће Вам \TeX{} мотор усклађен са Unicode-ом, као и \XeTeX{} или \LuaTeX. 

Требаће Вам такође начин да куцате Unicode карактере које нису директно доступне са стандардних тастатура. Препоручујемо инсталирање црквенословенског или продуженог руског распореда тастатуре, расположивог са \href{https://sci.ponomar.net/keyboard.html}
{Slavonic Computing Initiative website-а}. Такође је могуће куцати карактере користећи услугу Мапу знакова Windows-а или путем тачке коде, но то није препоручљиво. 

\subsection{GNU/Linux}

Да бисте могли да управљате TrueType или OpenType фонтовима, требало би да Ваш систем има \href{https://freetype.sourceforge.net}{freetype} библиотеку инсталирану и укључену; то се нормално ради подразумевано у свим модерним дистрибуцијама. Требаће Вам уређивач текста усклађен са Unicode-ом, као и  \href{https://www.libreoffice.org}{LibreOffice}. Молимо имајте на уму да Apache OpenOffice није више подржан онда препоручујемо употребу LibreOffice-а уместо тога. Ако користите \TeX{}, требаће Вам \TeX{} мотор усклађен са Unicode-ом, као и \XeTeX{} или \LuaTeX.

Требаће Вам управљачки програм тастатуре да уносите Unicode карактере. Под GNU/Linux-ом, тиме управља |m17n| библиотека и база података. Видите \href{https://sci.ponomar.net/keyboard.html}{Slavonic Computing Initiative website} за више детаља.

\subsection{OS X}

Није сигурно.

\section{Подручје приватне употребе}

Unicode Подручје приватне употребе (PUA) скуп је трију спектра тачака кодова (од U+E000 до U+F8FF, Раван 15 и Раван 16) које гарантовано никада неће бити додељени карактерима од стране Unicode Consortium: могу да их користе трећа лица да додељују сопствене карактере. Slavonic Computing Initiative је успоставио индустријски стандард за алокацију карактера у PUA, који је подробно описан у  \href{https://www.ponomar.net/files/pua_policy.pdf}{PUA Allocation Policy}-у.

PUA у тим фонтовима садржава разне додатне глифове: контекстуалне алтернативе, стилске алтернативе, лигатуре, хипотетичне глифове и глифове за специфичну намену, разне глифове које нису још кодиране у Unicode-у и разне техничке симболе. Највећи део тих глифова (алтернативне глифове и лигатуре) доступан је путем \hyperlink{OT}{OpenType} и \hyperlink{Graphite}{SIL Graphite} функција. Дакле, генерално Вам не треба да приступите глифовима у PUA-у директно. Могу да постоје неки изузеци:

\begin{itemize}

\item Ако Вам треба да приступите карактерима који нису још кодирани у Unicode-у и глифовима за специфичну намену.

\item Ако Вам треба да приступите глифовима и лигатурама у застарелим системима који не подржавају OpenType или Graphite функције.

\item Ако сте компјутерски програмер и треба Вам да обрађујете глифове на низом нивоу без ослањања на OpenType: имати све алтернативе пресликане на PUA-у омогућава лакши начин приступање глифовима тачкама кодовима уместо да обрадите индексе глифова, који могу променити између верзија фонта.

\end{itemize}

\noindent За карактере пресликане у PUA-у и друга техничка разматрања молимо погледајте \href{https://www.ponomar.net/files/pua_policy.pdf}{PUA Allocation Policy}.

\section{OpenType Технологија}
\hypertarget{OT}{}\label{OT}

OpenType је технологија \lq{}интелигентног фонта\rq{} за напредну типографију развијену од стране Microsoft Corporation-а и Adobe Systems-а и базирану на TrueType формату фонта. Омогућава коректну типографију у комплексним писмима пак обезбеђује напредне типографске ефекте. Ово се постиже примењујући разне \textit{функције}, или \textit{ознаке}, описане у OpenType спецификацији. Неке од ових функција би требало да буду укључене подразумевано, док су друге сматране опционалним, па могу да буду укључене и искључене од корисника када то жели.


\subsection{На Microsoft Windows-у}

Да бисте користили ове напредне типографске функције, уз \lq{}интелигентан\rq{} фонт (као  фонт у том пакету), треба Вам апликацију усклађену са OpenType-ом. Не све апликације подржавају тренутно OpenType, па не све апликације које тврде да подржавају OpenType уствари подржавају све функције или обезбеђују интерфејс да приступају функцијама. Старије верзије Microsoft Uniscribe библиотека нису подржавале OpenType функције за ћирилицу и глагољицу, но почевши од Windows-а 7, то је решено.

Генерално говорећи, добиваћете најбоље резултате у \XeTeX{}-у или \LuaTeX{}-у користећи \texttt{fontspec} пакет или користећи софтвер стоног издаваштва као и Adobe InDesign. Највећи део OpenType функција такође је доступан у Microsoft Office-у 2010 и каснијим верзијама. И LibreOffice подржава OpenType функције почевши од верзије 4.1, па је подршка за укључивање и искључивање опционалних функција додата у верзији 5.3.

\subsection{На GNU/Linux-у}

Подршка је OpenType-а обезбеђена од стране HarfBuzz библиотеке обликовања, која је доступна преко FreeType-а, део од већине стандардних дистрибуција X Window
System-а. Дакле, OpenType ће бити расположив у било којој апликацији која користи FreeType, мада неким апликацијама недостаје интерфејс да укључују и искључују опционалне функције. Генерално говорећи, добиваћете најбоље резултате у \XeTeX{}-у or \LuaTeX{}-у користећи \texttt{fontspec} пакет. И LibreOffice подржава OpenType функције почевши од верзије 4.1, па је подршка за укључивање и искључивање опционалних функција додата у верзији 5.3. Молимо погледајте секцију
\hyperlink{LO}{Подршка напредних функција у LibreOffice-у}, доле.

\subsection{OpenType Функције}

\subsubsection{Позиционирање комбинованог знака}
\hypertarget{mark}{}

OpenType омогућава интелигентно позиционирање дијакрита: ако куцате слово које следи дијакрит, дијакрит ће бити постављени тачно изнад или испод слова; то је обезбеђено од стране \texttt{mark} функције. Додатно, \texttt{mkmk} функција се користи да ставља два знака у односу на један други, тако да додатан дијакрит може да буде прописно смешћен испод првога. Ово је понашање демонстрирано доле:

\begin{figure}[h]
\centering
\begin{tabular}{ll}
\large{  {\slv а}  + {\slv ◌́} → {\slv а́ } } &   \\
\large{ {\slv А}  + {\slv ◌́} → {\slv А́ } } & (позиционирање глифова путем \emph{mark} функције) \\
\large{ {\slv ◌ⷭ} + {\slv  ◌‍҇} → {\slv ◌ⷭ҇ } } & (позиционирање глифова путем \emph{mkmk} функције) \\
\end{tabular}
\end{figure}

Фонт обезбеђује прописне \texttt{mark} и \texttt{mkmk} полазне тачке за сва ћирилчина и глагољска слова и комбиноване знакове, омогућавајући Вам да их куцате у готово свој комбинацији (макар и оне које су невероватне). Највећи део OpenType рендерера (осим старијих верзија Adobe Cooltype библиотеке) подржавају ове функције, онда бисте могли да постижете исправно позиционирање у највећем делу апликација усклађених са OpenType-ом (на пример, у MS Word-у 2010 или новијим верзијама, LibreOffice 4.1-у или новијим верзијама, па \XeTeX{}).

\subsubsection{Компоновање и декомпоновање глифова}
\hypertarget{ccmp}{}

Функција компоновања / декомпоновања глифова (\texttt{ccmp}) користи се да би састављала два карактера у један глиф за боље обрађивање глифова. Ова функција се такође користи да ствара композитних форма базичног глифа са дијакритским знаковима када употреба \texttt{mark} и \texttt{mkmk} самих не може да постигне потребно позиционирање. Такође се користи да се креирају алтернативне облике глифова, као и алтернативна верзија Псилиа коришћена над великим словима и скраћеним формама слова Ук коришћеног с акценатским ознакама, као што је демонстрирано у доле наведеним примерима:

\begin{figure}[h]
\centering
\begin{tabular}{ll}
\large{ {\slv ◌҆} } $\rightarrow$ \large { {\slv  ◌ } } & (замена глифа користећи \emph{ccmp} функцију) \\
\large{ {\slv ◌҆}  + {\slv ◌̀} $\rightarrow$ {\slv ◌҆̀} } & (замена лигатуре користећи \emph{ccmp} функцију) \\
\large{ {\slv т}  + {\slv } + {\slv в} $\rightarrow$ {\slv т‍в } } & (замена лигатуре користећи \emph{ccmp} функцију) \\
\large{ {\slv ꙋ}  + {\slv ◌ⷯ} $\rightarrow$ {\slv ꙋⷯ } } & (контекстуална замена користећи \emph{ccmp} функцију) \\
\end{tabular}
\end{figure}

Генерално говорећи, функција \emph{ccmp} не би требало да буде искључена (и често не може да буде), па би онда ова функционалност требало да функционише прописно у било којој апликацији усклађеној са OpenType-ом. За више детаља о лигатурама, видите \href{http://www.unicode.org/notes/tn41/}{Unicode
Technical Note \#41: Church Slavonic Typography in Unicode}.

\subsubsection{Функције базиране на језику}

Функције базиране на језику као и функција \texttt{locl} (локализоване форме) обезбеђују приступ алтернативним формама глифова специфичним за сваки језик, као и алтернативне форме ћирилчиног слова I коришћеног на украјинском и белоруском:

\begin{figure}[h]
\centering
\begin{tabular}{ll}
\large{  {\slv і } } &  (црквенословенски текст) \\
\large{ {\slv і̇ } } & (украјински текст) \\
\end{tabular}
\end{figure}

Да се бисте користили тим функцијама, треба Вам апликацију усклађену са OpenType-ом која подржава спецификовање језика текста, на пример \XeTeX{} или
\LuaTeX{} користећи пакете \texttt{fontspec} или \texttt{polyglossia}. Пошто Вам многе софтверске апликације не омогућавају да спецификујете црквенословенски као језик текста, подразумено је да се користи фонт да приказује црквенословенски текст, а онда сви глифови имају црквенословенски изглед сем ако се не спецификује други језик.  

LibreOffice Вам омогућава да спецификујете да је текст на црквенословенском почевши од верзије 5.0. То ће Вам омогућавати да искориштавате друге функције, као и црквенословенско растављање на слогове (погледајте
\href{https://sci.ponomar.net/tools.html}{Slavonic Computing Initiative website}
за више информација). Microsoft Corporation не препознаје црквенословенски као валидан језик, онда нећете моћи да подесите црквенословенски као језик текста ни у једном Microsoft производу.\footnote{Молимо немојте да контактирате одржаваоце фонтова о овом питању. Уместо тога, жалите се Microsoft Корисничком сервису у Србији на 0700 300 300 или у Црној Гори на 080 081 110.}

\subsubsection{Стилске алтернативе и стилски скупови}

Стилске алтернативе (функција \texttt{salt}) обезбеђују варијантне облике глифова који могу да буду одабрани од корисника по вољи. Типично, то су глифови који разликују од базичног глифа само по графичном изгледу где употреба ових глифова не прати ниједна правила базирана на језику или типографији, него је радије просто украшавање. На пример, обезбеђује се следећа варијантна форма U+1F545-а Симбол за Марково поглавље: 

\begin{center}
\begin{tabular}{ccccc}
U+1F545	& \multicolumn{4}{c}{Алтернативни глифови} \\
\hline
{\slv \Huge 🕅} &	\textcolor{gray}{\slv \Huge } & \textcolor{gray}{\slv \Huge } & \textcolor{gray}{\slv \Huge } & \textcolor{gray}{\slv \Huge }  \\
\hline
\end{tabular}
\end{center}

Стилски скупови се користе да укључују групу стилских варијантних глифова, дизајнираних да хармонизују визуелно, па им замењују аутоматски подразумеване форме. OpenType дозвољава да се спецификују до 20 стилских скупова, означавајући их као функције \texttt{ss01}, \texttt{ss02}, \ldots{} \texttt{ss20}. 

Употреба стилских алтернатива и стилских скупова захтева апликацију усклађену са OpenType-ом која обезбеђује интерфејс да се искључују и укључују напредне функције (пошто су подразумевано ове функције искључене). Ово је могуће у \XeTeX{}-у
или \LuaTeX{}-у користећи пакет \texttt{fontspec} и у LibreOffice-у
(почевши од верзије 5.3) употребом специјалне синтаксе која приквачује потребну опцију имену фонта. Погледајте секцију
\hyperlink{LO}{Подршка напредних функција у LibreOffice-у}, доле.
У Microsoft Office-у 2010 и каснијим верзијама, Стилски скупови могу да буду искључени и укључени у оквиру |OpenType функције| на картицу |Више опција| дијалога |Фонт|.
Међутим, Micorosoft Office Вам не омогућава да одабрете вишеструке стилске скупове истовремено ни да приступите функцији |salt|. Ако је потребно, можете да приступите алтернативним глифовима тачком кодом са Подручја приватне употребе (PUA).
Међутим, ослањање на PUA као механизам размена додатака обесхрабрено је.

\section{Технологија SIL Graphite}

\hypertarget{Graphite}{}\label{Graphite}

\href{http://scripts.sil.org/Graphite}{Graphite} је технологија \lq{}интелигентног фонта\rq{} развијена од стране \href{http://www.sil.org}{SIL International}-а. Пошто, за разлику од OpenType-а, Graphite нема предефинисане функције, обезбеђује пројектанту способност да контролише суптилне типографске функције којима може да буде тешко или немогући управљати OpenType-ом. Уствари, Graphite је у неким аспектима моћнији од OpenType-а, мада ова додатна моћ није потребна за стандардну црквенословенску типографију. Додатно, док подршка OpenType функција често варира од апликације до апликације, Graphite се ослања на један мотор, па су онда све Graphite-ове функције подржане када год апликација подржава Graphite. Међутим, Graphite не ужива широку подршку: уз SIL-ов сопствен уређивач \href{http://scripts.sil.org/WorldPadDownload}{WorldPad} (апликација само за Windows која захтева библиотеку извршавања .NET), Graphite је подржан у LibreOffice-у (на GNU/Linux-у и Windows-у, почевши од
OpenOffice.org-а верзија 3.2; на OS X, почевши од LibreOffice-а верзија 5.3),
Mozilla Firefox-у (почевши од верзије 11), па \XeTeX{} (почевши од верзије 0.997).
Подршка Graphite-а није расположива у Microsoft Office-у.

Имајте на уму да није тренутно могуће додати Graphite табеле фонтовима OpenType-CFF. Стога је Graphite само подржан у TrueType верзијама фонта.

\subsection{Graphite у LibreOffice-у}

Погледајте секцију
\hyperlink{LO}{Подршка напредних функција у LibreOffice-у}, доле.

\subsection{Graphite у \XeTeX{}-у}

Подршка је за Graphite расположива у \XeTeX{}-у, што значи да су Graphite функције сада доступне из \TeX{} докумената. Штавише, могуће је укључити рендерер фонта са пакетом \texttt{fontspec}, што знатно поједностављује одабирање фонтова инсталираних од стране система у \XeTeX{}-у и \LuaTeX{}-у. Ова функционалност захтева најмање \TeX{} Live 2010 или Mik\TeX{} 2.9.

Можете да активирате Graphite режим рендирања за било какав фонт путем опције \texttt{Renderer} (њена би вредност требало да буде подешена на \texttt{Graphite})
у листу аргумената наредбе селекције фонтова. Пошто нема стандардних функцијских ознака у Graphite-у, функцијски идентификатори и њихове поставке преносе се опцији \texttt{RawFeature} како следи:

\begin{verbatim}
\newfontfamily{\graphA}
      [Renderer=Graphite, RawFeature=
          {Symbol for Mark's Chapter=Alternative 1}]
{Ponomar Unicode TT}
\end{verbatim}

\noindent Молимо погледајте документацију \texttt{fontspec}-а за више информација.

\hypertarget{LO}{}\label{LO}

\section{Подршка напредних функција у LibreOffice-у}

Подршка је OpenType функција расположива у LibreOffice-у и у свим дериватима OpenOffice.org-а почевши од верзије 3.2 OpenOffice.org-а.
Додатно, LibreOffice (као и Apache OpenOffice)
аутоматски препознаје фонтове који садржавају Graphite табеле.
За такве фонтове, Graphite је рендирање подразумевано укључено и OpenType табеле распореда ће бити игнорисане. OpenType и SIL Graphite не могу да се користе истовремено у LibreOffice-у, онда ако не желите распоред SIL Graphite, требало би да користите само OpenType-CFF верзије фонтова.

Док исправно позиционирање, везивање и супституције функционисаће аутоматски било у фонтовима OpenType у SIL Graphite, раније верзије LibreOffice-а нису имале ниједан механизам да искључе и укључе опционалне функције. Подршка за искључивање и укључивање Graphite функција расположива је почевши од LibreOffice-а верзија 4.1.
Међутим, нема графичког интерфејса који може да се користити да укључи и искључи ове функције. Уместо тога, развијена је била синтакса специјалнног проширеног имена фонта: да бисте активирали опционалну функцију, њен идентификатор, а затим знак једнакости и идентификатор жељене поставке, директно се прилажу нисци имена фонта. Амперсанд се користи да издваја различите парове функција/поставки.

На пример, следећи \lq{}фонт\rq{} би требало да се користи да би се укљичила SIL Graphite-ова функција |mark| и приказали
алтернативни глифови за Симбол за Маркова поглавља U+1F545:

\begin{verbatim}
Ponomar Unicode TT:mark=1
\end{verbatim}

\noindent за први алтернативни глиф, \texttt{mark=2} за други алтернативни глиф, и тако даље.

Почевши од LibreOffice-а верзија 5.3, иста синтакса може да се користи такође да искључује и укључује опционалне OpenType функције, као и Стилске алтернативе (|salt|)
и Стилски скупови (|ss01|, ...) у верзијама OpenType фонтова.
На пример, да укључите Стилски скуп 1 (|ss01|) у фонту Ponomar Unicode, промените име фонта на следећи начин:

\begin{verbatim}
Ponomar Unicode:ss01=1
\end{verbatim}

\noindent Ова функција није расположива у Apache OpenOffice-у; пошто Apache OpenOffice није добро одржан, сугеришемо да корисници мигрирају у LibreOffice.

Та функционалност ће бити корисна за кориснике LibreOffice-а што се ослањају на аутоматско растављање на слогове. Пошто LibreOffice нема
\href{https://bugs.documentfoundation.org/show_bug.cgi?id=85731}
{ниједан механизам да подеси карактер за цртицу}, фонтови Ponomar Unicode
и Monomakh Unicode обезбеђују подвлаку као карактер за цртицу путем Стилског скупа 1 у OpenType-у и путем функције |hyph| у SIL Graphite-у.

Свакако је директно модификовање фонта веома неугодно, пошто је тешко сетити се кратких ознака и нумеричких вредности коришћених за идентификаторе функција/поставки у различитим фонтовима. Нажалост, тренутно нема графичког интерфејса да подржа исључивање и укључивање OpenType и SIL Graphite функција.

%You may try to install the \href{https://github.com/thanlwinsoft/groooext}
%{Graphite Font Extension}, which provides a dialog for easier feature selection.
%However, this extension has not been maintained since the passing of its
%developer in 2011, and so may not work correctly in later versions of LibreOffice.
%If you experience problems with Graphite features, you may get better
%results accessing glyphs directly by codepoint from the Private Use Area,
%though this is not recommended.

\section{Ponomar Unicode}

Ponomar Unicode је фонт који репродукује словни облик синодалних црквенословенских издања од почетка 20. века. Намењен је за рад са модерним црквенословенским текстовима (синодални црквенословенски). Ponomar Unicode је базиран на фонту Hirmos UCS дизајнираном од Влада Дороша, но је модификован од аутора овог пакета. Примери текста сложеног по Ponomar Unicode-у наведени су у наставку.

%\lineskip=0pt
%\lineskiplimit=-5em

\subsection{Синодални црквенословенски}

\begin{churchslavonic}
Бл҃же́нъ мꙋ́жъ, и҆́же не и҆́де на совѣ́тъ нечести́выхъ, и҆ на пꙋтѝ грѣ́шныхъ не ста̀, и҆ на сѣда́лищи гꙋби́телей не сѣ́де: но въ зако́нѣ гдⷭ҇ни во́лѧ є҆гѡ̀, и҆ въ зако́нѣ є҆гѡ̀ поꙋчи́тсѧ де́нь и҆ но́щь. И҆ бꙋ́детъ ꙗ҆́кѡ дре́во насажде́ное при и҆схо́дищихъ во́дъ, є҆́же пло́дъ сво́й да́стъ во вре́мѧ своѐ, и҆ ли́стъ є҆гѡ̀ не ѿпаде́тъ: и҆ всѧ̑, є҆ли̑ка а҆́ще твори́тъ, ᲂу҆спѣ́етъ. Не та́кѡ нечести́вїи, не та́кѡ: но ꙗ҆́кѡ пра́хъ, є҆го́же возмета́етъ вѣ́тръ ѿ лица̀ землѝ. Сегѡ̀ ра́ди не воскре́снꙋтъ нечести́вїи на сꙋ́дъ, нижѐ грѣ̑шницы въ совѣ́тъ првⷣныхъ. Ꙗ҆́кѡ вѣ́сть гдⷭ҇ь пꙋ́ть првⷣныхъ, и҆ пꙋ́ть нечести́выхъ поги́бнетъ.
\end{churchslavonic}

\subsection{Кијевски црквенословенски}

Кијевски црквенословенски користи неколико варијантних форма глифова, као и U+1C81 Дугоноги Де ({\slv ᲁ}) и U+A641 Варијантни Зе ({\slv ꙁ}):

\begin{churchslavonic}
Бл҃же́нъ мꙋ́жъ, и҆́же не и҆́ᲁе на совѣ́тъ нечести́выхъ, и҆ на пꙋтѝ грѣ́шныхъ не ста̀, и҆ на сѣᲁа́лищи гꙋби́телей не сѣ́ᲁе: но въ зако́нѣ гᲁⷭ҇ни во́лѧ є҆гѡ̀, и҆ въ зако́нѣ є҆гѡ̀ поꙋчи́тсѧ де́нь и҆ но́щь. И҆ бꙋ́ᲁетъ ꙗ҆́кѡ дре́во насажᲁе́ное при и҆схо́ᲁищихъ во́ᲁъ, є҆́же плоᲁъ сво́й да́стъ во вре́мѧ своѐ, и҆ ли́стъ є҆гѡ̀ не ѿпаᲁе́тъ: и҆ всѧ̑, є҆ли̑ка а҆́ще твори́тъ, ᲂу҆спѣ́етъ. Не та́кѡ нечести́вїи, не та́кѡ: но ꙗ҆́кѡ пра́хъ, є҆го́же воꙁмета́етъ вѣ́тръ ѿ лица̀ землѝ. Сегѡ̀ ра́ᲁи не воскре́снꙋтъ нечести́вїи на сꙋ́ᲁъ, нижѐ грѣ̑шницы въ совѣ́тъ првⷣныхъ. Ꙗ҆́кѡ вѣ́сть гᲁⷭ҇ь пꙋ́ть првⷣныхъ, и҆ пꙋ́ть нечести́выхъ поги́бнетъ.
\end{churchslavonic}

\subsection{Други језици}

Фонт Ponomar Unicode може такође да се користи да слаже литургијске текстове на другим језицима који користе црквенску азбуку. Три су такви примери у потпуности подржани од стране фонта: румуњски (молдавски) на његовој азбуци, алеутски (дијалект Лисичијих острва или источни) на његовој азбуци, па јакутски (саха) као што је написан на азбуци створеном од Епископа Диониса (Хитрова).

\noindent Ево примера Оченаша на румунској (молдавској) ћирилици: \\

\begin{churchslavonic}
Та́тъль но́стрꙋ ка́реле є҆́щй ꙟ҆ Че́рюрй: ᲃ︀фн҃цѣ́скъсе Нꙋ́меле тъ́ꙋ: ві́е ꙟ҆пъръці́ѧ та̀: фі́е во́ѧ та̀, прекꙋ́мь ꙟ҆ Че́рю̆ шѝ пре пъмѫ́нть. Пѫ́йнѣ но́астръ чѣ̀ ᲁепꙋ́рꙋрѣ ᲁъ́не но́аѡ а҆́стъꙁй. Шѝ не ꙗ҆́ртъ но́аѡ греша́леле но́астре, прекꙋ́мь шѝ но́й є҆ртъ́мь греши́цилѡрь но́щри. Ши́ нꙋ́не ᲁꙋ́че пре но́й ꙟ҆ и҆спи́тъ. Чѝ не и҆ꙁбъвѣ́ще ᲁе че́ль ръ́ꙋ. \\
\end{churchslavonic}

\noindent А ево примера Оченаша на алеутској ћирилици: \\

\begin{churchslavonic}
Тꙋмани́нъ А́даԟъ! А҆́манъ акꙋ́х̑тхинъ и́нинъ кꙋ́ҥинъ, А́са́нъ амчꙋг̑а́сѧ́да́г̑та, Аҥали́нъ а҆ԟа́г̑та, Анꙋхтана́тхинъ малга́г̑танъ и́нимъ кꙋ́ганъ ка́юхъ та́намъ кꙋ́ганъ. Ԟалга́дамъ анꙋхтана̀ ҥи̑нъ аԟача́ ꙋ̆а҆ѧ́мъ: ка́юхъ тꙋма́нинъ а́д̑ꙋнъ ҥи̑нъ игни́да, а҆ма́кꙋнъ тꙋ́манъ ка́юхъ малгалиги́нъ ҥи̑нъ ад̑ꙋг̑и́нанъ игнида́кꙋнъ: ка́юхъ тꙋ́манъ сꙋглатачх̑и́г̑анах̑тхинъ, та́г̑а ад̑алю́дамъ илѧ́нъ тꙋ́манъ аг̑г̑ича. \\
\end{churchslavonic}

\noindent А ево примера Оченаша на јакутском (Саха): \\

\begin{churchslavonic}
Халланнаръ юрдюлѧригѧрь баръ агабытъ бисенѧ ! Свѧтейдѧннинь а̄тыҥъ эенѧ ; кѧллинь царстваҥъ эенѧ ; сирь юрдюгѧрь кёҥюлюҥь эенѧ , халланъ юрдюгѧрь курдукъ боллунъ ; бюгюҥю кюннѧги асыръ аспытынъ бисенинь кулу бисеха бюгюнь ; бисиги да естѧрбитинь халларъ бисеха , хайтахъ бисиги да халларабытъ беэбить естѧхтѧрбитигѧрь ; килѧримѧ да бисигини альԫархайга ; хата быса бисигини албынтанъ . \\
\end{churchslavonic}

%\noindent Ево примера користећи Типиконове симболе из књиге Никите Сирникова {\slv Клю́чъ къ церко́вномꙋ ᲂу҆ста́вꙋ}:

%\begin{churchslavonic}
%і\textcolor{red}{і}꙼̇ ⧟̇҃ іѡа́ннꙋ і̲꙼ на лиⷮ бл҃жеⷩ҇, па́влꙋ пѣⷭ҇ г҃. а і҆оа́ннꙋ ѕ҃.

%и᷷͏҃і і\textcolor{red}{і}꙼̇ ⧟̲̇҃ кири́лꙋ і̲꙼ коⷣ и҆ и҆́коⷭ҇ о҆́бщїй и҆ коⷣ а҆фана́сїю.

%д҃ 🤉 іі̲ и҆си́дорꙋ ⹇ гео́ргїю кири́лѣ ⹉
%\end{churchslavonic}

\subsection{Карактеристике фонта}

Ponomar Unicode ставља неке карактере у Подручју приватне употребе (PUA).
За генерално PUA пресликавање, молимо погледајте
\href{https://www.ponomar.net/files/pua_policy.pdf}{Политику о алокацији PUA-а}.

Уз општа пресликавања PUA-а, неки карактери су били додељени отвореном спектру PUA-а. То су:

\begin{itemize}
\item U+F400 \textendash{} Алтернативни глифови Допусне вишејезичне равни (SMP): ова секција садржава копије у Основној вишејезичкој равни (BMP) за подршку у застарелим апликацијама. Тренутно, расположиви су следећи: U+F400 - U+F405 \textendash{} Типиконови симболи (копије од U+1F540 до U+1F545).
\item U+F410 \textendash{} Форме представљања: садржавају разне форме представљања и лигатуре које фонт користи интерно. Уопште, ово нису намњене да буду позване од корисника или од стране екстерних апликација.
\item U+F420 \textendash{} Језичке алтернативе: садржава алтернативне облике које су специфичне за сваки језик. За тренутак, ово су модерни интерпункцијски облици за употребу са латинским карактерима. То нису намењени да буду позвани екстерно.
\item U+F441 и даље \textendash{} стилске алтернативе латинских карактера (готичке форме). Ово могу да буду позване путем Стилског скупа 2, али, ако је потребно, могу да буду позване са PUA-а директно. Оне су пресликане у истом редоследу као и у Основном латинском блоку, почевши од U+F441-а (што одговара U+0041-у Латинско велико слово A). Уз репертоар Основног латинског, имамо и: U+F4DE \textendash{} готички Торн; U+F4FE \textendash{} Мало слово готички Торн; па U+F575 \textendash{} готички дуги S
\end{itemize}

Фонт обезбеђује неколико лигатура, које су направљене уводећи ознаку форматирања без ширине без прелома (U+200D) између два карактера. Списак је лигатура наведен на табели~\ref{ligs1}.

\begin{table}[htbp]
\centering
\caption{Лигатуре расположиве у Ponomar Unicode-у \label{ligs1}}
\begin{tabular}{lcc}
Име	& Секвенца	& Изглед \\
\hline
Лигатура А-У	& U+0430 U+200D U+0443	& {\slv{\large а‍у}}	\\
Лигатура Л-У	& U+043B U+200D U+0443 & {\slv{\large л‍у}}	\\
Лигатура Т-В	& U+0442 U+200D U+0432	& {\slv{\large т‍в}}	\\
\hline
\end{tabular}
\end{table}

\noindent У OpenType-у, дефинисано је неколико Стилских алтернатива. Ово је наведено на табели~\ref{salt1}.
Уз додатне глифове за Симбол за Марково поглавље, функција обезбеђује алтернативне декоративне форме слова U+0423 У које изгледа точно као U+A64A Ук (ова се употреба налази у неким публикацијама),
па алтернативну форму за U+0404 Широки Је за употребу у контекстима где треба да се разликује од U+0415 Је (понајвише за украјински текст по стилу
црквенословенског фонта).

\newfontfamily{\salt}[Alternate=0]{Ponomar Unicode}
\newfontfamily{\salta}[Alternate=1]{Ponomar Unicode}
\newfontfamily{\saltb}[Alternate=2]{Ponomar Unicode}
\newfontfamily{\saltc}[Alternate=3]{Ponomar Unicode}
\newfontfamily{\saltd}[Alternate=4]{Ponomar Unicode}
\newfontfamily{\salte}[Alternate=5]{Ponomar Unicode}
\newfontfamily{\saltf}[Alternate=6]{Ponomar Unicode}
\newfontfamily{\saltg}[Alternate=7]{Ponomar Unicode}

\begin{table}[htbp]
\centering
\caption{Стилске алтернативе у Ponomar Unicode-у \label{salt1}}
\begin{tabular}{lccccc}
	& Base Form	& \multicolumn{4}{c}{Алтернативе форме} \\
\hline
U+1F545	& {\slv{\large 🕅 }}	& {\salt\large 🕅} & {\salta\large 🕅} & {\saltb\large 🕅} & {\saltc\large 🕅}  \\
			&				& {\saltd\large 🕅} & {\salte\large 🕅} & {\saltf\large 🕅} & {\saltg\large 🕅} \\
U+0423		& {\slv\large У}	& {\salt\large У} \\
U+040E		& {\slv\large Ў}	& {\salt\large Ў} \\
U+0404		& {\slv\large Є}	& {\salt\large Є} \\
\hline
\end{tabular}
\end{table}

За ћирилчина слова, функција стилских алтернатива омогућава такође приступ скраћеним формама; редослед је алтернативних форма увек: ниже скраћивање, горње скраћивање, лево скраћивање, десно скраћивање. Табела~\ref{trunc} демонстрира које су скраћене форме расположиве. Генерално говорећи, скраћивањем би требало да се управља аутоматски софтвером за стоно издаваштво и \TeX{}-ом, мада је то тешко остварити.

\begin{table}[htbp]
\centering
\caption{Скраћене форме доступне путем функције Стилских алтернатива
у Ponomar Unicode-у \label{trunc}}
\begin{tabular}{lccccc}
	& Основна форма	& \multicolumn{4}{c}{Скраћене форме} \\
\hline
U+0440	& {\slv{\large р }}	& {\salt\large р} &  \\
U+0443  & {\slv{\large у }}	& {\salt\large у} &  \\
U+0444  & {\slv{\large ф }}	& {\salt\large ф} & {\salta\large ф} \\
U+0445  & {\slv{\large х}}	& {\salt\large х} & {\salta\large х} & {\saltb\large х}  \\
U+0446  & {\slv{\large ц }}	& {\salt\large ц} &  \\
U+0449  & {\slv{\large щ }}	& {\salt\large щ} &  \\
U+0471  & {\slv{\large ѱ }}	& {\salt\large ѱ} &  {\salta\large ѱ}\\
U+A641  & {\slv{\large ꙁ }}	& {\salt\large ꙁ} &  \\
U+A64B  & {\slv{\large ꙋ }}	& {\salt\large ꙋ} & {\salta\large ꙋ} & {\saltb\large ꙋ} \\
\hline
\end{tabular}
\end{table}

Стилски скуп 1 (|ss01|) се обезбеђује као привремено решење за\href{https://bugs.documentfoundation.org/show_bug.cgi?id=85731}
{LibreOffice Bug 85731}, што Вам не омогућава да спецификујете карактер за растављање на слогове у LibreOffice-у. Када је укључен, замењује сва појављивања U+002D-а Цртица-минус и U+2010-а Цртица U+005F-ом Ниска линија (подвлака) за употребу као карактер за растављање на слогове. Молимо имајте на уму да ће ова функција бити напуштена чим потребна функционалност буде додата LibreOffice-у.

Дефинисано је и Стилско скупа 2 (\lq{}ss02\rq{}), Готичке форме.
Када је овај стилски скуп укључен, латинска слова се појављују на готици насупрот њихових модерних форма. Ово је корисно за слагање латинског текста поред црквенословенског у неким контекстима. Погледајте следећи пример:

\newfontfamily{\blackletter}[StylisticSet=2]{Ponomar Unicode}

\begin{figure}[h]
\centering
\begin{tabular}{ll}
Нормалан &
{\slv \large The quick brown fox. 1234567890. А҆ сїѐ по слове́нски. } \\
Готички & 
{\blackletter \large The quick brown fox. 1234567890. А҆ сїѐ по слове́нски. } \\
\end{tabular}
\end{figure}

\noindent Имајте на уму да од верзије 2.0 фонта, ASCII цифре (обично називане
\lq{}арапски бројеви\rq{}) обезбеђују се на латинској форми. Користите Стилски скуп 2 да приступите готичким формама, ако је потребно.

\subsection{SIL Graphite Функције}

\newfontfamily{\graph}[Renderer=Graphite]{Ponomar Unicode TT}

SIL Graphite функције у фонту обезбеђују исту функционалност као и OpenType функције. Функција \lq{}Symbol for Mark's Chapter\rq{} (\lq{}mark\rq{}) обезбеђује алтернативе за U+1F545 Симбол за Марково поглавље, попут salt функције у OpenType-у. 
Следеће вредности производе резултате дате у табели~\ref{ponograph}.

\newfontfamily{\graphA}[Renderer=Graphite, RawFeature={Symbol for Mark's Chapter=Alternative 1}]{Ponomar Unicode TT}
\newfontfamily{\graphB}[Renderer=Graphite, RawFeature={Symbol for Mark's Chapter=Alternative 2}]{Ponomar Unicode TT}
\newfontfamily{\graphC}[Renderer=Graphite, RawFeature={Symbol for Mark's Chapter=Alternative 3}]{Ponomar Unicode TT}
\newfontfamily{\graphD}[Renderer=Graphite, RawFeature={Symbol for Mark's Chapter=Alternative 4}]{Ponomar Unicode TT}
\newfontfamily{\graphE}[Renderer=Graphite, RawFeature={Symbol for Mark's Chapter=Alternative 5}]{Ponomar Unicode TT}
\newfontfamily{\graphF}[Renderer=Graphite, RawFeature={Symbol for Mark's Chapter=Alternative 6}]{Ponomar Unicode TT}
\newfontfamily{\graphG}[Renderer=Graphite, RawFeature={Symbol for Mark's Chapter=Alternative 7}]{Ponomar Unicode TT}
\newfontfamily{\graphH}[Renderer=Graphite, RawFeature={Symbol for Mark's Chapter=Alternative 8}]{Ponomar Unicode TT}

\begin{table}[htbp]
\centering
\caption{Вредности функције Симболо за Марково поглавље (\lq{}mark\rq{}) у Ponomar Unicode-у \label{ponograph}}
\begin{tabular}{lcccc}
\hline
	& Основна форма	& Алтернатива 1	& Алтернатива 2	& Алтернатива 3	\\
U+1F545	& {\graph{\large 🕅 }}	& {\graphA{\large 🕅}} & {\graphB{\large 🕅}} & {\graphC{\large 🕅}} \\
& Алтернатива 4	& Алтернатива 5	& Алтернатива 6	& Алтернатива 7 \\
& {\graphD{\large 🕅}} & {\graphE{\large 🕅}} & {\graphF{\large 🕅}} & {\graphG{\large 🕅}} \\
	& Алтернатива 8 \\
& {\graphH{\large 🕅}} \\
\hline
\end{tabular}
\end{table}

\noindent Следеће додатне Graphite функције обезбеђују се (оне дуплирају функционалност OpenType функција):

\begin{itemize}
\item Функција \lq{}Truncation\rq{} (\lq{}trnc\rq{}) обезбеђује исту функционалност као и стилске алтернативе (за скраћење) горе наведене. Могуће вредности су:
\verb+1+ (ниско скраћење), \verb+2+ (горње скраћење),
\verb+3+ (лево скраћење) and \verb+4+ (десно скраћење).

\item Функција \lq{}Use blackletter characters for Latin\rq{} (\lq{}blck\rq{})
обезбеђује исту функционалност као и Стилски скуп 2 у OpenType-у (погледајте горе).
Могуће вредности су |0| (не) и |1| (да).

\item Функција \lq{}Use alternative form of U\rq{} (\lq{}altu\rq{})
обезбеђује алтернативну форму слова U+0423 У која изгледа точно као U+A64A Ук. Могуће вредности су |0| (не) и |1| (да).

\item Функција \lq{}Cyrillic i has dot\rq{} (\lq{}doti\rq{})
обезбеђује локализовану форму U+0456-а ћирилчини I за употребу у украјинском тексту. Могуће вредности јесу |0| (не) и |1| (да).

\item Функција \lq{}Use underscore for hyphenation\rq{} (\lq{}hyph\rq{})
обезбеђује исту функционалност као и Стилски скуп 1. Замењује сва појављивања U+002D-а Цртица-минус и U+2010-а Цртица U+005F-ом Ниска линија (подвлака) за употребу као карактер растављања на слогове. Могуће вредности су |0| (не) и |1| (да). Молимо имајте на уму да  да ће ова функција бити напуштена чим потребна функционалност буде додата LibreOffice-у.
\end{itemize}

\section{Fedorovsk Unicode}

Fedorovsk Unicode је базиран на фонту Fedorovsk дезајнираном од Никите Симонза.
Поново је кодиран за Unicode, са OpenType и Graphite функцијама додатим од Александра Андрејева. Фонт Fedorovsk је намераван да репродукује фонт штампованих издања Ивана Фјодорова објављених у Москви, на пример, Апостол од 1564.е Фонт је примарно намењен за слагање прениконских (старообредствичких) литургијских текстова или за рад са таквим текстовима у академском контексту.

\subsection{Узорци текста}
\newfontfamily{\rright}[StylisticSet=1,HyphenChar="200B]{Fedorovsk Unicode} %% FIX THIS

\subsubsection{Апостол од Ивана Фјодорова}

\begin{churchslavonic}
{\Large \rright
\textcolor{red}{П}е́рвᲂе ᲂу҆́бо︀ сло́во︀ сᲂтвᲂри́хъ о҆ всѣ́хъ , ѽ , ѳео҆́филе , о҆ ниⷯже начѧ́тъ і︮с︯ , твᲂри́тиже и҆ ᲂу҆чи́ти . д︀о︀ него́же дн҃е , запᲂвѣ́д︀авъ а҆пⷭ҇лᲂмъ дх҃ᲂмъ ст҃ыⷨ , и҆́хже и҆ꙁбра̀ вᲂзнесе́сѧ . преⷣ ни́миже и҆ пᲂста́ви себѐ жи́ва по страд︀а́нїи свᲂе҆́мъ . во︀ мно́зехъ и҆́стинныхъ зна́менїи҆хъ . дн҃ьми четы́ридесѧтьми ꙗ҆влѧ́ꙗсѧ и҆́мъ и҆ гл҃ѧ ꙗ҆́же о҆ црⷭ҇твїи бж҃їи . сни́миже и҆ ꙗ҆д︀ы̀ , пᲂвелѣва́ше и҆́мъ ѿ і҆е҆рᲂсали́ма не ѿлꙋча́тисѧ . но̑ жда́ти о҆бѣтᲂва́нїе ѿч︮е︯е , е҆́же слы́шасте ѿ́ менѐ . ꙗ҆́кѡ і҆ѡ҃а́ннъ ᲂу҆́бо︀ крⷭ҇ти́лъ е҆́сть вᲂдо́ю . вы́же и҆́мате крести́тисѧ дх҃ᲂмъ ст҃ы́мъ , не по мно́ꙁѣхъ си́хъ д︀︮н︯еⷯ .
}
\end{churchslavonic}

\subsubsection{Цветни триодион}
\newfontfamily{\lleft}[StylisticSet=2,HyphenChar="200B]{Fedorovsk Unicode}

\begin{churchslavonic}
{\Large \lleft
\textcolor{red}{стⷯры па́сцѣ . гла́съ , є҃ .} Д\textcolor{red}{а вᲂскрⷭ҇нетъ бг҃ъ ,꙳ и҆ разы́дꙋтсѧ вразѝ є҆гѡ̀ .}
Па́сха сщ҃е́ннаѧ на́мъ дне́сь пᲂказа́сѧ , па́сха но́ва ст҃а́ѧ , па́сха таи́нственнаѧ , па́сха всечестна́ѧ , па́сха хрⷭ҇та̀ и҆зба́вителѧ , па́сха непᲂро́чнаѧ , па́сха вели́каѧ , па́сха вѣ́рнымъ , па́сха двѣ́ри ра́йскїѧ на́мъ ѿверза́ющаѧ , па́сха всѣ́хъ ѡ҆сщ҃а́ющаѧ вѣ́рныхъ .
}
\end{churchslavonic}

\subsection{Функције OpenType}

Фонт обезбеђује неколико лигатура, које су направљене уводећи Ознаку форматирања без ширине без прелома (U+200D) између два карактера. Листа лигатура је наведена у табели~\ref{ligs2}.

\begin{table}[htbp]
\centering
\caption{Лигатуре расположиве у Fedorovsk Unicode-у \label{ligs2}}
\begin{tabular}{lcc}
Име	& Секвенца	& Изглед \\
\hline
Лигатура А-У	& U+0430 U+200D U+0443	& {\lleft{\large а‍у}}	\\
Лигатура Ел-У	& U+043B U+200D U+0443 & {\lleft{\large л‍у}}	\\
Лигатура А-Ижица & U+0430 U+200D U+0475	& {\lleft{\large а‍ѵ}}	\\
Лигатура Ел-Ижица & U+043B U+200D U+075 & {\lleft{\large л‍ѵ}}	\\
Лигатура Те-Ве	& U+0442 U+200D U+0432	& {\lleft{\large т‍в}}	\\
Лигатура Ер-Јат	& U+0440 U+200D U+0463 & {\lleft{\large р‍ѣ}}	\\
\hline
\end{tabular}
\end{table}

\noindent У OpenType-у, дефинисано је неколико стилских алтернатива.
Наведено је у табели~\ref{salt2}. Уз обезбеђивање алтернативних облика глифова за U+1F545 Симбол for Mark's Chapter, они Вам омогућавају да контролишете позиционирање дијакритичких ознака над одређеним словима.

\newfontfamily{\glyphfont}{Fedorovsk Unicode}
%\newfontfamily{\salt}[Alternate=0]{Fedorovsk Unicode}
%\newfontfamily{\salta}[Alternate=1]{Fedorovsk Unicode}
%\newfontfamily{\saltb}[Alternate=2]{Fedorovsk Unicode}
%\newfontfamily{\saltc}[Alternate=3]{Fedorovsk Unicode}
%\newfontfamily{\saltd}[Alternate=4]{Fedorovsk Unicode}
%\newfontfamily{\salte}[Alternate=5]{Fedorovsk Unicode}
%\newfontfamily{\saltf}[Alternate=6]{Fedorovsk Unicode}

\begin{table}[htbp]
\centering
\caption{Стилске алтернативе у Fedorovsk Unicode-у \label{salt2}}
\begin{tabular}{lcccccccc}
	& Основна форма	& \multicolumn{7}{c}{Алтернативне форме} \\
\hline
U+0404	& {\glyphfont{\large Є}} & {\salt\large Є} \\
U+0426	& {\glyphfont{\large Ц}} & {\salt\large Ц} \\
U+0491	& {\glyphfont{\large ґ}} & {\salt\large ґ} \\
U+A64C	& {\glyphfont{\large Ꙍ}} & {\salt\large Ꙍ} \\
U+047C	& {\glyphfont{\large Ѽ}} & {\salt\large Ѽ} \\
U+047E	& {\glyphfont{\large Ѿ}} & {\salt\large Ѿ} \\
U+047F	& {\glyphfont{\large ѿ}} & {\salt\large ѿ} \\
U+1F545	& {\glyphfont{\large 🕅 }}	& {\salt\large 🕅} & {\salta\large 🕅} & {\saltb\large 🕅} & {\saltc\large 🕅}  & {\saltd\large 🕅} & {\salte\large 🕅} & {\saltf\large 🕅} \\
U+0463 U+0486	& {\glyphfont{\large ѣ҆}} & {\salt\large ѣ҆}  \\
U+0463 U+0300	& {\glyphfont{\large ѣ̀}} & {\salt\large ѣ̀} & {\salta\large ѣ̀} \\
U+0463 U+0301	& {\glyphfont{\large ѣ́}} & {\salt\large ѣ́} & {\salta\large ѣ́} \\
U+0463 U+0311	& {\glyphfont{\large ѣ̑}} & {\salt\large ѣ̑} & {\salta\large ѣ̑} \\
U+0463 U+0486 U+0301	& {\glyphfont{\large ѣ҆́}} & {\salt\large ѣ҆́}  \\
U+A64B U+0486	& {\glyphfont{\large ꙋ҆}} & {\salt\large ꙋ҆}  \\
U+A64B U+0300	& {\glyphfont{\large ꙋ̀}} & {\salt\large ꙋ̀} & {\salta\large ꙋ̀} \\
U+A64B U+0301	& {\glyphfont{\large ꙋ́}} & {\salt\large ꙋ́} & {\salta\large ꙋ́} \\
U+A64B U+0311	& {\glyphfont{\large ꙋ̑}} & {\salt\large ꙋ̑} & {\salta\large ꙋ̑} & {\saltb\large ꙋ̑} \\
U+A64B U+0486 U+0301	& {\glyphfont{\large ꙋ҆́}} & {\salt\large ꙋ҆́}  \\
\hline
\end{tabular}
\end{table}

Додатно, три стилска скупа су била дефинисана у фонту.
Стилски скуп 1 (,,Акценти на десну страну``) позиционира акценте над Јатом
и Уком на десну страну и Стилски скуп 2 (,,Акценти на леву страну``)
позиционира акценте над Јатом и Уком на леву страну.
Ови стилски скупови су корисни када текст користи један од ових позиционирања свуда. Стилски скуп 10 (,,Варијанте са једнаком основном линијом``)
поставља велика слова на исту основну линију као и мала слова (корисно за рад са текстом у академском контексту где основна линија великих слова која је традиционално спуштена може да изазива проблеме са вертикалним простором када се обрађује текст који је било на латиници било на ћирилици). Ево примера:

\newfontfamily{\base}[StylisticSet=10]{Fedorovsk Unicode}

\begin{figure}[h]
\centering
\begin{tabular}{ll}
{\large \glyphfont Хрⷭ҇то́съ вᲂскр҃се и҆з̾ ме́ртвыхъ} & (регуларни текст) \\
{\large \base Хрⷭ҇то́съ вᲂскр҃се и҆з̾ ме́ртвыхъ} & (Стилски скуп 10 укључен) \\
\end{tabular}
\end{figure}

\subsection{Graphite Функције}

Стилске алтернативе Симбола за Марково поглавље, Слова Ге са дизалицом, па слова Је, Це, и Омега дуплиране су као Graphite функције у TTF верзији фонта, са именима ,,Symbol for Mark's Chapter``,
,,Ye``, ,,Tse``, ,,Ghe``, па ,,Omega`` односно.
За алтернативе за Симбол Марковог поглавља, вредности својстава су додељене да подударају \href{http://www.ponomar.net/files/pua_policy.pdf}
{Политици о алокацији Подручја приватне употребе} и другим фонтовима. Graphite функције су демонстриране у табели~\ref{fedorgraph}.

%\newfontfamily{\graph}[Renderer=Graphite]{Fedorovsk Unicode TT}
%\newfontfamily{\graphA}[Renderer=Graphite, RawFeature={Symbol for Mark's Chapter=Alternative 1}]{Fedorovsk Unicode TT}
%\newfontfamily{\graphB}[Renderer=Graphite, RawFeature={Symbol for Mark's Chapter=Alternative 2}]{Fedorovsk Unicode TT}
%\newfontfamily{\graphC}[Renderer=Graphite, RawFeature={Symbol for Mark's Chapter=Alternative 3}]{Fedorovsk Unicode TT}
%\newfontfamily{\graphD}[Renderer=Graphite, RawFeature={Symbol for Mark's Chapter=Alternative 4}]{Fedorovsk Unicode TT}
%\newfontfamily{\graphE}[Renderer=Graphite, RawFeature={Symbol for Mark's Chapter=Alternative 7}]{Fedorovsk Unicode TT}
%\newfontfamily{\graphF}[Renderer=Graphite, RawFeature={Symbol for Mark's Chapter=Alternative 8}]{Fedorovsk Unicode TT}
%\newfontfamily{\graphG}[Renderer=Graphite, RawFeature={Symbol for Mark's Chapter=Alternative 9}]{Fedorovsk Unicode TT}
\newfontfamily{\graphYe}[Renderer=Graphite, RawFeature={Ye=Alternative 1}]{Fedorovsk Unicode TT}
\newfontfamily{\graphTse}[Renderer=Graphite, RawFeature={Tse=Alternative 1}]{Fedorovsk Unicode TT}
\newfontfamily{\graphGhe}[Renderer=Graphite, RawFeature={Ghe=Alternative 1}]{Fedorovsk Unicode TT}
\newfontfamily{\graphOmega}[Renderer=Graphite, RawFeature={Omeg=Alternative 1}]{Fedorovsk Unicode TT}
\newfontfamily{\graphOt}[Renderer=Graphite, RawFeature={Ot=Alternative 1}]{Fedorovsk Unicode TT}

\begin{table}[htbp]
\centering
\caption{Алтернативе путем Graphite функција у Fedorovsk Unicode-у \label{fedorgraph}}
\begin{tabular}{lcccc}
\hline
	& Основна форма	& Алтернатива 1	& Алтернатива 2	& Алтернатива 3	\\
U+0404	& {\graph{\large Є }} & {\graphYe\large Є} \\
U+0426	& {\graph{\large Ц}} & {\graphTse\large Ц} \\
U+0491	& {\graph{\large ґ}} & {\graphGhe\large ґ} \\
U+A64C	& {\graph{\large Ꙍ}} & {\graphOmega\large Ꙍ} \\
U+047C	& {\graph{\large Ѽ}} & {\graphOmega\large Ѽ} \\
U+047E	& {\graph{\large Ѿ}} & {\graphOt\large Ѿ} \\
U+047F	& {\graph{\large ѿ}} & {\graphOt\large ѿ} \\
U+1F545	& {\graph{\large 🕅 }}	& {\graphA{\large 🕅}} & {\graphB{\large 🕅}} & {\graphC{\large 🕅}}  \\
	& 	& Alternative 4	& Alternative 7	& Alternative 8	  \\
	&	& {\graphD{\large 🕅}} & {\graphE{\large 🕅}} & {\graphF{\large 🕅}} \\
	&	& Alternative 9 \\
	&	& {\graphG{\large 🕅}} \\
\hline
\end{tabular}
\end{table}

\noindent Две додатне Graphite функције дефинисане су: ,,Accent Positions``,
са вредностима ,,left`` и ,,right``, која имитира понашање
стилских скупова 1 и 2; па ,,Equal Baseline`` (са вредношћу ,,yes``),
која имитира понашање стилског скупа 10.

\section{Menaion Unicode}

Фонт Menaion је намераван да се користи за рад са текстом
рукописа уставног доба. Садржава пуни репертоар потребних ћирилчиних и глагољских глифова као и глифове Византијске екфонетске нотације онакве се користе у ћирилчиним или глагољским рукописима.

Фонт је Menaion оригинално дизајниран од Виктора A. Баранова на
\href{http://www.manuscripts.ru/}{the Manuscript Project-у}. Поново је кодиран за Unicode од Александра Андрејева уз дозволу оригиналног аутора.

%\newfontfamily{\glyphfont}[HyphenChar="200B]{Menaion Unicode}

\subsection{Узроци текста}

Узроци текста у Menaion Unicode-у презентовани су у фигурама~\ref{men1}
и \ref{men2}. Молимо имајте на уму да комбинована глагољска слова
(Глагољска допуна) постала су расположива у Unicode-у 9.0. У старијим верзијама Microsoft софтвера, могуће је да исправно позиционирање глифова за ове карактере користећи OpenType неће бити могуће. Да остварите жељени учинак препоручујемо да користите LibreOffice, \XeTeX{}, \LuaTeX{}, или софтвере напредног стоног издаваштва као што је Adobe InDesign.

\begin{figure}[htbp]
\centering
\caption{ћирилчини текст из Остромирских јеванђеља (11. век) \label{men1}}
\begin{tabular}{lr}
 1& {\Large \glyphfont    Искони бѣ слово } \\
 2& {\Large \glyphfont    и слово бѣ отъ  } \\
 3& {\Large \glyphfont   б҃а и б҃ъ бѣ} \\ 
 4& {\Large \glyphfont    слово  𝀏̃  се бѣ} \\ 
 5& {\Large \glyphfont    искони оу} \\ 
 6& {\Large \glyphfont    б҃а  ⁘  и тѣмь в̇са бꙑ} \\ 
 7& {\Large \glyphfont    шѧ  𝀏̃  и беꙁ него ни} \\ 
 8& {\Large \glyphfont    чьтоже не бꙑсть  ·} \\ 
 9& {\Large \glyphfont   ѥже бꙑсть  𝀏̃  въ то} \\ 
10& {\Large \glyphfont    мь животъ бѣ  ·  и} \\ 
 1& {\Large \glyphfont    животъ бѣ свѣтъ} \\ 
 2& {\Large \glyphfont    чловѣкомъ  𝀏̃  и свѣ} \\ 
 3& {\Large \glyphfont    тъ въ тьмѣ свьти} \\ 
 4& {\Large \glyphfont    тьсѧ  ·  и тьма ѥго} \\ 
 5& {\Large \glyphfont    не обѧтъ  𝀏̃  бꙑсть} \\ 
 6& {\Large \glyphfont    члв҃къ посъланъ} \\ 
 7& {\Large \glyphfont    отъ б҃а  ·  имѧ ѥмоу} \\ 
 8& {\Large \glyphfont    иоанъ  𝀏̃  тъ приде} \\ 
 9& {\Large \glyphfont    въ съвѣдѣтель} \\ 
10& {\Large \glyphfont    ство  ·  да съвѣдѣте} \\ 
2.2  1& {\Large \glyphfont    льствоуѥть о свѣ} \\ 
 2& {\Large \glyphfont    тѣ  𝀏̃  да вьси вѣрѫ} \\ 
 3& {\Large \glyphfont    имѫть имь  ⁘  не бѣ} \\ 
 4& {\Large \glyphfont    тъ свѣтъ  ⁘  нъ да} \\ 
 5& {\Large \glyphfont    съвѣдѣтельствоу} \\ 
 6& {\Large \glyphfont    ѥть о свѣтѣ  𝀏̃̑ бѣ} \\ 
 7& {\Large \glyphfont    свѣтъ истиньнꙑ} \\ 
 8& {\Large \glyphfont    и  ·  иже просвѣщаѥ} \\ 
 9& {\Large \glyphfont    ть в́сꙗкого чл҃ка  ⸴} \\ 
10& {\Large \glyphfont   грѧдѫща въ миръ  𝀏̃̑} \\ 
\end{tabular}
\end{figure}

\begin{figure}[htbp]
\centering
\caption{Глагољски текст из Асеманијевог јеванђеља (11. век) \label{men2}}
\begin{tabular}{lr}
1 & {\Large \glyphfont   ⁘ ⰅⰂⰀ𞀌҇   ⰙⰕ҇   ⰋⰉ҇Ⱁ } \\
 2 & {\Large \glyphfont  Ⰻⱄⰽⱁⱀⰹ ⰱⱑ } \\
 3 & {\Large \glyphfont       ⱄⰾⱁⰲⱁ  · } \\
 4 & {\Large \glyphfont      ⰻ ⱄⰾⱁⰲⱁ } \\
 5 & {\Large \glyphfont       ⰱⱑ ⱋ̔ ⰱⰰ  · } \\
 6 & {\Large \glyphfont      ⰻ ⰱ͞ⱏ ⰱⱑ } \\
 7 & {\Large \glyphfont      ⱄⰾⱁⰲⱁ  · } \\
 8 & {\Large \glyphfont   Ⱄⰵ ⰱⱑ ⰻ̔ⱄⰽⱁ} \\
 9 & {\Large \glyphfont     ⱀⰻ  ·  ⱋ̔ ⰱ꙯ⰰ  ·  ⰲⱐ} \\
10 & {\Large \glyphfont     ⱄⱑ ⱅⱑⰿⱏ ⰱⱏⰻ} \\
11 & {\Large \glyphfont     ⱎⱔ  ·  Ⰻ̔ ⰱⰵⰶ ⱀⰵⰳⱁ } \\
12 & {\Large \glyphfont     ⱀⰹⱍⰵⱄⱁⰶⰵ } \\
13 & {\Large \glyphfont     ⱀⰵ ⰱⱏⰻⱄⱅⱏ  ·  ⰵ̔} \\
14 & {\Large \glyphfont     ⰶⰵ ⰱⱏⱄⱅⱏ  · } \\
15 & {\Large \glyphfont    Ⰲⱏ ⱅⱁⰿⱏ ⰶⰹⰲⱁ} \\
16 & {\Large \glyphfont     ⱅⱏ ⰱⱑ  ·  ⰻ ⰶⰹⰲⱁ} \\
17 & {\Large \glyphfont     ⱅⱏ ⰱⱑ ⱄⰲⱑⱅⱏ } \\
18 & {\Large \glyphfont     ⱍⰾ҃ⰽⰿⱏ  ·  ⰻ̔ ⱄⰲⱁⱑ } \\
19 & {\Large \glyphfont     ⰲⱏ ⱅⱐⰿⱑ ⱄⰲⱏ} \\
20 & {\Large \glyphfont     ⱅⰹⱅⱏ ⱄⱔ  ·  ⰻ ⱅⱐ} \\
21 & {\Large \glyphfont     ⰿⰰ ⰵ̔ⰳⱁ ⱀⰵ ⱁ̔ⰱⱔⱅ } \\
\end{tabular}
\end{figure}

\subsection{Обезбеђене лигатуре}

Фонт обезбеђује неколико лигатура, које су направљене уводећи
Ознаку форматирања без ширине без прелома (U+200D) између два карактера. Листа лигатура је наведена у табели~\ref{menligs}. Лигатуре могу да буду обрађене користећи било OpenType било
SIL Graphite.

%\newfontfamily{\graph}[Renderer=Graphite]{Menaion Unicode TT}

\begin{table}[htbp]
\centering
\caption{Лигатуре расположиве у фонту Menaion Unicode \label{menligs}}
\begin{tabular}{lcc}
Име	& Секвенца	& Изглед \\
\hline
Мала лигатура И-Је &	U+0438 U+200D U+0435 	& {\glyphfont{\large и‍е }} \\
Мал лигатура Ен-и	&	U+043d U+200D U+0438 	& {\graph{\large н‍и }} \\
Мала лигатура Ен-Мали јус	& U+043d U+200D U+0467 	& {\glyphfont{\large н‍ѧ }} \\
Мала лигатура Ес-Ве	&	U+0441 U+200D U+0432 	& {\glyphfont{\large с‍в }} \\
Мала лигатура Те-Ер	&	U+0442 U+200D U+0440 	& {\glyphfont{\large т‍р }} \\
Велика лигатура А-У	& 	U+0410 U+200D U+0423 	& {\glyphfont{\large А‍У }} \\
Мала лигатура А-У	&	U+0430 U+200D U+0443 	& {\glyphfont{\large а‍у }} \\
Мала лигатура А-Те		&	U+0430 U+200D U+0442 	& {\glyphfont{\large а‍т }} \\
Велика лигатура И-Је	&	U+0418 U+200D U+0415 	& {\glyphfont{\large И‍Е }} \\
Велика лигатура Ел-Ге		&	U+041b U+200D U+0413 	& {\glyphfont{\large Л‍Г }} \\
Мал лигатура Ел-Ге		&	U+043b U+200D U+0433 	& {\glyphfont{\large л‍г }} \\
Велика лигатура Ен-И	&	U+041d U+200D U+0418 	& {\glyphfont{\large Н‍И }} \\
Велика лигатура Ен-Мали јус	&	U+041d U+200D U+0466 	& {\graph{\large Н‍Ѧ }} \\
Велика лигатура Ес-Ве		&	U+0421 U+200D U+0412 	& {\glyphfont{\large С‍В }} \\
Мала лигатура Те-Јат		&	U+0442 U+200D U+0463 	& {\glyphfont{\large т‍ѣ }} \\
Велика лигатура Те-Ве	&	U+0422 U+200D U+0412	& {\glyphfont{\large Т‍В }} \\
Мала лигатура Те-Ве		&	U+0442 U+200D U+0432	& {\glyphfont{\large т‍в }} \\
Велика лигатура Те-И		&	U+0422 U+200D U+0418 	& {\glyphfont{\large Т‍И }} \\
Мала лигатура Те-И		&	U+0442 U+200D U+0438 	& {\glyphfont{\large т‍и }} \\
Велика лигатура Те-Ер		&	U+0422 U+200D U+0420 	& {\glyphfont{\large Т‍Р }} \\
Лигатура Велико А-Мало Те	&	U+0410 U+200D U+0442 	& {\glyphfont{\large А‍т }} \\
Велика лигатура Те-Меки знак	&	U+0422 U+200D U+042c 	& {\glyphfont{\large Т‍Ь }} \\
Мала лигатура Те-Мали знак	&	U+0442 U+200D U+044c 	& ‍{\graph{\large т‍ь }} \\
Мала лигатура Те-А		&	U+0442 U+200D U+0430 	& {\glyphfont{\large т‍а }} \\
\hline
\end{tabular}
\end{table}

\section{Pomorsky Unicode}

Фонт Pomorsky Unicode је верна (идеализована)
репродукција декоративног калиграфског стила књига и наслова поглавља, који је највероватније развијен у 1700-им од писара старообредничког
Виговског манастира. 
Ово се често види у песничким рукописима, литургијским рукописима,
хагиографским и полемичким делима Поморских и Федосејевских заједница, па је традиционалан и ,,органски`` стил утиснутог натписа којем недостаје икакав очигледан утицај из западноевропске и латинске тупографије. Фонт Pomorsky је оригинално дезинирајн од Никите Симонза.
Уређен је и поново кодиран за Unicode од Александра Андрејева.
Намењен је за употребу са \emph{иницијалима} и декоративним насловима.

Неколико верзија многих глифова обезбеђује се у фонту.
Украшене форме слова су подразумеване и обезбеђују се на точкама кодовима ћирилчиних великих слова; оне би требало да се користе колико год је могуће.
Лакше форме могу да се користе кад год словима треба мање украшен изглед, или када би диајакрити могли да се сукобљају са украшавањем (или када се украшање једног карактера сукобља са украшавањем другог); ове лакше форме су расположиве као
\verb+Stylistic Set 1+ или као Graphite функција ,,Use simple forms``
(\verb+smpl+). Има неколико додатних карактера који су стилске варијанте, што се обезбеђује као стилске алтернативе (\verb+salt+)
или као Graphite функција ,,Alternates`` (\verb+salt+).
Пошто је фонт намењен за иницијале и наслове, карактери малих слова нису расположиви.

%\newfontfamily{\glyphfont}{Pomorsky Unicode}
\newfontfamily{\simple}[StylisticSet=1]{Pomorsky Unicode}
%\newfontfamily{\salt}[Alternate=0]{Pomorsky Unicode}
%\newfontfamily{\salta}[Alternate=1]{Pomorsky Unicode}

Основна форма, ,,лака`` форма, па сваке стилске алтернативе једног карактера демонстриране су у табели~\ref{pomor}.

\begin{table}[htbp]
\centering
\caption{Облици карактера обезбеђени од стране Pomorsky Unicode-а \label{pomor}}
{\fontsize{38pt}{1.5em}
\begin{tabular}{cccc}
	{\glyphfont А}{\simple А}{\salt А}	& {\glyphfont Б}{\simple Б} & {\glyphfont В}{\simple В} & {\glyphfont Г}{\simple Г} \\

	{\glyphfont Е}{\simple Е}	& {\glyphfont Ж}{\simple Ж} & {\glyphfont Ѕ}{\simple Ѕ} & {\glyphfont З}{\simple З} \\
	
	{\glyphfont И}{\simple И}	& {\glyphfont Й}{\simple Й} & {\glyphfont І}{\simple І} & {\glyphfont Ї}{\simple Ї} \\

	{\glyphfont К}{\simple К}{\salt К}{\salta К}	& {\glyphfont Л}{\simple Л} & {\glyphfont М}{\simple М} & {\glyphfont Н}{\simple Н} \\

	{\glyphfont О}{\simple О}	& {\glyphfont Ѻ}{\simple Ѻ} & {\glyphfont П}{\simple П} & {\glyphfont Р}{\simple Р}{\salt Р}{\salta Р} \\

	{\glyphfont С}{\simple С}	& {\glyphfont Т}{\simple Т} & {\glyphfont ОУ}{\simple ОУ} & {\glyphfont Ꙋ}{\simple Ꙋ} \\

	{\glyphfont Ф}{\simple Ф}	& {\glyphfont Х}{\simple Х} & {\glyphfont Ѡ}{\simple Ѡ} & {\glyphfont Ѽ}{\simple Ѽ} \\

	{\glyphfont Ѿ}{\simple Ѿ}	& {\glyphfont Ц}{\simple Ц} & {\glyphfont Ч}{\simple Ч} & {\glyphfont Ш}{\simple Ш} \\

	{\glyphfont Щ}{\simple Щ}	& {\glyphfont Ъ}{\simple Ъ} & {\glyphfont Ы}{\simple Ы} & {\glyphfont Ь}{\simple Ь} \\

	{\glyphfont Ѣ}{\simple Ѣ}	& {\glyphfont Ю}{\simple Ю} & {\glyphfont Ꙗ}{\simple Ꙗ}{\salt Ꙗ} & {\glyphfont Ѧ}{\simple Ѧ} \\

	{\glyphfont Ѯ}{\simple Ѯ}	& {\glyphfont Ѱ}{\simple Ѱ} & {\glyphfont Ѳ}{\simple Ѳ} & {\glyphfont Ѵ}{\simple Ѵ} \\
\end{tabular}
}
\end{table}

\subsection{Узорци текста}

\begin{center}
\begin{tabular}{c}
{\fontsize{48pt}{2em} \glyphfont ЧИ́НЪ ВЕЧЕ́РНИ.} \\
{\fontsize{48pt}{2em} \simple ЧИ́НЪ ВЕЧЕ́РНИ.} \\
{\fontsize{48pt}{2em} \glyphfont СѶНѠ́ДИКЪ.} \\
{\fontsize{48pt}{2em} \simple СѶНѠ́ДИКЪ.} \\
\end{tabular}
\end{center}

\section{Monomakh Unicode}

Monomakh Unicode је базиран на фонту Monomachus дизајнираном од Алексеја Крјукова. Модификован је уз дозволу.
Monomakh Unicode је ћирилични фонт имплементиран у мешовитом уставном / полууставном стилу и намењен за покривање потреба истраживања која се баве са словенском историјом и филологијом. Укључује све историјске ћирилчине карактере који су тренутно дефинисани у Unicode-у; фонт укључује и скуп латинских слова дизајнираних да буду стилско компатибилни са ћирилчиним делом. Ово може да буде корисно да се слажу двојезична издања на црквенословенском и језицима написаним на латиннском писму, посебно они који користе многе дијакритичне ознаке, као што у доњем румунском примеру.

%\newfontfamily{\glyphfont}
%      [Renderer=Graphite, RawFeature=
%          {Localized Forms for Romanian=Required Localized Forms}]
%{Monomakh Unicode TT}

\subsection{Узорак двојезичног текста}

\begin{tabular}{p{2.25in}p{0.02in}p{2.25in}}
\begin{churchslavonic}
{\glyphfont Бл҃же́нъ мꙋ́жъ, и҆́же не и҆́де на совѣ́тъ нечести́выхъ, и҆ на пꙋтѝ грѣ́шныхъ не ста̀, и҆ на сѣда́лищи гꙋби́телей не сѣ́де: но въ зако́нѣ гдⷭ҇ни во́лѧ є҆гѡ̀, и҆ въ зако́нѣ є҆гѡ̀ поꙋчи́тсѧ де́нь и҆ но́щь. И҆ бꙋ́детъ ꙗ҆́кѡ дре́во насажде́ное при и҆схо́дищихъ во́дъ, є҆́же пло́дъ сво́й да́стъ во вре́мѧ своѐ.}
\end{churchslavonic}
& &
\begin{romanian}
{\glyphfont Fericit bărbatul, care n-a umblat în sfatul necredincioșilor și în calea păcătoșilor nu a stat și pe scaunul hulitorilor n-a șezut; ci în legea Domnului e voia lui și la legea Lui va cugeta ziua și noaptea. și va fi ca un pom răsădit lângă izvoarele apelor, care rodul său va da la vremea sa.}
\end{romanian}
\end{tabular}

\subsection{OpenType и SIL Graphite функције}

Monomakh фонт пружа неколико опционалних OpenType функција које могу да буду укључене или искључене од корисника, скупа са аналогним функцијама у SIL Graphite-у. То су:

\begin{itemize}
\item Стилски скуп 1 (\emph{ss01}) обезбеђен је као привремено решење \href{https://bugs.documentfoundation.org/show_bug.cgi?id=85731}
{LibreOffice Bug 85731}-а, који Вам не омогућава да спецификујете карактер растављање на слогове у LibreOffice-у. Када је укључено, замењује сва појављивања U+002D-а Цртица-минус и U+2010-а Цртица
U+005F-ом Ниска линија (подвлака) за употребу као карактер растављања на слогове. Graphite аналог је обезбеђен од стране функције 
,,Use underscore for hyphenation`` (\emph{hyph}),
која узима вредности |0| (no) и |1| (yes). Молимо имајте на уму да ће функције бити спуштена чим потребна фунционалност буде додата LibreOffice-у.
\item Стилски скуп 6 (\emph{ss06}) приказује U+0456 Ћирилчино мало слово украјинско / белоруско И са једном горњом тачком и Стилски скуп 7 (\emph{ss07}) приказује исти карактер са горње две тачке. Као подразумевано, U+0456 је приказан без тачки. Graphite аналог је обезбеђен од стране функције ,,Cyrillic Decimal I`` (\emph{deci}), која узима вредности 0 (подразумевану), 1 или 2, за број тачки на U+0456-у.
\item Стилски скуп 8 (\emph{ss08}) приказује карактере U+0417 Ћирилчино велико слово Зе и U+0437 Ћирилчино мало слово Зе као ,,оштру земљу``, т.ј., као карактере U+A640 Ћирилчино велико слово Земља и U+A641 Ћирилчино мало слово Земља, односно. На Graphite-у је иста функционалност обезбеђена од стране функције ,,Cyrillic Zemlya`` (\emph{zeml}), која узима вредности \emph{Round} (0) и \emph{Sharp} (1). Уопште, ова промена би требало да се прави на нивоу тачке кода, онда се коришћење ове функције не препоручује.
\item Стилски скуп 9 (\emph{ss09}) приказује карактере U+0427-а Ћирилчино велико слово Че и U+0447-а Ћирилчино мало слово Че у њиховој архаичној форми, са доњим продужетком слова на средини (нпр., {\fontspec{Monomakh Unicode}[StylisticSet=9] ч} уместо {\glyphfont ч}-а). Иста функционалност је обезбеђена од стране Graphite функције ,,Cyrillic Cherv`` (\emph{chrv}), која узима вредности \emph{Modern (Single Sided)} (0) и \emph{Old (Double Sided)} (1).
\item Стилски скуп 10 (\emph{ss10}) приказује карактере U+0429 Ћирилчино велико слово Шча и U+0449 Ћирилчино мало слово Шча у њиховој модерној форми, са доњим пордужетком слова на десној (нпр., {\fontspec{Monomakh Unicode}[StylisticSet=10] щ} instead of {\glyphfont щ}). Иста функционалност је обезбеђена од стране Graphite функције ,,Cyrillic Shcha`` (\emph{shch}), која узима вредности \emph{Modern (Descender Right)} (0) и \emph{Old (Descender Centered)} (1).
\item Стилски скуп 11 (\emph{ss11}) приказује карактере U+044B Ћирилчино мало слово Јеру и U+A651 Ћирилчино смало слово Јеру са Тврдим јером са два глифа повезана (нпр., {\fontspec{Monomakh Unicode}[StylisticSet=11] ы} уместо {\glyphfont ы}-а). Иста функционалност је обезбеђена од стране Graphite функције ,,Cyrillic Yery`` (\emph{yery}), која прима вредности \emph{Without a Connecting Line} (0) и \emph{With a Connecting Line} (1).
\item Стилски скуп 13 (\emph{ss13}) приказује карактер U+0463 Ћирилчино мало слово Јат са левим стаблом продуженим до основне линије (нпр., као {\fontspec{Monomakh Unicode}[StylisticSet=13] ѣ}). Иста функционалност је обезбеђена од стране Graphite функције ,,Cyrillic Yat`` (\emph{cyat}), која узима вредности \emph{With a Back Beak} (0) и \emph{With an Additional Vertical Stem} (1). Молимо имајте на уму да то није исто као и U+A653 Ћирилчино мало слово јотован Јат.
\item Иста функционалност ових стилских скупа је обезбеђена у OpenType-у и од стране функције Стилске алтернативе (\emph{salt}).
\item Претходне верзије фонта су обезбеђивале Стилски скуп 1 (\emph{ss01}), дуплиран као Graphite функција ,,Localized Forms for Romanian`` (\emph{rold}), која је приказивала U+015E Латинско велико слово S са седиљом, U+0162 Латинско велико слово T са седиљом, па њихове мале еквивалентне, као U+0218 Латинско велико слово S са доњом запетом, U+021A Латинско велико слово T са доњом запетом, па њихове мале еквивалентне. Међутим, пошто коришћење U+015E-а, U+0162-а и њихових малих еквивалената за кодирање румунског текста сматрана је погрешном, ова функција је спуштена. Снажно подстичемо кориснике да претварају свој текст на нивоу тачки кодова да користе исправне карактере за румунску ортографију. Међутим, ради компатибилности са текстом који је погрешно кодиран, та функвија је још увек расположива.
\item Стилски скуп 15 (\emph{ss15}), који обезбеђује комбинована ћирилчина слова са аутоматским \emph{покритијем} где то санкционише Синодална ортографија такође је спуштен и могуће је да ће бити уклоњен. Корисници би требало да кодирају експлицитно \emph{покритије} као U+0487 Комбиновано ћирилчино Покритије. Погледајте \href{http://www.unicode.org/notes/tn41/}
{UTN 41: Church Slavonic Typography in Unicode} за више информација.
\end{itemize}

Две додатне функције су расположиве у SIL Graphite-у самом:
\begin{itemize}
\item Graphite функција ,,Convert Arabic Digits to Church Slavonic`` (\emph{cnum}), када је укључена, приказиваће аутоматски западне цифре (,,арапске бројеве``) као ћириличне бројеве. То је помоћно, на пример, за нумерисање страна у софтверу који не подржава ћиричне бројеве.
\item Graphite функција ,,Convert HIP-6B Keystrokes to Church Slavonic Characters`` (\emph{hipb}), када је укључена, приказиваће текст кодиран у застарелој HIP кодној страници као црквенословенски. Коришћење ове функције се не фаворизује и корисницима се уместо тога препоручује да претварају текст кодиран по HIP-у на Unicode.
\end{itemize}

\section{Indiction Unicode}

Indiction Unicode фонт репродукује декоративни стил иницијала који користи синодална црквенословенска издања од касних 1800-их.

Оригинални Indyction фонт је развијен од Владислава В. Дороша и дистрибуисан
под именом Indyction UCS као део CSLTeX-а, који је лицениран под Јавном лиценцом \LaTeX{} пројекта.
Фонт је поново кодиран за Unicode и уређен од Александра Андрејева, па је сада
дистрибуисан као Indiction Unicode под SIL Јавном лиценцом фонтова.
Намењен је за коришћење са
\emph{иницијалима} у модерним црквенословенским издањима.
Облици карактера су демонстрирани у табели~\ref{indict}.

\begin{table}[htbp]
\centering
\caption{Облици карактера обезбеђени од стране Indiction Unicode-а \label{indict}}
{\fontsize{38pt}{1.5em}
\begin{tabular}{cccc}
	{\ind А}	& {\ind Б} & {\ind В} & {\ind Г} \\

	{\ind Е}	& {\ind Ж} & {\ind Ѕ} & {\ind З} \\
	
	{\ind И}	&  {\ind І} & {\ind К}	& {\ind Л} \\

	{\ind М} & {\ind Н} & 	{\ind О} & {\ind Ѻ} \\

	{\ind П} & {\ind Р} & {\ind С}	& {\ind Т} \\

	{\ind Ꙋ} & {\ind Ф} & {\ind Х} & {\ind Ѡ} \\

	{\ind Ѽ} & {\ind Ꙍ} & {\ind Ѿ}	& {\ind Ц} \\

	{\ind Ч} & {\ind Ш} & {\ind Щ} & {\ind Ъ} \\

	{\ind Ы} & {\ind Ь} & {\ind Ѣ} & {\ind Ю} \\

	{\ind Ꙗ} & {\ind Ѧ} & {\ind Ѯ} & {\ind Ѱ} \\

	{\ind Ѳ} & {\ind Ѵ} & {\ind Ѷ} \\
\end{tabular}
}
\end{table}

\subsection{Узорци текста}
\vspace{-1em}
\begin{churchslavonic}
\cuLettrine Бл҃же́нъ мꙋ́жъ, и҆́же не и҆́де на совѣ́тъ нечести́выхъ, и҆ на пꙋтѝ грѣ́шныхъ не ста̀, и҆ на сѣда́лищи гꙋби́телей не сѣ́де: но въ зако́нѣ гдⷭ҇ни во́лѧ є҆гѡ̀, и҆ въ зако́нѣ є҆гѡ̀ поꙋчи́тсѧ де́нь и҆ но́щь. И҆ бꙋ́детъ ꙗ҆́кѡ дре́во насажде́ное при и҆схо́дищихъ во́дъ, є҆́же пло́дъ сво́й да́стъ во вре́мѧ своѐ, и҆ ли́стъ є҆гѡ̀ не ѿпаде́тъ: и҆ всѧ̑, є҆ли̑ка а҆́ще твори́тъ, ᲂу҆спѣ́етъ.
\par
\end{churchslavonic}

\section{Познати проблеми}

Ево неких познатих проблема:

\begin{itemize}

\item Подешавање међусловног размака није расположиво за латинске карактере ни у једном од фонтова.
Пошто се не очекује да ће ти фонтови бити интезивно коришћени да слажу латински текст, решавање овог проблема нема високог приоритета.

\item Ponomar Unicode има подешавање међусловног размака базираном на Graphite-у почевши од верзије 2.0, но је дефектно. Поготово,
уводити дијакритичку ознаку прекидаће подешавање међусловног размака.
То ће бити поправљено у верзији 2.1.

\item Подешавање међусловног размака није расположиво у Graphite верзији Pomorsky Unicode-а.

\end{itemize}

\noindent Могуће је да има других проблема, али пре него што да пријавите такве проблеме,
молимо проверите да њихов софтвер прописно подржава OpenType и / или
SIL Graphite. Сугеришемо да проверите очекивано понашање у \XeTeX{}-у
или у \LuaTeX{}-у.

\section{Признања}

Аутори би желели да захвале следеће људе:

\begin{itemize}

\item Владислав Дорош, који је дозволио да се његов фонт
\href{http://irmologion.ru/fonts.html}{Hirmos} буде поново кодира на Unicode и модификује, што је довело до креирања фонта Ponomar.

\item Виктор Баранов \href{http://www.manuscripts.ru/}{Manuscripts}
пројекта, који је дозволио поновно кодирање и модификацију свог фонта Menaion.

\item Михаел Иванович због његове помоћи у дизајнирању карактера за Саха
(јакутски), делимично преузетих из његовог фонта Sakha UCS.

\item Алексеј Крјуков, који је одговорио на разна питања о FontForge-у,
дозволио да се његов фонт Monomachus модификује и препакира,
па чија обимна документација за фонт 
\href{https://github.com/akryukov/oldstand/}{Old Standard} консултована је и делимично делимично поново употребљена.

\item Мајк Крутиков, који је саставио \TeX{} пакет фонтова.

\item aleslavista, for translating the documentation into Serbo-Croatian.
\end{itemize}

\end{document}
