\documentclass[11pt]{ltxdoc}
\usepackage[usenames,dvipsnames,svgnames,table]{xcolor}
\usepackage{fontspec}
\usepackage{xltxtra,comment}
% code borrowed from Polyglossia documentation -- Thanks!
\definecolor{myblue}{rgb}{0.02,0.04,0.48}
\definecolor{lightblue}{rgb}{0.61,.8,.8}
\definecolor{myred}{rgb}{0.65,0.04,0.07}
\usepackage[
    bookmarks=true,
    colorlinks=true,
    linkcolor=myblue,
    urlcolor=myblue,
    citecolor=myblue,
    hyperindex=false,
    hyperfootnotes=false,
    pdftitle={Church Slavonic fonts},
    pdfauthor={Aleksandr Andreev},
    pdfkeywords={Church Slavic, Church Slavonic, Old Church Slavonic, Old Slavonic, fonts, Unicode}
    ]{hyperref}
\usepackage{polyglossia}
\setmainlanguage{serbian}
\setotherlanguages{english,russian,churchslavonic,romanian}
\usepackage{churchslavonic}
\usepackage{lettrine}

%% DOCUMENTATION VERSION AND RELEASE DATES
\def\filedate{2. јуна 2019.}
\def\fileversion{верзија 2.1}

%% fontspec declarations:
\setmainfont[Ligatures = TeX]{Libertinus Serif Regular}
\setsansfont{DejaVu Sans}
\setmonofont[Scale=MatchLowercase]{DejaVu Sans Mono}
\newfontfamily\churchslavonicfont[Script=Cyrillic,Ligatures=TeX,Scale=1.33333333,HyphenChar="005F]{PonomarUnicode.otf} 
\newfontfamily{\slv}[Scale=MatchLowercase]{Ponomar Unicode}
\newfontfamily{\ust}[Scale=MatchLowercase]{Menaion Unicode}
\newfontfamily{\ind}[Scale=1.333333333]{Indiction Unicode}
\newfontfamily{\vertograd}[Scale=1.333333]{Vertograd Unicode}
\newfontfamily{\firaslav}[Scale=MatchLowercase]{FiraSlav-Regular}

\linespread{1.05}
%\lineskip=0pt
\lineskiplimit=0em
\frenchspacing
\EnableCrossrefs
\CodelineIndex
\RecordChanges
% COMMENT THE NEXT LINE TO INCLUDE THE CODE
\AtBeginDocument{\OnlyDescription}

\makeatletter
\def\ps@cuNum%
\let\@evenfoot\@oddfoot
}%
\def\cu@lettrine{\lettrine[lines=3,findent=0pt,nindent=0pt]}
\def\cuLettrine{\cu@tokenizeletter\cu@lettrine}
\renewcommand{\LettrineFontHook}{\ind \cuKinovarColor}
\makeatother
\begin{document}

\title{Црквенословенски фонтови}
\author{Александр Андрејев\thanks{Коментаре се могу упутити \href{mailto:aleslavista@outlook.it}{Алесандроу Ческиниу}.} \and Никита Симонс}
\date{\filedate \qquad \fileversion\\
\footnotesize (\textsc{pdf} датотека генерисана \today)}

\maketitle
\tableofcontents

\section{Увод}

Црквенословенски је (такође познат као Старословенски; ISO 639-2 код |cu|) књижевни језик који користе Словенски народи; тренутно га користе као литургијски језик Руска православна црква, друге месне православне цркве, као и разни Католици византијског обреда и заједнице Старообредница. \texttt{fonts-churchslavonic} пакет обезбеђује фонтове ради приказивања црквенословенског текста.

Фонтови су креирани за обрађивање Unicode текста кодираног по UTF-8-у. Текст кодиран по застарелим кодним страницама (попут HIP-а и UCS-а) може бити претворен у Unicode користећи одвојену пакету услужних програма. Погледајте \href{https://sci.ponomar.net/}{Slavonic Computing Initiative website} за више информација.

\subsection{Лиценца}

Фонтови дистрибуисани у том пакету лиценцирају се под Отвореном лиценцом фонта SIL (верзија 1.1 или каснија).

Као слободни софтвер, ти фонтови се дистрибуишу у нади да ће бити корисни, али БЕЗ ИКАКВЕ ГАРАНЦИЈЕ; чак и без имплицитне гаранције ТРЖИШНОСТИ или ПОГОДНОСТИ ЗА ОДРЕЂЕНУ СВРХУ. Погледајте \href{https://scripts.sil.org/ofl}{SIL Open Font License} за више детаља.

Овај документ се лиценцира под Међународном лиценцом Кријејтив Комонс Ауторство-Делити под истим условима 4.0. Да видите копију те лиценце, посетите \href{http://creativecommons.org/licenses/by-sa/4.0/}{CreativeCommons website}.

\subsection{Опис}

Пакет обезбеђује неколико фонтова који су намењени за обрађивање црквенословенског текста разних рецензија и других текстова повезаних са црквенословенским језиком: модерни црквенословенски текст (,,Синодални црквеноловенски``), историјски штампани црквенословенски текст и рукописни уставски црквенословенски текст (било на ћирилици било на глагољици) као и текст на Саха (јакутски језик), алетском језику (наречје Лисичијих острва), па на румунској (молдавској) ћирилици, сви написани на црквеном писму. Опсег разних фонтова се слаже са смерницама за опсег фонта наведеним у \href{http://www.unicode.org/notes/tn41/}
{Unicode Technical Note \#41: Church Slavonic Typography in Unicode}. Уопштено говорећи, он укључује највећи део карактера (али не сви) у Unicode блоковима, Ћирилица,  Допуна ћирилице, Продужена ћирилица-A, Продужена ћирилица-B, Продужена ћирилица-C (до Unicode-а 9.0), Глагољица, и Допуна глагољице. Међутим, карактери који се не користе на црквенословенском језику нису укључени (осим неких карактера који се користе у модерном руском, украјинском, белоруском, српском и македонском језику за сврхе компатибилности са неким апликацијама).

\subsection{Инсталација и употреба}

Ако читате овај документ, онда сте вероватно већ преузели пакет фонтова. Можете проверите да ли имате најновију верзију посећујући \href{http://sci.ponomar.net/}{Slavonic Computing Initiative website}.

\subsubsection{Формати фонтова}

Сви фонтови су тренутно доступни у једином формату:

\begin{description}
\item[\XeTeXpicfile "opentype.png" width 4mm] \hyperlink{OT}{OpenType} фонтови са
PostScript контурама (такође звани OpenType-CFF фонтови).

\item[\XeTeXpicfile "deprecated.png" width 4mm] TrueType фонтови су сада напуштени, па се више не обезбеђују. Ако Вам требају TrueType фонтови погледајте
\href{https://github.com/slavonic/fonts-cu-legacy/}{Legacy Fonts package}.
\end{description}

\noindent Ваљда ће Вам требати застарели TrueType фонтови у следећим ситуацијама:

\begin{itemize}

% The information about versions before OpenOffice.org 3.2 is outdated
% and no longer viewed as relevant
%\item In older versions of OpenOffice.org, OpenType-CFF fonts 
%were not properly embedded into PDF files. Moreover, under Unix-based
%systems, OpenOffice.org could not access such fonts at all, so using TTF
%versions was the only option. This was fixed in OpenOffice.org 3.2 and LibreOffice.

\item OpenOffice.org и старије верзије LibreOffice-а захтевају употребу SIL Graphite-а,
која је доступна само у TTF верзији. То ограничење је поправљено почевши од
LibreOffice-а 5.3, који сада има
\href{https://wiki.documentfoundation.org/ReleaseNotes/5.3}{пуну подршку OpenType-а}.

\item OpenType-CFF фонтови нису били добро подржани у Јави пре Оракл Јаве SE 7.

\item У Мајкрософт производима, позиционирање OpenType глифова није подржано за глифове у Подручју приватне употребе или за карактере изван Unicode 7.0 спектра. Требало би да користите LibreOffice ако Вам треба позиционирање комбинованих глагољских карактера.

\end{itemize}

\subsection{Изворни пакети}

Можете такође преузети FontForge изворе за све фонтове са \href{https://github.com/typiconman/fonts-cu/}{GitHub repository-а}. То је корисно само ако намеравате да уредите фонтове у \href{http://fontforge.sourceforge.net}{FontForge} уређивачу фонта. Уопште, нећете добити никакво побољшање продуктивности из поновног склапања датотека фонтова, онда поновно склапање из извора није препоручљиво, осим ако имате стварну потребу за модификовањем фонтова, на пример, да бисте додали сопствене додатне глифове Подручју приватне употребе.

\subsection{Системски захтеви}

Сви ти фонтови су велики Unicode фонтови и захтевају оперативни систем и софтверско окружење усклађено са Unicode-ом. Изван окружења усклађеног са Unicode-ом, само ћете бити у стању, највише, да приступите првим 256 глифова фонта.

\subsubsection{Мајкрософт Виндоус}

Фонтови кодирани по Unicode-у подржани су почевши од Видноуса 2000. Требаће Вам програм за обраду текста који се може бавити документима заснованим на Unicode-у, попут Мајкрософт Ворда 97 и каснијих верзија, или \href{http://www.libreoffice.org}{LibreOffice}! Ако користите \TeX{},
требаће Вам \TeX{} мотор усклађен са Unicode-ом, попут \XeTeX{}-а или \LuaTeX-а. 

Требаће Вам такође начин куцања Unicode карактера који нису директно доступни са стандардних тастатура. Препоручујемо инсталирање црквенословенског или продуженог руског распореда тастатуре, доступног са \href{http://sci.ponomar.net/keyboard.html}{Slavonic Computing Initiative website}. 
Такође је могуће куцање карактера користећи услугу Мапу знакова Виндоуса или по кодној тачки, но то није препоручљиво.

\subsubsection{ГНУ/Линукс}

Како бисте могли управљати OpenType фонтовима, требало би Ваш систем да има \href{http://freetype.sourceforge.net}{freetype} библиотеку инсталирану и укључену; то се нормално ради подразумевано у свим модерним дистрибуцијама. Требаће Вам програм за обраду текста усклађен са Unicode-ом, попут \href{http://www.libreoffice.org}{LibreOffice}-а. Ако користите \TeX, требаће Вам \TeX{} мотор усклађен са Unicode-ом, попут
\XeTeX{}-а или \LuaTeX-а.

Требаће Вам управљачки програм тастатуре за уношење Unicode карактера. Под ГНУ/Линуксом, тиме управља |m17n| библиотека и база података. Погледајте \href{http://sci.ponomar.net/keyboard.html}{Slavonic Computing Initiative website} за више детаља.

\subsubsection{OS X}

Не знамо.

\subsection{Подручје приватне употребе}

Unicode Подручје приватне употребе (PUA) је скуп трију спектра кодних тачака (од U+E000-а до U+F8FF-а, Раван 15 и Раван 16) које гарантовано никада неће алоцирати карактерима Unicode Consortium: могу их користити трећа лица за алоцирање сопствених карактера. Slavonic Computing Initiative је успоставио индустријски стандард за алокацију карактера у PUA, што је подробно описано у \href{http://www.ponomar.net/files/pua_policy.pdf}{PUA Allocation Policy}-у.

PUA у тим фонтовима садржава разне додатне глифове: контекстуалне алтернативе, стилске алтернативе, лигатуре, глифове хипотетичне и за појединачну намену, разне глифове које нису још кодиране у Unicode-у и разне техничке симболе. Највећи део тих глифова (алтернативне глифове и лигатуре) нормално је приступачан путем \hyperlink{OT}{OpenType} функција. Дакле, уопштено Вам не треба да директно приступите глифовима у PUA-у. Ваљда има неких изузетака:

\begin{itemize}

\item Ако треба да приступите карактерима који још увек нису кодирани у Unicode-у и глифовима за појединачну намену.

\item Ако треба да приступите алтернативним глифовима и лигатурама у застарелим системима који не подржавају OpenType функције.

\item Ако сте компјутерски програмер и треба да радите са глифовима на низом нивоу без ослањања на OpenType-у: имати све алтернативе пресликане на PUA-у омогућава једноставан начин приступање глифовима кодном тачком уместо да радите са индексима глифова, који се могу променити између верзија фонта.

\end{itemize}

\noindent За карактере пресликане у PUA-у и друга техничка разматрања, молимо погледајте \href{https://www.ponomar.net/files/pua_policy.pdf}{Private Use Area Policy}.

\section{OpenType Технологија}
\hypertarget{OT}{}\label{OT}

OpenType је технологија ,,паметних фонтова`` за напредну типографију коју су развили Microsoft Corporation и Adobe Systems и заснива се на TrueType формату фонта. Омогућава исправну типографију у сложеним писмима пак обезбеђује напредне типографске ефекте. Ово се постиже примењујући разне \textit{функције}, или \textit{ознаке}, описане у OpenType спецификацији. Неке од тих функција би требало да се укључе подразумевано, док се друге сматрају опционалним, па их могу укљичити и искључити корисници када то желе.

\subsection{На Мајкрософт Виндоусу}

Како бисте користили те напредне типографске функције, уз ,,паметан`` фонт (као фонтови у том пакету), треба Вам апликација усклађена са OpenType-ом. Не све апликације тренутно подржавају OpenType, па не све апликације које тврде да подржавају OpenType уствари подржавају све функције или обезбеђују интерфејс за приступање функцијама. Старије верзије Microsoft Uniscribe библиотеке нису подржавале OpenType функције за ћирилицу и глагољицу, но почевши од Виндоуса 7, то је решено.

Уопштено говорећи, добићете најбоље резултате у \XeTeX{}-у или \LuaTeX{}-у користећи \texttt{fontspec} пакет или користећи софтвер напредног стоног издаваштва попут Adobe InDesign-а. Највећи део OpenType функција је такође приступачан у Мајкрософт Офису 2010 и каснијим верзијама. И LibreOffice подржава OpenType функције почевши од верзије 4.1, па је подршка за укључивање и искључивање опционалних функција додата у верзији 5.3. Молимо погледајте одељак \hyperlink{LO}{Подршка напредних функција у Libreoffice-у}, у наставку.

\subsection{На ГНУ/Линуксу}

Подршку OpenType-а обезбеђује HarfBuzz библиотека обликовања, која је приступачна преко FreeType-а, део од већине стандардних дистрибуција X Window System-а. Дакле, OpenType ће бити доступан у било којој апликацији која користи FreeType, мада неким апликацијама недостаје интерфејс за укључивање и ускључивање опционалних функција. Уопштено говорећи, добићете најбоље резултате у \XeTeX{}-у или \LuaTeX{}-у користећи \texttt{fontspec} пакет. LibreOffice такође подржава OpenType функције почевши од верзије 4.1, па је подршка за укључивање и искључивање опционалних функција додата у верзији 5.3. Молимо погледајте одељак \hyperlink{LO}{Подршка напредних функција у LibreOffice-у}, у наставку.

\subsection{OpenType функције}

\subsubsection{Позиционирање комбинованих знакова}
\hypertarget{mark}{}

OpenType омогућава паметно позиционирање дијакритпозица: ако куцате слово које следи дијакрит, дијакрит ће се поставити тачно изнад или испод слова; то обезбеђује \texttt{mark} функција. Додатно, \texttt{mkmk} функција се користи за стављање двају знакова у односу на једну другу, тако да додатан дијакрит може бити прописно смешћен испод првог. Ово понашање се демонстрира у наставку:

\begin{figure}[h]
\centering
\begin{tabular}{ll}
\large{  {\slv а}  + {\slv ◌́} → {\slv а́ } } &   \\
\large{ {\slv А}  + {\slv ◌́} → {\slv А́ } } & (позиционирање глифова путем \emph{mark} функције) \\
\large{ {\slv ◌ⷭ} + {\slv  ◌‍҇} → {\slv ◌ⷭ҇ } } & (позиционирање глифова путем \emph{mkmk} функције) \\
\end{tabular}
\end{figure}

Фонтови обезбеђују прописне \texttt{mark} и \texttt{mkmk} полазне тачке за сва ћирилчна и глагољска слова па комбиноване ознаке, омогућавајући Вам да их куцате у готово свој комбинацији (чак и оне које су невероватне). Највећи део OpenType рендерера (осим старијих верзија Adobe Cooltype библиотеке) подржавају ове функције, онда би требало да можете постићи исправно позиционирање у највећем делу апликација усклађених са OpenType-ом (на пример, у MS Word-у 2010 или новијим верзијама, LibreOffice 4.1-у или новијим верзијама, па у \XeTeX{}-у).

\subsubsection{Слагање и разлагање глифова}
\hypertarget{ccmp}{}

Функција счагања / разлагања глифова (\texttt{ccmp}) користи се за слагање двају карактера у један глиф ради боље прераде глифова. Та функција се такође користи за стварање композитних форма основног глифа са дијакритичким знацима када употреба \texttt{mark} и \texttt{mkmk} самих не може постићи потребно позиционирање. Такође се користи за стварање алтернативних облика глифова, попут алтернативних верзија Псилиа коришћених изнад великих слова и скраћених форма слова Ук коришћеног са акценатским знацима, као што се демонстрира у доле наведеним примерима:

\begin{figure}[h]
\centering
\begin{tabular}{ll}
\large{ {\slv ◌҆} } $\rightarrow$ \large { {\slv  ◌ } } & (замена глифова користећи \emph{ccmp} функцију) \\
\large{ {\slv ◌҆}  + {\slv ◌̀} $\rightarrow$ {\slv ◌҆̀} } & (замена глифова користећи \emph{ccmp} функцију) \\
\large{ {\slv т}  + {\slv } + {\slv в} $\rightarrow$ {\slv т‍в } } & (замена глифова користећи \emph{ccmp} функцију) \\
\large{ {\slv ꙋ}  + {\slv ◌ⷯ} $\rightarrow$ {\slv ꙋⷯ } } & (контекстуална замена користећи \emph{ccmp} функцију) \\
\end{tabular}
\end{figure}

Уопштено говорећи, \texttt{ccmp} функција не би требало да се искључи (и често просто не може), па би онда ова функционалност требало да прописно ради у било којој апликацији усклађеној са OpenType-ом. За више детаља о лигатурама, погледајте \href{http://www.unicode.org/notes/tn41/}{Unicode Technical Note \#41: Church Slavonic Typography in Unicode}.

\subsubsection{Функције засноване на језику}

Функције засноване на језику попут \texttt{locl} функције (локализоване форме) обезбеђују приступ алтернативним формама глифова специфичним за појединачни језик, попут алтернативних форма ћирилчиног слова I коришћеног на украјинском и белоруском језику:

\begin{figure}[h]
\centering
\begin{tabular}{ll}
\large{  {\slv і } } &  (црквенословенски текст) \\
\large{ {\slv і̇ } } & (украјински текст) \\
\end{tabular}
\end{figure}

Да се бисте користили тим функцијама, треба Вам апликација усклађена са OpenType-ом која подржава одређивање језика текста, на пример \XeTeX{} или \LuaTeX{} користећи пакете \texttt{fontspec} или \texttt{polyglossia}. Пошто Вам много софтверских апликација не омогућава да одредите црквенословенски као језик текста, подразумевано је да се фонт користи за приказивање црквенословенског текста, а онда сви глифови имају црквенословенски изглед осим ако се не одреди други језик.

LibreOffice Вам омогућава да одредите да је текст на црквенословенском језику почевши од верзије 5.0. То ће Вам омогућити да искористите друге функције, попут црквенословенског растављања на слогове (погледајте
\href{http://sci.ponomar.net/tools.html}{Slavonic Computing Initiative website}
за више информација). Microsoft Corporation не препознаје црквенословенски као важећи језик, онда нећете моћи подесити црквенословенски као језик текста ни у једном Мајкрософтовом производу. \footnote{Молимо немојте конктирати одржаваоце фонтова о овом питању. Уместо тога, жалите се Мајкрософтовом корисничком сервису у Србији на 0700 300 300 или у Црној Гори на 080 081 110.}

\subsubsection{Стилске алтернативе и стилски скупови}

Стилске алтернативе (\texttt{salt} функција) обезбеђују варијантне облике глифова које могу одабрати корисници по вољи. Типично, ово су глифови који разликују од основног глифа само по графичном изгледу где употреба тих глифова не прати ниједна правила заснована на језику или типографији, него је радије једноставно украшавање. На пример, обезбеђују се следеће варијантне форме Симбола за Маркова поглавља U+1F545:

\begin{center}
\begin{tabular}{ccccc}
U+1F545	& \multicolumn{4}{c}{Алтернативни глифови} \\
\hline
{\slv \Huge 🕅} &	\textcolor{gray}{\slv \Huge } & \textcolor{gray}{\slv \Huge } & \textcolor{gray}{\slv \Huge } & \textcolor{gray}{\slv \Huge }  \\
\hline
\end{tabular}
\end{center}

Стилски скупови се користе за укључивање групе стилских варијантних глифова, креираних да се визуелно хармонизују, па им аутоматкси замењују подразумеване форме. OpenType омогућава да се одреде до 20 стилских скупова, означавајући их као функције \texttt{ss01}, \texttt{ss02}, \ldots{} \texttt{ss20}.

Употреба стилских алтернатива и стилских скупова захтева апликацију усклађену са OpenType-ом која обезбеђује интерфејс за искључивање и укључивање напредних функција (пошто се подразумевано те функције искључују). Ово је могуће у \XeTeX{}-у
или \LuaTeX{}-у користећи \texttt{fontspec} пакет и у LibreOffice-у
(почевши од верзије 5.3) употребом посебне синтаксе која приквачује потребну опцију имену фонта. Погледајте одељак \hyperlink{LO}{Подршка напредних функција у LibreOffice-у}, у наставку. У Мајкрософт Офису 2010 и каснијим верзијама, Стилски скупови се могу бити искључити и укључити у оквиру |OpenType функције| на картицу |Више опција| дијалога |Фонт|. Међутим, Мајкрософт Офис Вам не омогућава да истовремено одабрете вишеструке стилске скупове ни да приступите |salt| функцији. Ако је потребно, можете приступити алтернативним глифовима кодном тачком са Подручја приватне употребе (PUA). Међутим, ослањање на PUA-у као механизам размене података не препоручује се.

\subsection{SIL Graphite технологија}

\hypertarget{Graphite}{}\label{Graphite} 

\begin{itemize}
\item[\XeTeXpicfile "deprecated.png" width 4mm] Од верзије 1.3 тог пакета, подршка за \href{https://scripts.sil.org/Graphite}{SIL Graphite} функције је прекинута. Ако Вам треба подршка Graphite-а, погледајте \href{https://github.com/slavonic/fonts-cu-legacy/}{Legacy Fonts package}.
\end{itemize}

\hypertarget{LO}{}\label{LO}

\subsection{Подршка напредних функција у LibreOffice-у}

Подршка OpenType функција је доступна у LibreOffice-у и у свим дериватима OpenOffice.org-а почевши од верзије 3.2 OpenOffice.org-а. Док исправно позиционирање, везивање и замене аутоматски ће радити, раније верзије LibreOffice-а нису имале ниједан механизам за искључивање и укључивање опционалних функција. Подршка искључивања и укључивања Graphite функција доступна је почевши од LibreOffice верзије 4.1. Међутим, нема графичког интерфејса који се може користити. Уместо тога, развијена је синтакса посебног проширеног имена фонта: како бисте активирали опционалну функцију, њен идентификатор, а затим знак једнакости и идентификатор жељене поставке, директно се прилажу нисци имена фонта. Амперсанд се користи за издвајање различитих парова функција/поставки.

На пример, следећи ,,фонт`` би требало да се користи за укључивање |ss01| функцију (Стилски скуп 1):

\begin{verbatim}
Ponomar Unicode:ss01=1
\end{verbatim}

Иста синтакса се користи за искључивање и укључивање опционалних Стилских алтернатива (|salt|), где \texttt{1} указује прву алтернативну глифу, texttt{2} --  другу алтернативну глифу, и тако даље. Запазите да ова функција није доступна у Apache OpenOffice-у; пошто Apache OpenOffice није добро одржаван, сугеришемо да корисници мигрирају у LibreOffice.

Ова функционалност ће бити корисна за кориснике LibreOffice-а што се ослањају на аутоматско растављање на слогове. Пошто LibreOffice нема \href{https://bugs.documentfoundation.org/show_bug.cgi?id=85731}{ниједан механизам за подешавање карактера за растављање на слогове}, Ponomar Unicode и Monomakh Unicode фонтови обезбеђују подвлаку као карактер за растављање на слогове путем Стилског скупа 1 у OpenType-у.

Свакако је директно модификовање фонта веома неугодно, пошто се тешко сетити кратких ознака и бројчаних вредности коришћених за идентификаторе функција/поставки у различитим фонтовима. Нажалост, тренутно нема графичког интерфејса за подржавање исључивања и укључивања OpenType и SIL Graphite функција.

%You may try to install the \href{https://github.com/thanlwinsoft/groooext}
%{Graphite Font Extension}, which provides a dialog for easier feature selection.
%However, this extension has not been maintained since the passing of its
%developer in 2011, and so may not work correctly in later versions of LibreOffice.
%If you experience problems with Graphite features, you may get better
%results accessing glyphs directly by codepoint from the Private Use Area,
%though this is not recommended.

\section{Фонтови за синодални словенски језик}

\subsection{Ponomar Unicode}

Ponomar Unicode је фонт који репродукује словни облик синодалних црквенословенских издања из почетка 20. века. Намењен је за обрађивање модерних црквенословенских текстова (синодални црквенословенски језик). Ponomar Unicode се заснива на Hirmos UCS фонту који је креирао Влад Дорош, али га је модификовао творац овог пакета. Примери текста сложеног по Ponomar Unicode-у наведени су у наставку.

%\lineskip=0pt
%\lineskiplimit=-5em

\begin{churchslavonic}
Бл҃же́нъ мꙋ́жъ, и҆́же не и҆́де на совѣ́тъ нечести́выхъ, и҆ на пꙋтѝ грѣ́шныхъ не ста̀, и҆ на сѣда́лищи гꙋби́телей не сѣ́де: но въ зако́нѣ гдⷭ҇ни во́лѧ є҆гѡ̀, и҆ въ зако́нѣ є҆гѡ̀ поꙋчи́тсѧ де́нь и҆ но́щь. И҆ бꙋ́детъ ꙗ҆́кѡ дре́во насажде́ное при и҆схо́дищихъ во́дъ, є҆́же пло́дъ сво́й да́стъ во вре́мѧ своѐ, и҆ ли́стъ є҆гѡ̀ не ѿпаде́тъ: и҆ всѧ̑, є҆ли̑ка а҆́ще твори́тъ, ᲂу҆спѣ́етъ. Не та́кѡ нечести́вїи, не та́кѡ: но ꙗ҆́кѡ пра́хъ, є҆го́же возмета́етъ вѣ́тръ ѿ лица̀ землѝ. Сегѡ̀ ра́ди не воскре́снꙋтъ нечести́вїи на сꙋ́дъ, нижѐ грѣ̑шницы въ совѣ́тъ првⷣныхъ. Ꙗ҆́кѡ вѣ́сть гдⷭ҇ь пꙋ́ть првⷣныхъ, и҆ пꙋ́ть нечести́выхъ поги́бнетъ.
\end{churchslavonic}

\textbf{Кијевски црквенословенски језик} користи неколико варијантних форма глифова, попут U+1C81 Дугоногог Де-а ({\slv ᲁ}) и U+A641 Варијантног Зе-а ({\slv ꙁ}):

\begin{churchslavonic}
Бл҃же́нъ мꙋ́жъ, и҆́же не и҆́ᲁе на совѣ́тъ нечести́выхъ, и҆ на пꙋтѝ грѣ́шныхъ не ста̀, и҆ на сѣᲁа́лищи гꙋби́телей не сѣ́ᲁе: но въ зако́нѣ гᲁⷭ҇ни во́лѧ є҆гѡ̀, и҆ въ зако́нѣ є҆гѡ̀ поꙋчи́тсѧ де́нь и҆ но́щь. И҆ бꙋ́ᲁетъ ꙗ҆́кѡ дре́во насажᲁе́ное при и҆схо́ᲁищихъ во́ᲁъ, є҆́же пло́ᲁъ сво́й да́стъ во вре́мѧ своѐ, и҆ ли́стъ є҆гѡ̀ не ѿпаᲁе́тъ: и҆ всѧ̑, є҆ли̑ка а҆́ще твори́тъ, ᲂу҆спѣ́етъ. Не та́кѡ нечести́вїи, не та́кѡ: но ꙗ҆́кѡ пра́хъ, є҆го́же воꙁмета́етъ вѣ́тръ ѿ лица̀ землѝ. Сегѡ̀ ра́ᲁи не воскре́снꙋтъ нечести́вїи на сꙋ́ᲁъ, нижѐ грѣ̑шницы въ совѣ́тъ првⷣныхъ. Ꙗ҆́кѡ вѣ́сть гᲁⷭ҇ь пꙋ́ть првⷣныхъ, и҆ пꙋ́ть нечести́выхъ поги́бнетъ.
\end{churchslavonic}

\textbf{Други језици} Ponomar Unicode фонт се може користити такође за слагање литургијских текстова на другим језицима који користе црквенску ћириличну азбуку. Три такви примери у потпуности подржава фонт: румунски језик (молдавски) на његовој ћирилчној азбуци, алеутски јеик (наречје Лисичијих острва или источни) на његовој азбуци, па јакутски језик (Саха) као што је написан на азбуци коју је владика Дионисије (Хитров) створио.

\noindent Ево примера Оченаша на румунској (молдавској) ћирилици: \\

\begin{churchslavonic}
Та́тъль но́стрꙋ ка́реле є҆́щй ꙟ҆ Че́рюрй: ᲃ︀фн҃цѣ́скъсе Нꙋ́меле тъ́ꙋ: ві́е ꙟ҆пъръці́ѧ та̀: фі́е во́ѧ та̀, прекꙋ́мь ꙟ҆ Че́рю̆ шѝ пре пъмѫ́нть. Пѫ́йнѣ но́астръ чѣ̀ ᲁепꙋ́рꙋрѣ ᲁъ́не но́аѡ а҆́стъꙁй. Шѝ не ꙗ҆́ртъ но́аѡ греша́леле но́астре, прекꙋ́мь шѝ но́й є҆ртъ́мь греши́цилѡрь но́щри. Ши́ нꙋ́не ᲁꙋ́че пре но́й ꙟ҆ и҆спи́тъ. Чѝ не и҆ꙁбъвѣ́ще ᲁе че́ль ръ́ꙋ. \\
\end{churchslavonic}

\noindent А ево примера Оченаша на алеутској ћирилици: \\ 

\begin{churchslavonic}
Тꙋмани́нъ А́даԟъ! А҆́манъ акꙋ́х̑тхинъ и́нинъ кꙋ́ҥинъ, А́са́нъ амчꙋг̑а́сѧ́да́г̑та, Аҥали́нъ а҆ԟа́г̑та, Анꙋхтана́тхинъ малга́г̑танъ и́нимъ кꙋ́ганъ ка́юхъ та́намъ кꙋ́ганъ. Ԟалга́дамъ анꙋхтана̀ ҥи̑нъ аԟача́ ꙋ̆а҆ѧ́мъ: ка́юхъ тꙋма́нинъ а́д̑ꙋнъ ҥи̑нъ игни́да, а҆ма́кꙋнъ тꙋ́манъ ка́юхъ малгалиги́нъ ҥи̑нъ ад̑ꙋг̑и́нанъ игнида́кꙋнъ: ка́юхъ тꙋ́манъ сꙋглатачх̑и́г̑анах̑тхинъ, та́г̑а ад̑алю́дамъ илѧ́нъ тꙋ́манъ аг̑г̑ича. \\
\end{churchslavonic}

\noindent А ево примера Оченаша на јакутском језику (Саха): \\

\begin{churchslavonic}
Халланнаръ юрдюлѧригѧрь баръ агабытъ бисенѧ ! Свѧтейдѧннинь а̄тыҥъ эенѧ ; кѧллинь царстваҥъ эенѧ ; сирь юрдюгѧрь кёҥюлюҥь эенѧ , халланъ юрдюгѧрь курдукъ боллунъ ; бюгюҥю кюннѧги асыръ аспытынъ бисенинь кулу бисеха бюгюнь ; бисиги да естѧрбитинь халларъ бисеха , хайтахъ бисиги да халларабытъ беэбить естѧхтѧрбитигѧрь ; килѧримѧ да бисигини альԫархайга ; хата быса бисигини албынтанъ . \\
\end{churchslavonic}

\subsubsection{Напредне функције фонта}

Ponomar Unicode ставља неке карактере у Подручје приватне употребе (PUA). За опште PUA пресликавање, молимо погледајте \href{http://www.ponomar.net/files/pua_policy.pdf}{PUA Allocation Policy}.

Уз општа пресликавања PUA-а, неки карактери су додељени одељку отвореног спектра PUA-а. То су:

\begin{itemize}
\item U+F400 \textendash{} Алтернативе за глифове Допусне вишејезичне равни (SMP): овај одељак садржава копије у Основној вишејезичкој равни (BMP) за подршку у застарелим апликацијама. Тренутно, следећи су доступни: U+F400 - U+F405 \textendash{} Типиконови симболи (копије од U+1F540 до U+1F545).
\item U+F410 \textendash{} Форме представљања: садржавају разне форме представљања и лигатуре које унутрашње користи фонт. Уопште, ово нису намeњене да их позову корисници или спољне апликације.
\item U+F420 \textendash{} Језичке алтернативе: садржава алтернативне облике глифова који су специфични за појединачни језик. За тренутак, ово су модерни интерпункцијски облици за употребу са латинским карактерима. То нису намењени да се споља позову.
\item U+F441 и даље \textendash{} стилске алтернативе латинских карактера (готичке форме). Ово се могу позвати путем Стилског скупа 2, но, ако је потребно, могу се директно позвати са PUA-а. Оне се пресликавају у истом редоследу као и у Основном латинском блоку, почевши од U+F441-а (што одговара на U+0041 Латинско велико слово A). Уз репертоар Основног латинског, имамо и: U+F4DE \textendash{} Готички Торн; U+F4FE \textendash{} Мало слово готички Торн; па U+F575 \textendash{} Готички дуги S.
\end{itemize}

Фонт обезбеђује неколико лигатура, које се праве уводећи Ознаку форматирања без ширине без прелома (U+200D) између два карактера. Списак лигатура је наведен у табели~\ref{ligs1}.

\begin{table}[htbp]
\centering
\caption{Лигатуре доступне у Ponomar Unicode-у \label{ligs1}}
\begin{tabular}{lcc}
Име	& Секвенца	& Изглед \\
\hline
Лигатура А-У	& U+0430 U+200D U+0443	& {\slv{\large а‍у}}	\\
Лигатура Ел-У	& U+043B U+200D U+0443 & {\slv{\large л‍у}}	\\
Лигатура Те-Ве	& U+0442 U+200D U+0432	& {\slv{\large т‍в}}	\\
\hline
\end{tabular}
\end{table}

\noindent У OpenType-у, дефинише се неколико стилских алтернатива. Оне су наведене у Табели~\ref{salt1}. Уз додатне глифове за Симбол за Марково поглавље, функција обезбеђује алтернативне украсне форме слова U+0423 У које изгледа тачно као U+A64A Ук (ова употреба се налази у неким публикацијама), па алтернативну форму за U+0404 Широки Је за употребу у контекстима где треба да се разликује од U+0415 Је-а (понајвише за украјински текст по црквенословенског стилу).

\newfontfamily{\salt}[Alternate=0]{Ponomar Unicode}
\newfontfamily{\salta}[Alternate=1]{Ponomar Unicode}
\newfontfamily{\saltb}[Alternate=2]{Ponomar Unicode}
\newfontfamily{\saltc}[Alternate=3]{Ponomar Unicode}
\newfontfamily{\saltd}[Alternate=4]{Ponomar Unicode}
\newfontfamily{\salte}[Alternate=5]{Ponomar Unicode}
\newfontfamily{\saltf}[Alternate=6]{Ponomar Unicode}
\newfontfamily{\saltg}[Alternate=7]{Ponomar Unicode}

\begin{table}[htbp]
\centering
\caption{Стилске алтернативе у Ponomar Unicode-у \label{salt1}}
\begin{tabular}{lccccc}
	& Основна форма	& \multicolumn{4}{c}{Алтернативне форме} \\
\hline
U+1F545	& {\slv{\large 🕅 }}	& {\salt\large 🕅} & {\salta\large 🕅} & {\saltb\large 🕅} & {\saltc\large 🕅}  \\
			&				& {\saltd\large 🕅} & {\salte\large 🕅} & {\saltf\large 🕅} & {\saltg\large 🕅} \\
U+0423		& {\slv\large У}	& {\salt\large У} \\
U+040E		& {\slv\large Ў}	& {\salt\large Ў} \\
U+0404		& {\slv\large Є}	& {\salt\large Є} \\
\hline
\end{tabular}
\end{table}

За ћирилчина слова, функција стилских алтернатива омогућава такође приступ скраћеним формама; редослед алтернативних форма је увек: ниже скраћивање, горње скраћивање, лево скраћивање, десно скраћивање. Табела~\ref{trunc} демонстрира које скраћене форме јесу доступне. Уопштено говорећи, скраћивањем би требало да аутоматски управља софтвер за стоно издаваштво и \TeX{}, мада је то тешко постићи.

\begin{table}[htbp]
\centering
\caption{Скраћене форме приступачне путем функције Стилских алтернатива
у Ponomar Unicode-у \label{trunc}}
\begin{tabular}{lccccc}
	& Основна форма	& \multicolumn{4}{c}{Скраћене форме} \\
\hline
U+0440	& {\slv{\large р }}	& {\salt\large р} &  \\
U+0443  & {\slv{\large у }}	& {\salt\large у} &  \\
U+0444  & {\slv{\large ф }}	& {\salt\large ф} & {\salta\large ф} \\
U+0445  & {\slv{\large х}}	& {\salt\large х} & {\salta\large х} & {\saltb\large х}  \\
U+0446  & {\slv{\large ц }}	& {\salt\large ц} &  \\
U+0449  & {\slv{\large щ }}	& {\salt\large щ} &  \\
U+0471  & {\slv{\large ѱ }}	& {\salt\large ѱ} &  {\salta\large ѱ}\\
U+A641  & {\slv{\large ꙁ }}	& {\salt\large ꙁ} &  \\
U+A64B  & {\slv{\large ꙋ }}	& {\salt\large ꙋ} & {\salta\large ꙋ} & {\saltb\large ꙋ} \\
\hline
\end{tabular}
\end{table}

Стилски скуп 1 (|ss01|) се обезбеђује као привремено заобилазно решење за \href{https://bugs.documentfoundation.org/show_bug.cgi?id=85731}{LibreOffice Bug 85731}-у, што Вам не омогућава да одредите карактер за растављање на слогове у LibreOffice-у. Када се укључи, замењује сва појављивања U+002D-а Цртице-минуса и U+2010-а Цртице U+005F-ом Ниском линијом (подвлаком) за употребу као карактер за растављање на слогове. Молимо запазите да ће се ова функција напустити чим се потребна функционалност буде додала LibreOffice-у.

Дефинише се и Стилски скуп 2 (,,ss02``), Готичке форме. Када се тај стилски скуп укључи, латинска слова се појављују на готици насупрот њиховим модерним формама. То је корисно ради слагања латинског текста поред црквенословенског у неким контекстима. Погледајте следећи пример:

\newfontfamily{\blackletter}[StylisticSet=2]{Ponomar Unicode}

\begin{figure}[h]
\centering
\begin{tabular}{ll}
Regular &
{\slv \large The quick brown fox. 1234567890. А҆ сїѐ по слове́нски. } \\
Blackletter & 
{\blackletter \large The quick brown fox. 1234567890. А҆ сїѐ по слове́нски. } \\
\end{tabular}
\end{figure}

\noindent Запазите да од верзије 2.0 фонта, ASCII цифре (обично називане
,,арапски бројеви``) обезбеђују се на латинској форми. Користите Стилски скуп 2 за приступање готичким формама, ако је потребно.

\section{Фонтови за праниконски штампани словенски језик}

\subsection{Fedorovsk Unicode}

Fedorovsk Unicode се заснива на Fedorovsk фонту који је креирао Никита Симонс.
Поново се кодирао за Unicode, са OpenType функцијама које је додао Александр Андрејев. Fedorovsk фонт намерава да репродукује фонт штампованих издања Ивана Фјодорова објављених у Москви, на пример, Апостол из 1564. Фонт је првенствено намењен за слагање праниконских (старообредствичких) литургијских текстова или за обређивање таквих текстова у академском контексту.

\newfontfamily{\rightFedor}[StylisticSet=1,HyphenChar="200B]{Fedorovsk Unicode}

Ево примера из Апостола Ивана Фјодорова 1564. године

\begin{churchslavonic}
{\Large \rightFedor
\textcolor{red}{П}е́рвᲂе ᲂу҆́бо︀ сло́во︀ сᲂтвᲂри́хъ о҆ всѣ́хъ , ѽ , ѳео҆́филе , о҆ ниⷯже начѧ́тъ і︮с︯ , твᲂри́тиже и҆ ᲂу҆чи́ти . д︀о︀ него́же дн҃е , запᲂвѣ́д︀авъ а҆пⷭ҇лᲂмъ дх҃ᲂмъ ст҃ыⷨ , и҆́хже и҆ꙁбра̀ вᲂзнесе́сѧ . преⷣ ни́миже и҆ пᲂста́ви себѐ жи́ва по страд︀а́нїи свᲂе҆́мъ . во︀ мно́зехъ и҆́стинныхъ зна́менїи҆хъ . дн҃ьми четы́ридесѧтьми ꙗ҆влѧ́ꙗсѧ и҆́мъ и҆ гл҃ѧ ꙗ҆́же о҆ црⷭ҇твїи бж҃їи . сни́миже и҆ ꙗ҆д︀ы̀ , пᲂвелѣва́ше и҆́мъ ѿ і҆е҆рᲂсали́ма не ѿлꙋча́тисѧ . но̑ жда́ти о҆бѣтᲂва́нїе ѿч︮е︯е , е҆́же слы́шасте ѿ́ менѐ . ꙗ҆́кѡ і҆ѡ҃а́ннъ ᲂу҆́бо︀ крⷭ҇ти́лъ е҆́сть вᲂдо́ю . вы́же и҆́мате крести́тисѧ дх҃ᲂмъ ст҃ы́мъ , не по мно́ꙁѣхъ си́хъ д︀︮н︯еⷯ .
}
\end{churchslavonic}

Ево примера из Цветног триодиона 1648. године

\newfontfamily{\leftFedor}[StylisticSet=2,HyphenChar="200B]{Fedorovsk Unicode}

\begin{churchslavonic}
{\Large \leftFedor
\textcolor{red}{стⷯры па́сцѣ . гла́съ , є҃ .} Д\textcolor{red}{а вᲂскрⷭ҇нетъ бг҃ъ ,꙳ и҆ разы́дꙋтсѧ вразѝ є҆гѡ̀ .}
Па́сха сщ҃е́ннаѧ на́мъ дне́сь пᲂказа́сѧ , па́сха но́ва ст҃а́ѧ , па́сха таи́нственнаѧ , па́сха всечестна́ѧ , па́сха хрⷭ҇та̀ и҆зба́вителѧ , па́сха непᲂро́чнаѧ , па́сха вели́каѧ , па́сха вѣ́рнымъ , па́сха двѣ́ри ра́йскїѧ на́мъ ѿверза́ющаѧ , па́сха всѣ́хъ ѡ҆сщ҃а́ющаѧ вѣ́рныхъ .
}
\end{churchslavonic}

\subsubsection{Напредне функције фонта}

Фонт обезбеђује неколико лигатура, које се праве уводећи Ознаку форматирања без ширине без прелома (U+200D) између два карактера. Списак лигатура је наведен у Табели~\ref{ligs2}.

\begin{table}[htbp]
\centering
\caption{Лигатуре доступне у Fedorovsk Unicode-у \label{ligs2}}
\begin{tabular}{lcc}
Име	& Секвенца	& Изглед \\
\hline
Лигатура А-У	& U+0430 U+200D U+0443	& {\leftFedor{\large а‍у}}	\\
Лигатура Ел-У	& U+043B U+200D U+0443 & {\leftFedor{\large л‍у}}	\\
Лигатура А-Ижица & U+0430 U+200D U+0475	& {\leftFedor{\large а‍ѵ}}	\\
Лигатура Ел-Ижица & U+043B U+200D U+075 & {\leftFedor{\large л‍ѵ}}	\\
Лигатура Те-Ве	& U+0442 U+200D U+0432	& {\leftFedor{\large т‍в}}	\\
Лигатура Ер-Јат	& U+0440 U+200D U+0463 & {\leftFedor{\large р‍ѣ}}	\\
\hline
\end{tabular}
\end{table}

\noindent У OpenType-у, дефинише се неколико стилских алтернатива. Наведене су у Табели~\ref{salt2}. Уз обезбеђивање алтернативних облика глифова за U+1F545 Симбол за Марково поглавље, оне Вам омогућавају да контролишете позиционирање дијакритичких накова изнад извесних слова.

\newfontfamily{\glyphfont}{Fedorovsk Unicode}
\newfontfamily{\saltFedor}[Alternate=0]{Fedorovsk Unicode}
\newfontfamily{\saltaFedor}[Alternate=1]{Fedorovsk Unicode}
\newfontfamily{\saltbFedor}[Alternate=2]{Fedorovsk Unicode}
\newfontfamily{\saltcFedor}[Alternate=3]{Fedorovsk Unicode}
\newfontfamily{\saltdFedor}[Alternate=4]{Fedorovsk Unicode}
\newfontfamily{\salteFedor}[Alternate=5]{Fedorovsk Unicode}
\newfontfamily{\saltfFedor}[Alternate=6]{Fedorovsk Unicode}

\begin{table}[htbp]
\centering
\caption{Стилске алтернативе у Fedorovsk Unicode-у \label{salt2}}
\begin{tabular}{lcccccccc}
	& Основна форма	& \multicolumn{7}{c}{Алтернативне форме} \\
\hline
U+0404	& {\glyphfont{\large Є}} & {\saltFedor\large Є} \\
U+0426	& {\glyphfont{\large Ц}} & {\saltFedor\large Ц} \\
U+0491	& {\glyphfont{\large ґ}} & {\saltFedor\large ґ} \\
U+A64C	& {\glyphfont{\large Ꙍ}} & {\saltFedor\large Ꙍ} \\
U+047C	& {\glyphfont{\large Ѽ}} & {\saltFedor\large Ѽ} \\
U+047E	& {\glyphfont{\large Ѿ}} & {\saltFedor\large Ѿ} \\
U+047F	& {\glyphfont{\large ѿ}} & {\saltFedor\large ѿ} \\
U+1F545	& {\glyphfont{\large 🕅 }}	& {\saltFedor\large 🕅} & {\saltaFedor\large 🕅} & {\saltbFedor\large 🕅} & {\saltcFedor\large 🕅}  & {\saltdFedor\large 🕅} & {\salteFedor\large 🕅} & {\saltfFedor\large 🕅} \\
U+0463 U+0486	& {\glyphfont{\large ѣ҆}} & {\saltFedor\large ѣ҆}  \\
U+0463 U+0300	& {\glyphfont{\large ѣ̀}} & {\saltFedor\large ѣ̀} & {\saltaFedor\large ѣ̀} \\
U+0463 U+0301	& {\glyphfont{\large ѣ́}} & {\saltFedor\large ѣ́} & {\saltaFedor\large ѣ́} \\
U+0463 U+0311	& {\glyphfont{\large ѣ̑}} & {\saltFedor\large ѣ̑} & {\saltaFedor\large ѣ̑} \\
U+0463 U+0486 U+0301	& {\glyphfont{\large ѣ҆́}} & {\saltFedor\large ѣ҆́}  \\
U+A64B U+0486	& {\glyphfont{\large ꙋ҆}} & {\saltFedor\large ꙋ҆}  \\
U+A64B U+0300	& {\glyphfont{\large ꙋ̀}} & {\saltFedor\large ꙋ̀} & {\saltaFedor\large ꙋ̀} \\
U+A64B U+0301	& {\glyphfont{\large ꙋ́}} & {\saltFedor\large ꙋ́} & {\saltaFedor\large ꙋ́} \\
U+A64B U+0311	& {\glyphfont{\large ꙋ̑}} & {\saltFedor\large ꙋ̑} & {\saltaFedor\large ꙋ̑} & {\saltbFedor\large ꙋ̑} \\
U+A64B U+0486 U+0301	& {\glyphfont{\large ꙋ҆́}} & {\saltFedor\large ꙋ҆́}  \\
\hline
\end{tabular}
\end{table}

Додатно, три стилска скупа су се дефинисала у фонту. Стилски скуп 1 (,,Нагласци на десну страну``) позиционира нагласке изнад Јата и Ука на десној страни и Стилски скуп 2 (,,Нагласци на леву страну``) позиционира нагласке изнад Јата и Ука на левој страни. Ти стилски скупови су корисни када текст користи један од тих позиционирања свуда. Стилски скуп 10 (,,Варијанте са једнаком основном линијом``) поставља велика слова на исту основну линију као и мала слова (корисно за рад са фонтом у академском контексту где основна линија великих слова која је традиционално спуштена може изазвати проблеме са вертикалним простором када се обради текст који је било на латиници било на ћирилици). Ево примера:

\newfontfamily{\base}[StylisticSet=10]{Fedorovsk Unicode}

\begin{figure}[h]
\centering
\begin{tabular}{ll}
{\large \glyphfont Хрⷭ҇то́съ вᲂскр҃се и҆з̾ ме́ртвыхъ} & (редовни текст) \\
{\large \base Хрⷭ҇то́съ вᲂскр҃се и҆з̾ ме́ртвыхъ} & (Стилски скуп 10 укључен) \\
\end{tabular}
\end{figure}

\section{Фонтови за обрађивање древних рукописа}

\subsection{Menaion Unicode}

Menaion фонт се намерава користити за обрађивање текста рукописа из уставског доба. Садржава пуни репертоар потребних ћирилчних и глагољских глифова као и глифове византијске екфонетске нотације онакве се користе у ћирилчним или глагољским рукописима.

Menaion фонт је првобитно креирао Виктор A. Баранов на \href{http://www.manuscripts.ru/}{the Manuscript Project}-у. Поново га је кодирао за Unicode Александр Андрејев уз дозволу првобитног твораца.

\newfontfamily{\menaion}[HyphenChar="200B]{Menaion Unicode}

Узроци текста у Menaion Unicode-у се представљају у фигурама~\ref{men1}
и \ref{men2}. Молимо запазите да су комбинована глагољска слова (Глагољска допуна) постала доступним у Unicode-у 9.0. У старијим верзијама Мајкрософтових софтвера, ваљда исправно позиционирање глифова за ове карактере користећи OpenType функције неће бити могуће. Да постигнете жељени ефекат препоручујемо да користите LibreOffice, \XeTeX{}, \LuaTeX{}, или софтвере напредног стоног издаваштва као и Adobe InDesign.

\begin{figure}[htbp]
\centering
\caption{Ћирилчни текст из Остромировог јеванђеља (XI. век) \label{men1}}
\begin{tabular}{lr}
 1& {\Large \menaion    Искони бѣ слово } \\
 2& {\Large \menaion    и слово бѣ отъ  } \\
 3& {\Large \menaion   б҃а и б҃ъ бѣ} \\ 
 4& {\Large \menaion    слово  𝀏̃  се бѣ} \\ 
 5& {\Large \menaion    искони оу} \\ 
 6& {\Large \menaion    б҃а  ⁘  и тѣмь в̇са бꙑ} \\ 
 7& {\Large \menaion    шѧ  𝀏̃  и беꙁ него ни} \\ 
 8& {\Large \menaion    чьтоже не бꙑсть  ·} \\ 
 9& {\Large \menaion   ѥже бꙑсть  𝀏̃  въ то} \\ 
10& {\Large \menaion    мь животъ бѣ  ·  и} \\ 
 1& {\Large \menaion    животъ бѣ свѣтъ} \\ 
 2& {\Large \menaion    чловѣкомъ  𝀏̃  и свѣ} \\ 
 3& {\Large \menaion    тъ въ тьмѣ свьти} \\ 
 4& {\Large \menaion    тьсѧ  ·  и тьма ѥго} \\ 
 5& {\Large \menaion    не обѧтъ  𝀏̃  бꙑсть} \\ 
 6& {\Large \menaion    члв҃къ посъланъ} \\ 
 7& {\Large \menaion    отъ б҃а  ·  имѧ ѥмоу} \\ 
 8& {\Large \menaion    иоанъ  𝀏̃  тъ приде} \\ 
 9& {\Large \menaion    въ съвѣдѣтель} \\ 
10& {\Large \menaion    ство  ·  да съвѣдѣте} \\ 
2.2  1& {\Large \menaion    льствоуѥть о свѣ} \\ 
 2& {\Large \menaion    тѣ  𝀏̃  да вьси вѣрѫ} \\ 
 3& {\Large \menaion    имѫть имь  ⁘  не бѣ} \\ 
 4& {\Large \menaion    тъ свѣтъ  ⁘  нъ да} \\ 
 5& {\Large \menaion    съвѣдѣтельствоу} \\ 
 6& {\Large \menaion    ѥть о свѣтѣ  𝀏̃̑ бѣ} \\ 
 7& {\Large \menaion    свѣтъ истиньнꙑ} \\ 
 8& {\Large \menaion    и  ·  иже просвѣщаѥ} \\ 
 9& {\Large \menaion    ть в́сꙗкого чл҃ка  ⸴} \\ 
10& {\Large \menaion   грѧдѫща въ миръ  𝀏̃̑} \\ 
\end{tabular}
\end{figure}

\begin{figure}[htbp]
\centering
\caption{Глагољски текст из Асеманијевог јеванђеља (XI. век) \label{men2}}
\begin{tabular}{lr}
1 & {\Large \menaion   ⁘ ⰅⰂⰀ𞀌҇   ⰙⰕ҇   ⰋⰉ҇Ⱁ } \\
 2 & {\Large \menaion  Ⰻⱄⰽⱁⱀⰹ ⰱⱑ } \\
 3 & {\Large \menaion       ⱄⰾⱁⰲⱁ  · } \\
 4 & {\Large \menaion      ⰻ ⱄⰾⱁⰲⱁ } \\
 5 & {\Large \menaion       ⰱⱑ ⱋ̔ ⰱⰰ  · } \\
 6 & {\Large \menaion      ⰻ ⰱ͞ⱏ ⰱⱑ } \\
 7 & {\Large \menaion      ⱄⰾⱁⰲⱁ  · } \\
 8 & {\Large \menaion   Ⱄⰵ ⰱⱑ ⰻ̔ⱄⰽⱁ} \\
 9 & {\Large \menaion     ⱀⰻ  ·  ⱋ̔ ⰱ꙯ⰰ  ·  ⰲⱐ} \\
10 & {\Large \menaion     ⱄⱑ ⱅⱑⰿⱏ ⰱⱏⰻ} \\
11 & {\Large \menaion     ⱎⱔ  ·  Ⰻ̔ ⰱⰵⰶ ⱀⰵⰳⱁ } \\
12 & {\Large \menaion     ⱀⰹⱍⰵⱄⱁⰶⰵ } \\
13 & {\Large \menaion     ⱀⰵ ⰱⱏⰻⱄⱅⱏ  ·  ⰵ̔} \\
14 & {\Large \menaion     ⰶⰵ ⰱⱏⱄⱅⱏ  · } \\
15 & {\Large \menaion    Ⰲⱏ ⱅⱁⰿⱏ ⰶⰹⰲⱁ} \\
16 & {\Large \menaion     ⱅⱏ ⰱⱑ  ·  ⰻ ⰶⰹⰲⱁ} \\
17 & {\Large \menaion     ⱅⱏ ⰱⱑ ⱄⰲⱑⱅⱏ } \\
18 & {\Large \menaion     ⱍⰾ҃ⰽⰿⱏ  ·  ⰻ̔ ⱄⰲⱁⱑ } \\
19 & {\Large \menaion     ⰲⱏ ⱅⱐⰿⱑ ⱄⰲⱏ} \\
20 & {\Large \menaion     ⱅⰹⱅⱏ ⱄⱔ  ·  ⰻ ⱅⱐ} \\
21 & {\Large \menaion     ⰿⰰ ⰵ̔ⰳⱁ ⱀⰵ ⱁ̔ⰱⱔⱅ } \\
\end{tabular}
\end{figure}

\subsubsection{Напредне функције фонта}

Фонт обезбеђује неколико лигатура, које се праве уводећи Ознаку форматирања без ширине без прелома (U+200D) између два карактера. Списак лигатура је наводен у Табели~\ref{menligs}.

\begin{table}[htbp]
\centering
\caption{Лигатуре доступне у Menaion Unicode фонту \label{menligs}}
\begin{tabular}{lcc}
Име	& Секвенца	& Изглед \\
\hline
Мала лигатура И-Је &	U+0438 U+200D U+0435 	& {\menaion{\large и‍е }} \\
Мала лигатура Ен-И	&	U+043d U+200D U+0438 	& {\menaion{\large н‍и }} \\
Мала лигатура Ен-Мали јус	& U+043d U+200D U+0467 	& {\menaion{\large н‍ѧ }} \\
Мала лигатура Ес-Ве	&	U+0441 U+200D U+0432 	& {\menaion{\large с‍в }} \\
Мала лигатура Те-Ер	&	U+0442 U+200D U+0440 	& {\menaion{\large т‍р }} \\
Велика лигатура А-У	& 	U+0410 U+200D U+0423 	& {\menaion{\large А‍У }} \\
Мала лигатура А-У	&	U+0430 U+200D U+0443 	& {\menaion{\large а‍у }} \\
Мала лигатура А-Те		&	U+0430 U+200D U+0442 	& {\menaion{\large а‍т }} \\
Велика лигатура И-Је	&	U+0418 U+200D U+0415 	& {\menaion{\large И‍Е }} \\
Велика лигатура Ел-Ге		&	U+041b U+200D U+0413 	& {\menaion{\large Л‍Г }} \\
Мала лигатура Ел-Ге		&	U+043b U+200D U+0433 	& {\menaion{\large л‍г }} \\
Велика лигатура Ен-И	&	U+041d U+200D U+0418 	& {\menaion{\large Н‍И }} \\
Велика лигатура Ен-Мали јус	&	U+041d U+200D U+0466 	& {\menaion{\large Н‍Ѧ }} \\
Велика лигатура Ес-Ве		&	U+0421 U+200D U+0412 	& {\menaion{\large С‍В }} \\
Мала лигатура Те-Јат		&	U+0442 U+200D U+0463 	& {\menaion{\large т‍ѣ }} \\
Велика лигатура Те-Ве	&	U+0422 U+200D U+0412	& {\menaion{\large Т‍В }} \\
Мала лигатура Те-Ве		&	U+0442 U+200D U+0432	& {\menaion{\large т‍в }} \\
Велика лигатура Те-И		&	U+0422 U+200D U+0418 	& {\menaion{\large Т‍И }} \\
Мала лигатура Те-И		&	U+0442 U+200D U+0438 	& {\menaion{\large т‍и }} \\
Велика лигатура Те-Ер		&	U+0422 U+200D U+0420 	& {\menaion{\large Т‍Р }} \\
Лигатура Велико А-Мало Те	&	U+0410 U+200D U+0442 	& {\menaion{\large А‍т }} \\
Велика лигатура Те-Меки знак	&	U+0422 U+200D U+042c 	& {\menaion{\large Т‍Ь }} \\
Мала лигатура Те-Меки знак	&	U+0442 U+200D U+044C 	& ‍{\menaion{\large т‍ь }} \\
Мала лигатура Те-А		&	U+0442 U+200D U+0430 	& {\menaion{\large т‍а }} \\
\hline
\end{tabular}
\end{table}

\section{Фонтови за академски рад}

\subsection{Monomakh Unicode}

Monomakh Unicode се заснива на Monomachus фонту који је креирао Алексеј Крјуков. Модификовао се уз дозволу. Monomakh Unicode је ћирилични фонт имплементиран у мешовитом уставском / полууставском стилу и намењен за задовољавање потреба истраживања која се баве словенском историјом и филологијом. Укључује све историјске ћирилчне карактере који се тренутно дефинишу у Unicode-у; фонт укључује и скуп латинских слова креираних да буду стилско компатибилни са ћирилчним делом. То је ваљда корисно ради слагања двојезичних издања на црквенословенском језику и језицима написаним на латиннском писму, посебно они који користе много дијакритичких знакова, као и у румунском примеру у наставку.

\newfontfamily{\monomakh}{Monomakh Unicode}

\begin{tabular}{p{2.25in}p{0.02in}p{2.25in}}
\begin{churchslavonic}
{\monomakh Бл҃же́нъ мꙋ́жъ, и҆́же не и҆́де на совѣ́тъ нечести́выхъ, и҆ на пꙋтѝ грѣ́шныхъ не ста̀, и҆ на сѣда́лищи гꙋби́телей не сѣ́де: но въ зако́нѣ гдⷭ҇ни во́лѧ є҆гѡ̀, и҆ въ зако́нѣ є҆гѡ̀ поꙋчи́тсѧ де́нь и҆ но́щь. И҆ бꙋ́детъ ꙗ҆́кѡ дре́во насажде́ное при и҆схо́дищихъ во́дъ, є҆́же пло́дъ сво́й да́стъ во вре́мѧ своѐ.}
\end{churchslavonic}
& &
\begin{romanian}
{\monomakh Fericit bărbatul, care n-a umblat în sfatul necredincioșilor și în calea păcătoșilor nu a stat și pe scaunul hulitorilor n-a șezut; ci în legea Domnului e voia lui și la legea Lui va cugeta ziua și noaptea. și va fi ca un pom răsădit lângă izvoarele apelor, care rodul său va da la vremea sa.}
\end{romanian}
\end{tabular}

\subsubsection{Напредне функције фонта}

Monomakh фонт пружа неколико опционалних OpenType функција које могу укључити или искључити корисници. То су:

\begin{itemize}
\item Стилски скуп 1 (\emph{ss01}) се обезбеђује као привремено решење  \href{https://bugs.documentfoundation.org/show_bug.cgi?id=85731}{LibreOffice Bug 85731}-а, који Вам не омогућава да одредите карактер растављање на слогове у LibreOffice-у. Када се укључи, замењује сва појављивања U+002D-а Цртице-минуса и U+2010-а Цртице U+005F-ом Ниском цртом (подвлаком) за употребу као карактер растављања на слогове. Молимо запазите да ће се ова функција напустити чим се потребна фунционалност буде додала LibreOffice-у.
\item Стилски скуп 6 (\emph{ss06}) приказује U+0456 Ћирилчино мало слово украјинско / белоруско И са једном горњом тачком и Стилски скуп 7 (\emph{ss07}) приказује исти карактер са две горње тачке. Подразумевано, U+0456 се приказује без тачки.
\item Стилски скуп 8 (\emph{ss08}) приказује карактере U+0417 Ћирилчино велико слово Зе и U+0437 Ћирилчино мало слово Зе као ,,оштру земљу``, т.ј., као карактере U+A640 Ћирилчино велико слово Земља, односно U+A641 Ћирилчино мало слово Земља. Уопште, ова промена би требало да се направи на нивоу кодне тачке, онда се коришћење ове функције не препоручује.
\item Стилски скуп 9 (\emph{ss09}) приказује карактере U+0427-а Ћирилчино велико слово Че и U+0447-а Ћирилчино мало слово Че у њиховој архаичној форми, са доњим продужетком слова на средини (нпр., {\fontspec{Monomakh Unicode}[StylisticSet=9] ч} уместо {\glyphfont ч}).
\item Стилски скуп 10 (\emph{ss10}) приказује карактере U+0429 Ћирилчино велико слово Шча и U+0449 Ћирилчино мало слово Шча у њиховој модерној форми, са доњим пордужетком слова на десној (нпр., {\fontspec{Monomakh Unicode}[StylisticSet=10] щ} уместо {\glyphfont щ}).
\item Стилски скуп 11 (\emph{ss11}) приказује карактере U+044B Ћирилчино велико слово Јеру и U+A651 Ћирилчино мало слово Јеру са Дебелим јером са два глифа повезана (нпр., {\fontspec{Monomakh Unicode}[StylisticSet=11] ы} уместо {\glyphfont ы}).
\item Стилски скуп 13 (\emph{ss13}) приказује карактер U+0463 Ћирилчино мало слово Јат са левим стаблом продуженим до основне линије (нпр., као {\fontspec{Monomakh Unicode}[StylisticSet=13] ѣ}). Молимо запазите да то није исто као и U+A653 Ћирилчино мало слово јотован Јат.
\item Иста функционалност тих стилских скупова такође обезбеђује у OpenType-у функција Стилске алтернативе (\emph{salt}).
\item[\XeTeXpicfile "deprecated.png" width 4mm] Претходне верзије фонта су обезбеђивале Стилски скуп 1 (\emph{ss01}), који је приказивао U+015E Латинско велико слово S са седиљом, U+0162 Латинско велико слово T са седиљом, па њихове мале еквивалентне, као U+0218 Латинско велико слово S са доњом запетом, U+021A Латинско велико слово T са доњом запетом, па њихове мале еквивалентне. Међутим, пошто се коришћење U+015E-а, U+0162-а и њихових малих еквивалената ради кодирања румунског текста сматра погрешним, ова функција се напушта. Корисници се снажно подстичу да претворе свој текст на нивоу кодне тачке да би користили исправне карактере за румунску ортографију. Међутим, ради компатибилности са текстом који је погрешно кодиран, та је функција још увек доступна.
\item[\XeTeXpicfile "deprecated.png" width 4mm] Стилски скуп 15 (\emph{ss15}), који обезбеђује комбинована ћирилчна слова са аутоматским \emph{покритијем} где то санкционише синодална ортографија такође се напушта и може се уклонити. Корисници би требало да експлицитно кодирају \emph{покритије} као U+0487 Комбиновано ћирилчно покритије. Погледајте \href{http://www.unicode.org/notes/tn41/}{UTN 41: Church Slavonic Typography in Unicode} за више информација.
\end{itemize}

Две додатне функције су биле доступне само у SIL Graphite-у; међутим је подршка SIL Graphite-а прекинута. Ако Вам требају те функције, погледајте \href{https://github.com/slavonic/fonts-cu-legacy/}{Legacy Fonts package}. 
\begin{itemize}
\item[\XeTeXpicfile "deprecated.png" width 4mm] Graphite функција ,,Convert Arabic Digits to Church Slavonic`` (\emph{cnum}), када се укључи, аутоматски ће приказати западне цифре (,,арапске бројеве``) као ћириличне бројеве. Ово помаже, на пример, за нумерисање страница у софтверу који не подржава ћириличне бројеве.
\item[\XeTeXpicfile "deprecated.png" width 4mm] Graphite функција ,,Convert HIP-6B Keystrokes to Church Slavonic Characters`` (\emph{hipb}), када се укључи, приказаће текст кодиран у застарелој HIP кодној страници као црквенословенски. Коришћење те функције се не препоручује и корисници се уместо тога подстичу да претворе текст кодиран по HIP-у на Unicode.
\end{itemize}

\section{Украсни фонтови}

\subsection{Indiction Unicode}

Indiction Unicode фонт репродукује украсни стил иницијала који користе синодална црквенословенска издања из касних 1800-их.

Првобитни Indyction фонт је развио Владислав В. Дорош и дистрибуише се
под именом Indyction UCS као део CSLTeX-а, који се лиценира под Јавном лиценцом \LaTeX{} пројекта. Фонт је поново кодирао за Unicode и уредио Александр Андрејев, па се сада
дистрибуише као Indiction Unicode под SIL Јавном лиценцом фонтова. Намењен је за коришћење са \emph{буквицима} (иницијалима) у модерним црквенословенским издањима.

\begin{churchslavonic}
\cuLettrine Бл҃же́нъ мꙋ́жъ, и҆́же не и҆́де на совѣ́тъ нечести́выхъ, и҆ на пꙋтѝ грѣ́шныхъ не ста̀, и҆ на сѣда́лищи гꙋби́телей не сѣ́де: но въ зако́нѣ гдⷭ҇ни во́лѧ є҆гѡ̀, и҆ въ зако́нѣ є҆гѡ̀ поꙋчи́тсѧ де́нь и҆ но́щь. И҆ бꙋ́детъ ꙗ҆́кѡ дре́во насажде́ное при и҆схо́дищихъ во́дъ, є҆́же пло́дъ сво́й да́стъ во вре́мѧ своѐ, и҆ ли́стъ є҆гѡ̀ не ѿпаде́тъ: и҆ всѧ̑, є҆ли̑ка а҆́ще твори́тъ, ᲂу҆спѣ́етъ.
\par
\end{churchslavonic}

\subsection{Vertograd Unicode}

Vertograd Unicode (заснован на Vertograd UCS-у Влада Дороша) други је фонт за украсне иницијале и наслове. Фонт се уопште користио у предреволуционарним издањима синодалног црквенословенског језика. Запазите да нека слова која се типично не појављују у великој форми нису доступна. Молимо поднесите питање ако Вам треба слово које није доступно.

\renewcommand{\LettrineFontHook}{\vertograd \cuKinovarColor}
\begin{churchslavonic}
\cuLettrine Бл҃же́нъ мꙋ́жъ, и҆́же не и҆́де на совѣ́тъ нечести́выхъ, и҆ на пꙋтѝ грѣ́шныхъ не ста̀, и҆ на сѣда́лищи гꙋби́телей не сѣ́де: но въ зако́нѣ гдⷭ҇ни во́лѧ є҆гѡ̀, и҆ въ зако́нѣ є҆гѡ̀ поꙋчи́тсѧ де́нь и҆ но́щь. И҆ бꙋ́детъ ꙗ҆́кѡ дре́во насажде́ное при и҆схо́дищихъ во́дъ, є҆́же пло́дъ сво́й да́стъ во вре́мѧ своѐ, и҆ ли́стъ є҆гѡ̀ не ѿпаде́тъ: и҆ всѧ̑, є҆ли̑ка а҆́ще твори́тъ, ᲂу҆спѣ́етъ.
\par
\end{churchslavonic}

\subsection{Cathisma Unicode}

Cathisma Unicode се заснива на Kathisma UCS-у, који је креирао Влад Дорош. Фонт се користи за наслове у много литургијских издања XVIII.--XX. века.

\newfontfamily{\cathisma}{Cathisma Unicode}
\begin{center}
\begin{tabular}{c}
{\fontsize{42pt}{2em} \cathisma ЧИ́НЪ ОУ҆́ТРЕНИ ВСЕНО́ЩНАГѠ БДѢ́НЇѦ} \\
\end{tabular}
\end{center}

\subsection{Oglavie Unicode}

Oglavie Unicode се заснива на Oglavie UCS-у, који је креирао Влад Дорош. Фонт се такође користи за украсне наслове у много литургијских издања XVIII.--XX. века.

\newfontfamily{\oglavie}{Oglavie Unicode}
\begin{center}
\begin{tabular}{c}
{\fontsize{42pt}{2em} \oglavie ТРЇѠ́ДЬ НО́ТНАГѠ ПѢ́НЇѦ} \\
\end{tabular}
\end{center}

\subsection{Pomorsky Unicode}

Pomorsky Unicode фонт је верна репродукција украсног краснописног стила књига и наслова поглавља, који су највероватније развили у 1700-им писари старообредничког Виговског манастира. Он се често види у песничким рукописима, литургијским рукописима,
хагиографским и полемичким делима Поморцијских и Федосејевских заједница, па је традиционалан и ,,органски`` стил утиснутог натписа којем недостаје икакав очигледан утицај из западноевропске и латинске тупографије. Pomorsky фонт је првобитно креирао Никита Симонс. Намењен је за употребу са \emph{буквицима} (иницијалима) и украсним насловима.

\newfontfamily{\pomorsky}{Pomorsky Unicode}
\newfontfamily{\simple}[StylisticSet=1]{Pomorsky Unicode}
\newfontfamily{\pomorskysalt}[Alternate=0]{Pomorsky Unicode}
\newfontfamily{\pomorskysalta}[Alternate=1]{Pomorsky Unicode}

\begin{center}
\begin{tabular}{c}
{\fontsize{48pt}{2em} \pomorsky ЧИ́НЪ ВЕЧЕ́РНИ.} \\
\end{tabular}
\end{center}

\subsubsection{Напредне функције фонта}

Неколико верзија много глифова се обезбеђује у фонту. Украшене форме слова су подразумеване и обезбеђују се на кодним тачкама ћирилчних великих слова; она би требало да се користе што је више могуће. Једноставније форме се могу користити кад год словима треба мање украшен изглед, или када би се диајакрити могли сукобити са украшавањем (или када се украшање једног карактера сукоби са украшавањем другог); те једноставне форме су доступне као
\verb+Stylistic Set 1+. Има неколико додатних карактера који су стилске варијанте, што се обезбеђује као стилске алтернативе (\verb+salt+). Пошто је фонт намењен за иницијале и наслове, карактери малих слова нису доступни. Основне форме, ,,једноставне`` форме, па се сваке стилске алтернативе карактера демонстрирају у Табели~\ref{pomor}.

\begin{table}[htbp]
\centering
\caption{Облици карактера обезбеђени Pomorsky Unicode-ом \label{pomor}}
{\fontsize{38pt}{1.5em}
\begin{tabular}{cccc}
	{\pomorsky А}{\simple А}{\pomorskysalt А}	& {\pomorsky Б}{\simple Б} & {\pomorsky В}{\simple В} & {\pomorsky Г}{\simple Г} \\

	{\pomorsky Е}{\simple Е}	& {\pomorsky Ж}{\simple Ж} & {\pomorsky Ѕ}{\simple Ѕ} & {\pomorsky З}{\simple З} \\
	
	{\pomorsky И}{\simple И}	& {\pomorsky Й}{\simple Й} & {\pomorsky І}{\simple І} & {\pomorsky Ї}{\simple Ї} \\

	{\pomorsky К}{\simple К}{\pomorskysalt К}{\pomorskysalta К}	& {\pomorsky Л}{\simple Л} & {\pomorsky М}{\simple М} & {\pomorsky Н}{\simple Н} \\

	{\pomorsky О}{\simple О}	& {\pomorsky Ѻ}{\simple Ѻ} & {\pomorsky П}{\simple П} & {\pomorsky Р}{\simple Р}{\pomorskysalt Р}{\pomorskysalta Р} \\

	{\pomorsky С}{\simple С}	& {\pomorsky Т}{\simple Т} & {\pomorsky ОУ}{\simple ОУ} & {\pomorsky Ꙋ}{\simple Ꙋ} \\

	{\pomorsky Ф}{\simple Ф}	& {\pomorsky Х}{\simple Х} & {\pomorsky Ѡ}{\simple Ѡ} & {\pomorsky Ѽ}{\simple Ѽ} \\

	{\pomorsky Ѿ}{\simple Ѿ}	& {\pomorsky Ц}{\simple Ц} & {\pomorsky Ч}{\simple Ч} & {\pomorsky Ш}{\simple Ш} \\

	{\pomorsky Щ}{\simple Щ}	& {\pomorsky Ъ}{\simple Ъ} & {\pomorsky Ы}{\simple Ы} & {\pomorsky Ь}{\simple Ь} \\

	{\pomorsky Ѣ}{\simple Ѣ}	& {\pomorsky Ю}{\simple Ю} & {\pomorsky Ꙗ}{\simple Ꙗ}{\pomorskysalt Ꙗ} & {\pomorsky Ѧ}{\simple Ѧ} \\

	{\pomorsky Ѯ}{\simple Ѯ}	& {\pomorsky Ѱ}{\simple Ѱ} & {\pomorsky Ѳ}{\simple Ѳ} & {\pomorsky Ѵ}{\simple Ѵ} \\
\end{tabular}
}
\end{table}

\section{Технички фонтови: FiraSlav}

ФираСлав је једносезонски фонт за црквенословенски језик,
намењен уређивању црквенословенског текста у уређивачу текста.
Сви дијакритички знакови и комбинована слова представљени су као размачни симболи
и задржаван је једностаран изглед.
Фонт укључује и обичне (\verb+FiraSlav Regular+) и подебљане (\verb+FiraSlav Bold+).
Посебно је корисно за софтвер и апликације за развој веб локација:

\begin{center}
\begin{tabular}{l}
{\firaslav  \$number =~ /{\textasciicircum}(?:҂([\$h]))*(?:҂([\$o]))*([\$h]?)([клмнѯопч]?)([\$o]?)\$/; } \\
{\firaslav  var letter = '(?:ᲂу{\textbar}Оу{\textbar}оу{\textbackslash}{\textbackslash}S)[̀́̑҆̾̏҇҃ⷠⷡⷢⷣⷷⷤⷥꙵꙶⷦ]*'; } \\
{\firaslav Бл҃же́нъ мꙋ́жъ, и҆́же не и҆́де на совѣ́тъ нечести́выхъ} \\
\end{tabular}
\end{center}

\section{Познати проблеми}

Погледајте \href{https://github.com/typiconman/fonts-cu/issues/}{Issue Tracker}. Пре него што пријавите проблеме, молимо проверите да Ваш софтвер прописно подржава OpenType. Сугеришемо да проверите очекивано понашање у \XeTeX{}-у или \LuaTeX{}-у.

\section{Признања}

Творци би желели да захвале следеће људе:

\begin{itemize}

\item Владислав Дорош, који је дозволио да се његов фонт \href{http://irmologion.ru/fonts.html}{Hirmos} поново кодира на Unicode и модификује, што је довело до стварања Ponomar фонта.

\item Виктор Баранов \href{http://www.manuscripts.ru/}{Manuscripts} пројекта, који је дозволио поновно кодирање и модификацију свог Menaion фонта.

\item Михаел Иванович због његове помоћи у креирању карактера за Саха
(јакутски језик), делимично преузетих из његовог фонта Sakha UCS.

\item Алексеј Крјуков, који је одговорио на разна питања о FontForge-у,
дозволио да се његов фонт Monomachus модификује и препакира,
па чија обимна документација за \href{https://github.com/akryukov/oldstand/}{Old Standard} фонт се прочитала и делимично поново употребила.

\item Мајк Крутиков, који је саставио \TeX{} пакет фонтова.

\item Алесандро Ческини, који је превео документацију у српскохрватски језик.

\end{itemize}

\end{document}
